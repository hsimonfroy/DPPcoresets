

\chapter{Improving concentration with DPP}
\section{A quantitative result on variance}


\cite{bardenet2021sgddpp} show the existence of a sequence of DPP kernels $(\opekernel{m})$, independent of $\threequery$, whose induced estimator has asymptotic variance $\OO( m ^{-(1+\frac 1 d)})$. More precisely, 

\begin{tcolorbox}
	\begin{theorem}[From \cite{bardenet2021sgddpp}]
		\label{thm_sgdpaper}
		Let $m\in \NN$ and $\mathcal{S} \sim  \mathcal{DPP}(\opekernel{m})$.
		\begin{equation}
			\estloss{\mathcal{S}}{\threequery} := \sum_{x \in \mathcal{S}} \frac{\threequery(x)}{\opekernel{m}(x,x)}.
		\end{equation}
		Equation (S14) yields that 
		\begin{equation}
			\Var{}{\estloss{\mathcal{S}}{\threequery}} = \lipschitz^2_\threequery \OO( m ^{-(1+\frac 1 d)}) +\OO( n^{-1/2}).
		\end{equation}
		where $\lipschitz_\threequery := \operatorname{Lip}\left\{\frac{m\threequery}{K^{(m)}_{q, \tilde \gamma}} \right\}$ is the Lipschitz constant of $x \mapsto \frac{m\threequery(x)}{K^{(m)}_{q, \tilde \gamma}(x,x)}$.
	\end{theorem}
\end{tcolorbox}
\note{}{
	explain how the estimator is built\\
	$\mathcal{X} \subseteq \RR^d$\\
	bounded condition?
}



We first recall some classical notations.
\begin{itemize}
	\item We denote by $\empdistr{\mathcal{S}}{}:=\frac{1}{|\mathcal{S}|} \sum_{x \in \mathcal{S}}$ the empirical measure based on sample $\mathcal{S}  \subseteq \mathcal{X}$. 
	\item Hence for all function $\threequery$, we denote by $\empdistr{\mathcal{S}}{\threequery}:=\int_{\mathcal{X}} \threequery(x) d\empdistr{\mathcal{S}}{x} = \frac{1}{|\mathcal{S}|} \sum_{x \in \mathcal{S}} \threequery(x)$ the expectation of $\threequery$ with respect to $\empdistr{\mathcal{S}}{}$. Furthermore, given a distribution $\PP{}{}$ on $\mathcal{S}$, we denote by $\meanempdistr{\threequery}:=\EE{}{\empdistr{\mathcal{S}}{\threequery}}$, its expectation with respect to $\PP{}{}$. 
	\item Finally, the induced $L^1(\empdistr{\mathcal{S}}{})$ distance between two functions $\threequery$ and $\threequery'$ would be denoted by $\dlone{\empdistr{\mathcal{S}}{}}{\threequery}{\threequery'}:=\empdistr{\mathcal{S}}{|\threequery - \threequery'|}$.
\end{itemize}  


We can then reformulate the result from Theorem \ref{thm_sgdpaper} into the following corollary.
\begin{tcolorbox}
	\begin{corollary}
		\label{cor_sgdpaper}
		Let $m\in \NN$ and $\mathcal{S} \sim  \mathcal{DPP}(\opekernel{m})$.
		For all function $\twoquery$, we have
		\begin{equation*}
			\Var{}{\empdistr{\mathcal{S}}{\twoquery}} = \lipschitz^2_\twoquery \OO( m ^{-(1+\frac 1 d)}) +\OO( n^{-1/2})
		\end{equation*}
		where $\lipschitz_\twoquery := \operatorname{Lip}\left\{\frac{\opekernel{m}}{K^{(m)}_{q, \tilde \gamma}} \twoquery \right\}$ is the Lipschitz constant of $x \mapsto\frac{\opekernel{m}(x,x)}{K^{(m)}_{q, \tilde \gamma}(x,x)} \twoquery(x) $.
	\end{corollary}
\end{tcolorbox}

\begin{proof}
	We simply apply Theorem \ref{thm_sgdpaper} with $\threequery = \frac{\opekernel{m}g}{m}$, such that
	\begin{align*}
		\Var{}{\empdistr{\mathcal{S}}{\twoquery}} = \Var{}{\estloss{\mathcal{S}}{\threequery}} = \lipschitz^2_\threequery \OO( m ^{-(1+\frac 1 d)}) +\OO( n^{-1/2}),
	\end{align*}
	where $\lipschitz_\threequery$ is the Lipschitz constant of $x \mapsto \frac{m\threequery(x)}{K^{(m)}_{q, \tilde \gamma}(x,x)} = \frac{\opekernel{m}(x,x)}{K^{(m)}_{q, \tilde \gamma}(x,x)} \twoquery(x) $.

\end{proof}





\section{Concentration for fixed query}
\subsection{Main assumption}

Let $\qset$ be the function space on which we want the coreset property to hold.
Then let define the following function space defined on $\mathcal{X}$
		\begin{equation}
		\twoqset_m := \frac{m\qset}{\opekernel{m}\loss{\qset}} = \left\{x \mapsto \frac{m\query(x)}{\opekernel{m}(x,x)\loss{\query}} \mid \query \in \qset\right\}.
	\end{equation}
We have that for any $\query \in \qset$, we have $g := \frac{m\query}{\opekernel{m}\loss{\query}} \in \twoqset_m$ such that
	\begin{equation}
		\label{def_ftog}
		\frac{\estloss{\mathcal{S}}{\query}}{\loss{\query}} = \empdistr{\mathcal{S}}{g}
		\quad \text{ and }\quad 
		\meanempdistr{g} = \EE{}{\empdistr{\mathcal{S}}{g}} = \frac{\EE{}{\estloss{\mathcal{S}}{\query}}}{\loss{\query}} = 1.
	\end{equation}




In the following sections, we crucially assume the functions sets $\twoqset_m$ verify boundedness in Lipschitz constant, infinite norm, and pseudo-dimension.
Formally, we assume it exists positive reals $\lipschitz, \bound \in \RR_+$ and natural integer $\pdim \in \NN$, that all three only depends on $\qset$, such that for all $m \in \NN$
\begin{itemize}
	\label{asm_gm}
	\item  $\sup_{\twoquery \in \twoqset_m} \operatorname{Lip}\left\{\frac{\opekernel{m}}{K^{(m)}_{q, \tilde \gamma}}\twoquery \right\} =: \lipschitz < +\infty$
	\item $\sup_{\twoquery \in \twoqset_m} \|\twoquery\|_{\infty} =: \bound  < +\infty$
	\item $\operatorname{pdim}\twoqset_m \leq \pdim$
\end{itemize}
We further discuss to what extend these assumptions are relevant.
\note{}{discuss}




\subsection{Chebyshov bound}

Let define for any $\epsilon>0$ and $m\in \NN$, the Chebyshov concentration bound
\begin{equation}
	\label{def_boundratecheb}
	\boundrate_{\textrm{Cheb}}(\epsilon, m) := \frac{1}{\epsilon^2} \sup_{\twoqset \in \{\twoqset_m \mid m \in \NN\}} \sup_{\twoquery \in \twoqset} \Var{}{\empdistr{\mathcal{S}}{\twoquery}}.
\end{equation}
We know from Corollary \ref{cor_sgdpaper} and assumption on Lipschitz constant boundedness \ref{asm_gm} that $\boundrate_{\textrm{Cheb}}(\epsilon, m) = \frac {1} {\epsilon^2}\left(\lipschitz^2 \OO( m ^{-(1+\frac 1 d)}) +\OO( n^{-1/2})\right)$.



\begin{tcolorbox}
	\begin{theorem}[Chebyshov bound for fixed query]
		\label{thm_chebfixedtheta} 
		Let $m\in \NN$ and $\mathcal{S} \sim  \mathcal{DPP}(\opekernel{m})$. 

		Then for all $\epsilon >0$ and all $\query \in \qset$
		\begin{equation*}
			\PP{}{\dnude{\nu}{\estloss{\mathcal{S}}{\query}}{\loss{\query}}>\epsilon\loss{\query}} \leq \boundrate_{\textrm{Cheb}}(\epsilon, m)
		\end{equation*}
		
		
		Moreover, for all $\delta>0$ and for $n$ sufficiently large
		\begin{equation*}
			m \gtrsim \left(\frac{\lipschitz^2}{\delta\epsilon^2} \right)^{\frac{1}{1+\frac 1 d}}
			\implies 
			\text{$\mathcal{S}$ is $1-\delta$-surely an $\epsilon$-coreset for $\query$ }
		\end{equation*}
	\end{theorem}
\end{tcolorbox}





\begin{proof}
	Let $\query \in \qset$, thus we know $g := \frac{m\query}{\opekernel{m}\loss{\query}} \in \twoqset_m$ verifies \ref{def_ftog}.

	Then applying Bienaym\'e-Chebyshov inequality on the probability of $\mathcal{S}$ being an $\epsilon$-coreset for $\query$ yields 
	\begin{align*}
		\PP{}{\dnude{\nu}{\estloss{\mathcal{S}}{\query}}{\loss{\query}}>\epsilon\loss{\query}}
		&= \PP{}{\dnude{\nu}{\frac{\estloss{\mathcal{S}}{\query}}{\loss{\query}}}{1}>\epsilon}
		= \PP{}{\dnude{\nu}{\empdistr{\mathcal{S}}{\twoquery}}{\meanempdistr{\twoquery}}>\epsilon}\\ 
		&\leq \frac{1}{\varepsilon ^{2}}\Var{}{\empdistr{\mathcal{S}}{\twoquery}}\\
		&\leq \boundrate_{\textrm{Cheb}}(\epsilon, m)
		=\frac {1} {\epsilon^2}\left(\lipschitz^2 \OO( m ^{-(1+\frac 1 d)}) +\OO( n^{-1/2})\right)
	\end{align*}
	Hence, a sufficient condition for $\mathcal{S} \sim \mathcal{DPP}(\opekernel{m})$ to satisfy $1-\delta$-surely the $\epsilon$-coreset for $\query$  property \ref{eqn_querycoresetprop} is to have
	\begin{gather*}
		\boundrate_{\textrm{Cheb}}(\epsilon, m) \leq \delta 
		\iff
		m^{1+\frac 1 d} \gtrsim \frac{\lipschitz^2}{\delta \epsilon^2 + \OO(n^{-1/2})} = \frac {\lipschitz^2} {\delta\epsilon^2} \frac{1}{1 + \frac{1}{\delta \epsilon^2}\OO(n^{-1/2})}.
	\end{gather*} 
	Then this means that for sufficiently large $n$ (potentially $n\gtrsim \delta^{-2} \epsilon^{-4}$), we can control the second factor and thus obtain the bound
	\begin{equation*}
		m \gtrsim \left(\frac{\lipschitz^2}{\delta\epsilon^2} \right)^{\frac{1}{1+\frac 1 d}}.
	\end{equation*}
\end{proof}









\subsection{Breuer and Duits bound}

% \begin{tcolorbox}
%     \begin{theorem}[From \cite{breuer2013nevai}]
%         $$\PP{}{\left|X_f-\mathbb{E} X_f\right| > \epsilon} \leq \begin{cases}2 \exp \left(-\frac{\epsilon^2}{4 A \Var{}{X_f}}\right), & \text { if } \epsilon<\frac{2 A \Var{}{X_f}{3\|f\|_{\infty}} \\ 2 \exp \left(-\frac{\epsilon}{6\|f\|_{\infty}}\right), & \text { if } \epsilon \geq \frac{2 A \Var{}{X_f}}{3\|f\|_{\infty}}\end{cases}$$
% \end{tcolorbox}


\begin{tcolorbox}
	\begin{theorem}[From \cite{breuer2013nevai}]
		\label{thm_breuer}
		Let $\epsilon>0$, any bounded function $\threequery$, any projective DPP kernel $K$, and let $\mathcal{S} \sim  \mathcal{DPP}(K)$.\\

		Then for any linear statistics $X_\threequery := \sum_{x\in\mathcal{S}}\query(x)$
		\begin{equation*}
			\PP{}{\dnude{}{X_f}{\EE{}{X_\threequery}} > \epsilon} \leq	
			\begin{cases}
				2 \exp \left(-\frac{\epsilon^2}{4 A \Var{}{X_\threequery}}\right) 
				& \text { if } \epsilon<\frac{2 A \Var{}{X_\threequery}}{3\|\threequery\|_{\infty}}\\
				2 \exp \left(-\frac{\epsilon}{6\|\threequery\|_{\infty}}\right) 
				& \text { otherwise }
			\end{cases}
		\end{equation*}	
		where $A \simeq 7819$ and thus does not depend on $\threequery$, $K$ or $\epsilon$.
	\end{theorem}
\end{tcolorbox}

Let define for any $\epsilon>0$ and $m\in \NN$, 
\begin{equation*}
	\boundrate_{\textrm{BD}}(\epsilon,m) := 2 \exp \left( -\left(\max\left\{
			4 A \boundrate_{\textrm{Cheb}}(\epsilon, m),\
			\frac{6 \bound}{\epsilon m}\right\}\right)^{-1}.
			\right)
\end{equation*}










\begin{tcolorbox}
	\begin{theorem}[Breuer and Duits bound for fixed query]
		\label{thm_breuerfixedtheta}
		Let $m\in \NN$ and $\mathcal{S} \sim  \mathcal{DPP}(\opekernel{m})$. 

		Then for all $\epsilon >0$ and all $\query \in \qset$
		\begin{equation*}
			\PP{}{\dnude{\nu}{\estloss{\mathcal{S}}{\query}}{\loss{\query}}>\epsilon\loss{\query}} \leq \boundrate_{\textrm{BD}}(\epsilon, m)
		\end{equation*}
		
		Moreover, for all $\delta>0$ and for $n$ sufficiently large
		\begin{equation*}
			m \gtrsim \bound \left(\frac{\lipschitz^2}{\epsilon^2} \log  \frac{2}{\delta }\right)^{\frac{1}{1+\frac 1 d}}
			\implies 
			\text{$\mathcal{S}$ is $1-\delta$-surely an $\epsilon$-coreset for $\query$ }
		\end{equation*}
	\end{theorem}
\end{tcolorbox}





\begin{proof}
	Let $\query \in \qset$, thus we know $g := \frac{m\query}{\opekernel{m}\loss{\query}} \in \twoqset_m$ verifies \ref{def_ftog}.

	Then we apply the Breuer and Duits bound \ref{thm_breuer} taking $h = g/m$, such that $X_h = \empdistr{\mathcal{S}}{g}$.	Introducing $\epsilon_{\twoquery, m} := \frac{2 A m \Var{}{\empdistr{\mathcal{S}}{\twoquery}}}{3\|\twoquery \|_{\infty}}$ for conciseness, this gives
	\begin{align*}
		\PP{}{\dnude{\nu}{\estloss{\mathcal{S}}{\query}}{\loss{\query}}>\epsilon\loss{\query}}
		&= \PP{}{\dnude{\nu}{\empdistr{\mathcal{S}}{\twoquery}}{\meanempdistr{\twoquery}}>\epsilon}\\ 
		&\leq 2 \exp \left(
		\begin{cases}
			\frac{-\epsilon^2}{4 A \Var{}{\empdistr{\mathcal{S}}{\twoquery}}}
			& \text { if } \epsilon<\epsilon_{\twoquery, m}\\
			\frac{-\epsilon m}{6\|\twoquery\|_{\infty}} 
			& \text { otherwise }
		\end{cases}
		\right)\\
		&\leq 2 \exp \left(
			\begin{cases}
				\left(-4 A \boundrate_{\textrm{Cheb}}(\epsilon, m)\right)^{-1}
				& \text { if } \epsilon<\epsilon_{\twoquery, m}\\
				\frac{-\epsilon m}{6 \bound} 
				& \text { otherwise }
			\end{cases}
			\right)\\
			&\leq 2 \exp \left(
				\max\left\{
				\left(-4 A \boundrate_{\textrm{Cheb}}(\epsilon, m)\right)^{-1},\
				\frac{-\epsilon m}{6 \bound} \right\}
				\right)
		= \boundrate_{\textrm{Breuer}}(\epsilon,m)
	\end{align*}
	where we used the definition \ref{def_boundratecheb} of $\boundrate_{\textrm{Cheb}}$ and the assumption \ref{asm_gm} that $\|\twoquery\|_{\infty} \leq \bound$.
	
	
	Hence, a sufficient condition for $\mathcal{S} \sim \mathcal{DPP}(\opekernel{m})$ to satisfy $1-\delta$-surely the $\epsilon$-coreset for $\query$  property \ref{eqn_querycoresetprop} is to have
	\begin{gather*}
		\boundrate_{\textrm{Breuer}}(\epsilon,m) \leq \delta
		\iff 
		\left(\log \frac{2}{\delta }\right)^{-1} \geq
		\max\left\{
			4 A \boundrate_{\textrm{Cheb}}(\epsilon, m),\
			\frac{6 \bound}{\epsilon m}\right\}
		\\
		\iff \left(\log \frac{2}{\delta }\right)^{-1} \text{ is greater than both }
		4 A \boundrate_{\textrm{Cheb}}(\epsilon, m)
		\text{ and }
		\frac{6 \bound}{\epsilon m}
	\end{gather*}
	For the first inequality, we know from Theorem \ref{thm_chebfixedtheta} a sufficient condition is that for $n$ sufficiently large 
	\begin{equation*}
		m \gtrsim \left(\frac{\lipschitz^2}{\epsilon^2} \log  \frac{2}{\delta } \right)^{\frac{1}{1+\frac 1 d}}.
	\end{equation*}
	For the second inequality, a sufficient condition is that
	\begin{equation*}
		m \geq \frac{6 \bound}{\epsilon} \log  \frac{2}{\delta }.
	\end{equation*}
	Combining this two cases, a sufficient condition is given by
	\begin{equation*}
		m \gtrsim \bound\left(\frac{\lipschitz^2}{\epsilon^2} \log  \frac{2}{\delta } \right)^{\frac{1}{1+\frac 1 d}}.
	\end{equation*}

\end{proof}
	\note{}{discuss the second case}
		 
		 



\section{Extension to all queries}
\label{sec_extension_all_queries}
In order to obtain an $\epsilon$-coreset for $\qset$, the $\epsilon$-coreset for $\query$ property \ref{eqn_querycoresetprop} must holds simultaneously for all queries $\query \in \qset$.









\begin{tcolorbox}
	\begin{theorem}[Infinite union bound]
		\label{thm_infi_union_bound}
		Let $\threeqset \subseteq [0,\bound]^{\mathcal{X}}$ be a set of bounded functions defined on a base set $\mathcal{X}$, with $d'_{\threeqset} := \operatorname{pdim}\threeqset$ its pseudo-dimension. Let moreover $(\PP{m}{})_{m\in \NN}$ be a sequence of distributions supported on $\binom{\mathcal{X}}{m}$.\\

		Assume it exists a bounding function $\boundrate$ such that for all $\epsilon >0$, function $\threequery \in \threeqset$, integer $m \in \NN$, and multiset $\mathcal{S} \sim \PP{m}{}$, we have the bound
		\begin{equation}
			\PP{m}{\dnude{\nu}{\empdistr{\mathcal{S}}{\threequery}}{\meanempdistr{\threequery}} > \epsilon} \leq \boundrate(\epsilon, m)
		\end{equation}

		Then for all $\epsilon >0$ and all $m \in \NN$ such that $\boundrate(\epsilon/2, m) \leq 1/2$, it holds
		\begin{equation}
			\PP{m}{\exists \threequery \in \threeqset,\ \dnude{\nu}{\empdistr{\mathcal{S}}{\threequery}}{\meanempdistr{\threequery}} > \epsilon} \leq 
			8\left(\frac{8e\bound}{\epsilon}\right)^{2d'_{\threeqset}} \boundrate(\epsilon/16, m)
		\end{equation}
	\end{theorem}
\end{tcolorbox}


We delay the proof of Theorem \ref{thm_infi_union_bound}

\begin{tcolorbox}
	\begin{theorem}[Breuer and Duits bound for all queries]
		\label{thm_breuerallqueries}
		Let $m\in \NN$ and $\mathcal{S} \sim  \mathcal{DPP}(\opekernel{m})$. 

		Then for all $\epsilon \in ]0,\bound]$ and all $\query \in \qset$
		\begin{equation*}
			\PP{}{\exists \query \in \qset,\ \dnude{\nu}{\estloss{\mathcal{S}}{\query}}{\loss{\query}}>\epsilon\loss{\query}} 
			\leq 
			8\left(\frac{8e\bound}{\epsilon}\right)^{2\pdim}   \boundrate_{\textrm{BD}}(\epsilon/16, m)
		\end{equation*}
		
		Moreover, for all $\delta >0$ and for $n$ sufficiently large
		\begin{equation*}
			m \gtrsim \bound \left(\frac{\lipschitz^2}{\epsilon^2} \left( \pdim \log \frac{\bound}{\epsilon} + \log \frac{1}{\delta }  \right)\right)^{\frac{1}{1+\frac 1 d}}
			\implies 
			\text{$\mathcal{S}$ is $1-\delta$-surely an $\epsilon$-coreset for $\qset$ }
		\end{equation*}
	\end{theorem}
\end{tcolorbox}


\begin{proof}
	Let $\query \in \qset$, thus we know $g := \frac{m\query}{\opekernel{m}\loss{\query}} \in \twoqset_m$ verifies \ref{def_ftog}.

	We know from Theorem \ref{thm_breuerfixedtheta} that for all $epsilon \in ]0,\bound]$ and $m \in \NN$ we have
	\begin{equation*}
		\PP{}{\dnude{\nu}{\empdistr{\mathcal{S}}{\twoquery}}{\meanempdistr{\twoquery}} > \epsilon}  \leq \boundrate_{\textrm{BD}}(\epsilon, m)
	\end{equation*}
	
	Hypothesis of Theorem \ref{thm_infi_union_bound} are thus satisfied, and we can apply it taking $\threeqset=\twoqset_m$, which yields that for all $m \in \NN$ such that $\boundrate_{\textrm{BD}}(\epsilon/2, m) \leq 1/2$, it holds
	\begin{align*}
		\PP{}{\exists \query \in \qset,\ \dnude{\nu}{\estloss{\mathcal{S}}{\query}}{\loss{\query}}>\epsilon\loss{\query}} 
		= \PP{}{\exists \twoquery \in \twoqset_m,\ \dnude{\nu}{\empdistr{\mathcal{S}}{\twoquery}}{\meanempdistr{\twoquery}} >  \epsilon}\\
		\leq  8\left(\frac{8e\bound}{\epsilon}\right)^{2d'_{\twoqset_m}}  \boundrate_{\textrm{BD}}(\epsilon/16, m)
		\leq  8\left(\frac{8e\bound}{\epsilon}\right)^{2\pdim} \boundrate_{\textrm{BD}}(\epsilon/16, m)
	\end{align*}
	where we used the assumption \ref{asm_gm} that for all $m \in \NN$, $\operatorname{pdim}\twoqset_m \leq \pdim$.

	Hence, a sufficient condition for $\mathcal{S} \sim \mathcal{DPP}(\opekernel{m})$ to satisfy $1-\delta$-surely the $\epsilon$-coreset for $\qset$ property \ref{eqn_qsetcoresetprop} is to have
	\begin{gather*}
		\boundrate_{\textrm{BD}}(\epsilon/16, m) \leq \frac{\delta}{8} \left(\frac{8e\bound}{\epsilon}\right)^{-2\pdim}
		\iff
		m \gtrsim \bound \left(\frac{256\lipschitz^2}{\epsilon^2} \log \left( \frac{16}{\delta }\left(\frac{8e\bound}{\epsilon}\right)^{2\pdim}\right) \right)^{\frac{1}{1+\frac 1 d}}\\
		\iff m \gtrsim \bound \left(\frac{\lipschitz^2}{\epsilon^2} \left(\log \frac{1}{\delta } + \pdim \log \frac{\bound}{\epsilon} \right)\right)^{\frac{1}{1+\frac 1 d}}
	\end{gather*}
	This rate is conditioned to the fact that $m$ is such that $\boundrate_{\textrm{BD}}(\epsilon/2, m) \leq 1/2$. But we know it holds as soon as $m \gtrsim \bound \left(\frac{4\lipschitz^2}{\epsilon^2} \log 4 \right)^{\frac{1}{1+\frac 1 d}}$, which is trivially implied by the obtained bound.

\end{proof}





\section{Proof of Theorem \ref{thm_infi_union_bound}}


We follow a similar proof scheme as in section 9.4 of \cite{haussler1992decisiontheoricgeneralizationofPACmodel}. We specifically revisit Lemma 12. and 13., getting rid of independency hypothesis, and making intermediary results more flexible to further improvements.


\begin{tcolorbox}
	\begin{lemma}[Symmetrisation]
		\label{lem_symm}
		Assume the hypothesis of \ref{thm_infi_union_bound}.
		Let furthermore be $\epsilon>0$, $m\in \NN$, and $\mathcal{S}_1, \mathcal{S}_2 \overset{i.i.d.}{\sim} \PP{m}{}$, two multisets of size $m$ independently sampled from the same distribution.\\
		  
		Then for all $m \in \NN$ such that $\boundrate(\epsilon/2, m) \leq 1/2$
		\begin{equation*}
			\PP{m}{ \exists \threequery \in \threeqset,\ \dnude{\nu}{\empdistr{\mathcal{S}_1}{\threequery}}{\meanempdistr{\threequery}} > \epsilon} \leq 2\PP{m}{\exists \threequery \in \threeqset,\ \dnude{\nu}{\empdistr{\mathcal{S}_1}{\threequery}}{\empdistr{\mathcal{S}_2}{\threequery}} > \epsilon/2 }
		\end{equation*}
	\end{lemma}
\end{tcolorbox}


\begin{proof}
	Let $\epsilon>0$ and take $m \in \NN$ such that $\boundrate(\epsilon/2, m) \leq 1/2$. Then let $\mathcal{S}_1$ be sampled such that $\exists \threequery \in \threeqset,\ \dnude{\nu}{\empdistr{\mathcal{S}_1}{\threequery}}{\meanempdistr{\threequery}} > \epsilon$. This obviously happens with probability $\PP{m}{\exists \threequery \in \threeqset, \dnude{\nu}{\empdistr{\mathcal{S}_1}{\threequery}}{\meanempdistr{\threequery}} > \epsilon }$.

	For such an $f$, we then independently sample $\mathcal{S}_2$ such that $\dnude{\nu}{\empdistr{\mathcal{S}_2}{\threequery}}{\meanempdistr{\threequery}} \leq \epsilon/2$. Because $\boundrate(\epsilon, m) \leq 1/2$, we know this happens with probability greater than $1-1/2 = 1/2$, and we thus have
	\begin{align*}
		&\PP{m}{ \exists \threequery \in \threeqset,\ \dnude{\nu}{\empdistr{\mathcal{S}_1}{\threequery}}{\meanempdistr{\threequery}} > \epsilon}
		\frac 1 2 \\
		\leq\ &\PP{m}{\exists \threequery \in \threeqset,\ \dnude{\nu}{\empdistr{\mathcal{S}_1}{\threequery}}{\meanempdistr{\threequery}} > \epsilon \wedge \dnude{\nu}{\empdistr{\mathcal{S}_2}{\threequery}}{\meanempdistr{\threequery}} \leq \epsilon/2 } \\
		\leq\ &\PP{m}{\exists \threequery \in \threeqset,\ \dnude{\nu}{\empdistr{\mathcal{S}_1}{\threequery}}{\empdistr{\mathcal{S}_2}{\threequery}} > \epsilon/2 }
	\end{align*}
	where we lastly used the triangular inequality 
	\begin{equation*}
		\dnude{\nu}{\empdistr{\mathcal{S}_1}{\threequery}}{\meanempdistr{\threequery}} - \dnude{\nu}{\empdistr{\mathcal{S}_2}{\threequery}}{\meanempdistr{\threequery}} \leq   \dnude{\nu}{\empdistr{\mathcal{S}_1}{\threequery}}{\empdistr{\mathcal{S}_2}{\threequery}}
	\end{equation*} 
\end{proof}







\begin{tcolorbox}
    \begin{definition}[Covering and Packing]
        Let $(\threeqset, d)$ be a metric space. 
        \begin{itemize}
            \item For any $\epsilon> 0$, a subset $\threeqset' \subseteq \threeqset$ is said to be $\epsilon$-separated if for all distinct $\threequery'_1, \threequery'_2 \in \threeqset'$, $d(\threequery'_1, \threequery'_2) > \epsilon$.
            \item The $\epsilon$-packing number on $(\threeqset, d)$, denoted by $\packing(\epsilon, \threeqset, d)$, is then defined as cardinality of the largest $\epsilon$-separated subset $\threeqset'$ of $\threeqset$.
        \end{itemize}
        Intuitively, the $\epsilon$-packing number is the maximal number of balls of radius $\epsilon/2$ that can fit into $\threeqset$ without intersecting.
        \begin{itemize}
            \item For any $\epsilon> 0$, a subset $\threeqset'$ of $\threeqset$ is said to be an $\epsilon$-cover of $\threeqset$ if for all $\threequery \in \threeqset$, it exists $\threequery' \in \threeqset'$ with $d(\threequery, \threequery') \leq \epsilon$.
            \item The $\epsilon$-covering number on $(\threeqset, d)$, denoted by $\covering(\epsilon, \threeqset, d)$, is then defined as cardinality of the smallest $\epsilon$-cover of $\threeqset$.
        \end{itemize}
        Intuitively, the $\epsilon$-covering number is the minimal number of balls of radius $\epsilon$ than can fill $\threeqset$, with possible overlaps.
    \end{definition}
\end{tcolorbox}

    One can easily check that for all $\epsilon>0$
    \begin{equation}
        \packing(2\epsilon, \threeqset, d) \leq \covering(\epsilon, \threeqset, d) \leq \packing(\epsilon, \threeqset, d)
    \end{equation}





\begin{tcolorbox}
    \begin{theorem}[From \cite{pollard1984_convergence_stoch_proc} and \cite{haussler1995spherepacking}]
        \label{thm_pack}
        For any set $\mathcal{X}$, any probability distribution $\mu$ on $\mathcal{X}$, any
        set $\threeqset \subseteq [0,\bound]^{\mathcal{X}}$ of $\mu$-measurable positive functions on $\mathcal{X}$ bounded by some real $\bound$, and any $\epsilon\in]0,\bound]$, one have
        \begin{align*}
            \packing(\epsilon, \threeqset, \dlone{\mu}{}{}) 
			\leq \min \left\{
			2\left(\frac{2eM}{\epsilon}\log\frac{2e\bound}{\epsilon}\right)^{d'_{\threeqset}},\ 
			e(d'_{\threeqset}+1) \left(\frac{2e \bound}{\epsilon}\right)^{d'_{\threeqset}}\right\}
        \end{align*}
        where $d'_{\threeqset} =\operatorname{pdim}\threeqset$ is the pseudo-dimension of $\threeqset$.
    \end{theorem}
\end{tcolorbox}
Note that the second bound is better than the first one when $\epsilon$ is sufficiently small compared to $d'_{\threeqset}$ and vice versa.







\begin{tcolorbox}
	\begin{lemma}[Conjectured]
		\label{lem_infi_union_bound}
		Let $\epsilon >0$, $m\in \NN$, and $\mathcal{S}_1, \mathcal{S}_2 \overset{i.i.d.}{\sim} \PP{}{}$, two multisets of size $m$ independently sampled from the same distribution. Then
		
		\begin{equation*}
			\PP{}{\exists \threequery \in \threeqset,\ \dnude{\nu}{\empdistr{\mathcal{S}_1}{\threequery}}{\empdistr{\mathcal{S}_2}{\threequery}} > \epsilon} \leq \sup_{\mathcal{S} \in \binom{\mathcal{X}}{2m} } \covering(\epsilon/4, \threeqset, \dlone{\empdistr{\mathcal{S}}{}}{}{}) \sup_{\threequery \in \threeqset} \PP{}{\dnude{\nu}{\empdistr{\mathcal{S}_1}{\threequery}}{\empdistr{\mathcal{S}_2}{\threequery}} > \epsilon/4 }
		\end{equation*}
	\end{lemma}
\end{tcolorbox}



\begin{proof}[Draft of Proof]
	Let $\mathcal{S}_1$ and $\mathcal{S}_2$ sampled such that $\exists \threequery \in \threeqset,\ \dnude{\nu}{\empdistr{\mathcal{S}_1}{\threequery}}{\empdistr{\mathcal{S}_2}{\threequery}} > \epsilon$. 
	
	We denote their multiset union by $\mathcal{S} := \mathcal{S}_1 \uplus \mathcal{S}_2 \in \binom{\mathcal{X}}{2m}$. Let then be taken $\threeqset'_{\mathcal{S}}$, a minimal $\epsilon/4$-cover of $\threeqset$ for the $\dlone{\empdistr{\mathcal{S}}{}}{}{}$ topology, then $|\threeqset'_{\mathcal{S}}| = \covering(\epsilon/4, \threeqset, \dlone{\empdistr{\mathcal{S}}{}}{}{})$. We thus know it exists $\threequery' \in \threeqset'_{\mathcal{S}}$ such that $\dlone{\empdistr{\mathcal{S}}{}}{\threequery}{\threequery'} \leq \epsilon/4$.


	Then triangular inequalities yields that
	\begin{align*}
		\dnude{\nu}{\empdistr{\mathcal{S}_1}{\threequery}}{\empdistr{\mathcal{S}_2}{\threequery}} &\leq \dnude{\nu}{\empdistr{\mathcal{S}_1}{\threequery'}}{\empdistr{\mathcal{S}_2}{\threequery'}} + \dnude{\nu}{\empdistr{\mathcal{S}_1}{\threequery}}{\empdistr{\mathcal{S}_1}{\threequery'}} + \dnude{\nu}{\empdistr{\mathcal{S}_2}{\threequery}}{\empdistr{\mathcal{S}_2}{\threequery'}}  \\
		&\leq \dnude{\nu}{\empdistr{\mathcal{S}_1}{\threequery'}}{\empdistr{\mathcal{S}_2}{\threequery'}}  + \empdistr{\mathcal{S}_1}{\lvert \threequery-\threequery'\rvert} + \empdistr{\mathcal{S}_2}{\lvert\threequery-\threequery'\rvert}\\
		&\leq \dnude{\nu}{\empdistr{\mathcal{S}_1}{\threequery'}}{\empdistr{\mathcal{S}_2}{\threequery'}} + 2 \dlone{\empdistr{\mathcal{S}}{}}{\threequery}{\threequery'}\\
		\iff \dnude{\nu}{\empdistr{\mathcal{S}_1}{\threequery'}}{\empdistr{\mathcal{S}_2}{\threequery'}} &\geq \dnude{\nu}{\empdistr{\mathcal{S}_1}{\threequery}}{\empdistr{\mathcal{S}_2}{\threequery}} - 2  \dlone{\empdistr{\mathcal{S}}{}}{\threequery}{\threequery'} > \epsilon/2 
	\end{align*}
	Therefore
	\begin{equation*}
		\PP{}{\exists \threequery \in \threeqset,\ \dnude{\nu}{\empdistr{\mathcal{S}_1}{\threequery}}{\empdistr{\mathcal{S}_2}{\threequery}} > \epsilon} \leq \PP{}{\exists \threequery \in \threeqset'_{\mathcal{S}},\ \dnude{\nu}{\empdistr{\mathcal{S}_1}{\threequery}}{\empdistr{\mathcal{S}_2}{\threequery}} > \epsilon/4}
	\end{equation*}
	By the law of total expectation, we obtain
	\begin{align*}
		\PP{}{\exists \threequery \in \threeqset'_{\mathcal{S}},\ \dnude{\nu}{\empdistr{\mathcal{S}_1}{\threequery}}{\empdistr{\mathcal{S}_2}{\threequery}} > \epsilon/4}
		&=\EE{}{ \1 \{\exists \threequery \in \threeqset'_{\mathcal{S}},\ \dnude{\nu}{\empdistr{\mathcal{S}_1}{\threequery}}{\empdistr{\mathcal{S}_2}{\threequery}} > \epsilon/4 \} }\\
		&=\EE{}{ \PP{}{\exists \threequery \in \threeqset'_{\mathcal{S}},\ \dnude{\nu}{\empdistr{\mathcal{S}_1}{\threequery}}{\empdistr{\mathcal{S}_2}{\threequery}} > \epsilon/4 \mid \threeqset'_{\mathcal{S}}}  }\\
		&=\EE{}{ \PP{}{\bigcup_{\threequery \in \threeqset'_{\mathcal{S}}} \{ \dnude{\nu}{\empdistr{\mathcal{S}_1}{\threequery}}{\empdistr{\mathcal{S}_2}{\threequery}} > \epsilon/4 \mid \threeqset'_{\mathcal{S}}\} }  }\\		
        &\overset{?}{\leq} \sup_{\threeqset'_{\mathcal{S}}} |\threeqset'_{\mathcal{S}}| \sup_{\threequery \in \threeqset} \PP{}{\dnude{\nu}{\empdistr{\mathcal{S}_1}{\threequery}}{\empdistr{\mathcal{S}_2}{\threequery}} > \epsilon/4 }\\
		&= \sup_{\mathcal{S} \in \binom{\mathcal{X}}{2m} } N(\epsilon/4, \threeqset, \dlone{\empdistr{\mathcal{S}}{}}{}{} ) \sup_{\threequery \in \threeqset} \PP{}{\dnude{\nu}{\empdistr{\mathcal{S}_1}{\threequery}}{\empdistr{\mathcal{S}_2}{\threequery}} > \epsilon/4 }
	\end{align*}
$\overset{?}{\leq}$ indicates this inequality is still to be proven. It consists of bounding the union probability of random event other a random set. Since we can bound uniformly both, event probability, and set cardinality, it could seem intuitive this random union bound hold. However, the conditioning by $\threeqset'_{\mathcal{S}}$ makes this bound non trivial and require further assumptions.

\end{proof}


From this we are now able to prove Theorem \ref{thm_infi_union_bound}. 
\begin{proof}
	Let $\epsilon>0$ and take $m \in \NN$ such that $\boundrate(\epsilon/2, m) \leq 1/2$. Combining Lemmas \ref{lem_symm} and \ref{lem_infi_union_bound} gives
	\begin{align*}
		\PP{}{ \exists \threequery \in \threeqset,\ \dnude{\nu}{\empdistr{\mathcal{S}_1}{\threequery}}{\meanempdistr{\threequery}} > \epsilon} 
		&\leq 2\PP{}{\exists \threequery \in \threeqset,\ \dnude{\nu}{\empdistr{\mathcal{S}_1}{\threequery}}{\empdistr{\mathcal{S}_2}{\threequery}} > \epsilon/2 }\\
		&\leq 2 \sup_{\mathcal{S} \in \binom{\mathcal{X}}{2m} } \covering(\epsilon/8, \threeqset, \dlone{\empdistr{\mathcal{S}}{}}{}{}) \sup_{\threequery \in \threeqset} \PP{}{\dnude{\nu}{\empdistr{\mathcal{S}_1}{\threequery}}{\empdistr{\mathcal{S}_2}{\threequery}} > \epsilon/8 }
	\end{align*}

In order to bound the last term, first consider applying the union bound
	\begin{align*}
		\PP{}{\dnude{\nu}{\empdistr{\mathcal{S}_1}{\threequery}}{\empdistr{\mathcal{S}_2}{\threequery}} > \epsilon/8 } 
		&\leq \PP{}{\dnude{\nu}{\empdistr{\mathcal{S}_1}{\threequery}}{\meanempdistr{\threequery}} + \dnude{\nu}{\meanempdistr{\threequery}}{\empdistr{\mathcal{S}_2}{\threequery}}> \epsilon/8 }\\
		&\leq \PP{}{\dnude{\nu}{\empdistr{\mathcal{S}_1}{\threequery}}{\meanempdistr{\threequery}} > \epsilon/16 \vee  \dnude{\nu}{\meanempdistr{\threequery}}{\empdistr{\mathcal{S}_2}{\threequery}}> \epsilon/16 }\\
		&\leq \PP{}{\dnude{\nu}{\empdistr{\mathcal{S}_1}{\threequery}}{\meanempdistr{\threequery}} > \epsilon/16 \vee  \dnude{\nu}{\meanempdistr{\threequery}}{\empdistr{\mathcal{S}_2}{\threequery}}> \epsilon/16 }\\
		&\leq 2\PP{}{\dnude{\nu}{\empdistr{\mathcal{S}_1}{\threequery}}{\meanempdistr{\threequery}} > \epsilon/16 }\\
		&\leq 2 \boundrate(\threeqset, \epsilon/16, m)
	\end{align*}
Second, we know from Theorem \ref{thm_pack}
\begin{align*}
	\covering(\epsilon/8, \threeqset, \dlone{\empdistr{\mathcal{S}}{}}{}{})
	\leq \packing(\epsilon/8, \threeqset, \dlone{\empdistr{\mathcal{S}}{}}{}{})
	&\leq 2\left(\frac{16eM}{\epsilon}\log\frac{16e\bound}{\epsilon}\right)^{d'_{\threeqset}}
	&\leq  2\left(\frac{8e\bound}{\epsilon}\right)^{2d'_{\threeqset}}
\end{align*}
where we used the fact that $a \log a < (a/2)^2$ whenever $a \geq 5$, which is the case for $\frac{8eM}{\epsilon} \geq 5$ since $\epsilon \in ]0,\bound]$.

The two precedent bound does neither depend on $\mathcal{S}$ nor $f$, and therefore 
\begin{align*}
	\PP{}{ \exists \threequery \in \threeqset,\ \dnude{\nu}{\empdistr{\mathcal{S}_1}{\threequery}}{\meanempdistr{\threequery}} > \epsilon} 
	&\leq 8\left(\frac{8e\bound}{\epsilon}\right)^{2d'_{\threeqset}}\boundrate(\epsilon/16, m)\\
\end{align*}
which is the desired result.

\end{proof}













\begin{tcolorbox}[colback=red!10,title= Useless?]
	We define $\forall a,b \geq 0$ and $\forall \nu >0$
	\begin{equation}
		\dnu{\nu}{a}{b} := \frac{|a-b|}{a+b+\nu}
	\end{equation}
	
	From that definition, one can easily check
	\begin{proposition}
		\label{prop_dnu}
		For all $\nu>0$, the function $d_\nu$ verifies the following properties
		\begin{itemize}
			\item $d_\nu$ is a distance i.e. it verifies positivity, separation, symmetry and triangular inequality.
			\item $\forall a,b \geq 0,\ 0\leq \dnu{\nu}{a}{b} < \min( \frac{|a-b|}{a}, 1)$
			\item Moreover, if $\exists \bound\geq 0$ such that $a,b \le \bound$, then 
			\begin{equation*}
				\frac{|a-b|}{\nu + 2\bound} \le \dnu{\nu}{a}{b} \le \frac{|a-b|}{\nu}
			\end{equation*}
		\end{itemize}
	\end{proposition}


    
    Using properties \ref{prop_dnu} of $\dnude{\nu}{}{}$ distance yields that
    \begin{align*}
        \dnude{\nu}{\empdistr{\mathcal{S}_1}{\query}}{\empdistr{\mathcal{S}_1}{\query^*}} + \dnude{\nu}{\empdistr{\mathcal{S}_2}{\query}}{\empdistr{\mathcal{S}_2}{\query^*}}
        &\le \frac{\lvert \empdistr{\mathcal{S}_1}{\query-\query^*}\rvert}{\nu} + \frac{\lvert \empdistr{\mathcal{S}_2}{\query-\query^*}\rvert}{\nu} \\
        &\le \frac{2}{\nu}  \dlone{\empdistr{\mathcal{S}}{}}{\query}{\query^*}
    \end{align*}
\end{tcolorbox}






\begin{tcolorbox}
	\begin{corollary}
		\label{thm_anyboundrate}
		Define the following space of functions defined on $\mathcal{X}$
		\begin{equation}
			\twoqset_m := \frac{m\threeqset}{\opekernel{m}\loss{\threeqset}} = \left\{\frac{m\threequery}{\opekernel{m}\loss{\threequery}} \mid \threequery \in \threeqset\right\}
		\end{equation}
		We denote its pseudo-dimension $\pdim:=\operatorname{pdim}\twoqset_m$, and assume its functions are bounded by a constant $\bound$.\\

		Then for all $\epsilon>0$ and all $m \geq m_{\epsilon}$
		\begin{equation}
			\PP{}{\exists \threequery \in \threeqset,\ \dnude{\nu}{\estloss{\mathcal{S}}{\threequery}}{\loss{\threequery}}>\epsilon \loss{\threequery}} 
			\leq 12(\pdim+1) \left(\frac{48 \bound}{\epsilon}\right)^{\pdim} \boundrate(\threeqset', \epsilon, m)
		\end{equation}
	\end{corollary}
\end{tcolorbox}
	


\begin{proof}
	Observe first that for any function $f$
	\begin{equation*}
		\estloss{\mathcal{S}}{f} = \empdistr{\mathcal{S}}{ \frac{m\threequery}{\opekernel{m}} }
	\end{equation*}
	Then by Theorem \ref{thm_chebfixedtheta}, for all $\epsilon >0$ , $m\in \NN$, and $\twoquery \in \twoqset_m$
	\begin{align*}
		\PP{}{\dnude{\nu}{\empdistr{\mathcal{S}}{\twoquery}}{\meanempdistr{\twoquery}}>\epsilon} 
		&= \PP{}{\dnude{}{\estloss{\mathcal{S}}{\frac{\twoquery\opekernel{m}}{m}}}{\loss{\frac{\twoquery\opekernel{m}}{m}}}>\epsilon} \\
		&\leq  \boundrate(\frac{\twoqset_m\opekernel{m}}{m}, \epsilon, m) = \boundrate(\threeqset', \epsilon, m)
   \end{align*}
   Moreover, we know that $\boundrate$ and $m_\epsilon$ are such that
   \begin{equation*}
		m \geq m_{\epsilon} \implies \boundrate(\threeqset, \epsilon, m) \leq 1/2
   \end{equation*}
   
   Therefore $\boundrate$ satisfies the two hypothesis of Theorem \ref{thm_infi_union_bound}, and thus for all $\epsilon >0$ and $m \geq m_\epsilon$
   \begin{align*}
	\PP{}{\exists \threequery \in \threeqset,\ \dnude{\nu}{\estloss{\mathcal{S}}{\threequery}}{\loss{\threequery}}>\epsilon \loss{\threequery}} 
	&= \PP{}{\exists \twoquery \in \twoqset_m,\ \dnude{\nu}{\empdistr{\mathcal{S}}{\twoquery}}{\meanempdistr{\twoquery}} > \epsilon}\\
	&\leq 12(\pdim+1) \left(\frac{48 \bound}{\epsilon}\right)^{\pdim} \boundrate(\threeqset', \epsilon, m)
   \end{align*}
\end{proof}

\note{}{Conclude!}






\begin{tcolorbox}[colback=red!75, title=Useless?]
	
	We can thus define for any set of function $\qset$, any $m\in \NN$, and $\epsilon>0$, the Chebyshov concentration bound
	\begin{equation*}
		\boundrate_{\textrm{Cheb}}(\qset, \epsilon, m) := \frac{1}{\epsilon^2} \sup_{\query \in \qset} \Var{}{\estloss{\mathcal{S}}{\query}}
	\end{equation*}
	
	
	\begin{tcolorbox}
		\begin{theorem}[Chebyshov bound for fixed query]
			Let $m\in \NN$ and $\mathcal{S} \sim  \mathcal{DPP}(\opekernel{m})$. 
			Define $\qset':=\frac{\qset}{\loss{\qset}}=\left\{\frac{\query}{\loss{\query}}\mid \query \in \qset\right\}$ and $\lipschitz_{\qset'}$ its Lipschitz constant bound.\\
	
			Then for all $\epsilon, \delta >0$, all $\query \in \qset$, and $n$ sufficiently large
			\begin{equation}
				m \gtrsim \left(\frac{\lipschitz^2_{\qset'}}{\delta\epsilon^2} \right)^{\frac{1}{1+\frac 1 d}} \implies \text{$\mathcal{S}$ is $1-\delta$-surely an $\epsilon$-coreset for $\qset$ }
			\end{equation}
			where $y \gtrsim x$ is a transitive notation for $y = \Omega(x)$ i.e. $y$ is lower bounded by $x$ up to a constant factor.
		\end{theorem}
	\end{tcolorbox}
	
	
	\begin{proof}
		Applying Bienaym\'e-Chebyshov inequality on the probability of $\mathcal{S}$ being a coreset yields 
		\begin{align*}
			 \PP{}{\dnude{\nu}{\estloss{\mathcal{S}}{\query}}{\loss{\query}}>\epsilon\loss{\query}} 
			 &\leq \frac{1}{\varepsilon ^{2}}\Var{}{\frac{\estloss{\mathcal{S}}{\query}}{\loss{\query}} }
			 = \frac{1}{\varepsilon ^{2}}\Var{}{\estloss{\mathcal{S}}{\frac{\query}{\loss{\query}}} }\\
			 &\leq \boundrate_{\textrm{Cheb}}(\qset', \epsilon, m)\\
			 &=\frac {1} {\epsilon^2}\left(\lipschitz^2_{\qset'} \OO( m ^{-(1+\frac 1 d)}) +\OO( n^{-1/2})\right)
		\end{align*}
		Hence, a sufficient condition for $\mathcal{S} \sim \mathcal{DPP}(\opekernel{m})$ to satisfy $1-\delta$-surely the $\epsilon$-coreset property \ref{def_coresetprop} is to have
		\begin{gather*}
			\boundrate_{\textrm{Cheb}}(\qset', \epsilon, m) \leq \delta\\
			\iff\\
			m^{1+\frac 1 d} \gtrsim \frac{\lipschitz^2_{\qset'}}{\delta \epsilon^2 + \OO(n^{-1/2})} = \frac {\lipschitz^2_{\qset'}} {\delta\epsilon^2} \frac{1}{1 + \frac{1}{\delta \epsilon^2}\OO(n^{-1/2})}
		\end{gather*} 
		Then this means that for sufficiently large $n$ (potentially $n\gtrsim \delta^{-2} \epsilon^{-4}$), we can control the second factor and thus obtain the bound
		\begin{equation*}
			m \gtrsim \left(\frac{\lipschitz^2_{\qset'}}{\delta\epsilon^2} \right)^{\frac{1}{1+\frac 1 d}}
		\end{equation*}
	\end{proof}



	
We apply Theorem \ref{thm_breuer} for $\estloss{\mathcal{S}}{\query} = X_{\nicefrac{\query}{\opekernel{m}}}$. We define the Breuer concentration bound $\boundrate_{\textrm{Breuer}}$ such that
\begin{align*}
	\PP{}{ \dnude{\nu}{\estloss{\mathcal{S}}{\query}}{\loss{\query}}  > \epsilon} 
	&\leq 2 \exp \left(
	\begin{cases}
		\frac{-\epsilon^2}{4 A \Var{}{\estloss{\mathcal{S}}{\query} }}
		& \text { if } \epsilon<\epsilon_{\query, m}\\
		\frac{-\epsilon}{6\|\nicefrac{\query}{\opekernel{m}}\|_{\infty}} 
		& \text { otherwise }
	\end{cases}
	\right)\\
	&\leq 2 \exp \left(
		\begin{cases}
			\left(-4 A \boundrate_{\textrm{Cheb}}(\qset, \epsilon, m)\right)^{-1}
			& \text { if } \epsilon<\epsilon_{\query, m}\\
			\left(-\epsilon^{-1} 6\sup_{\query}\|\nicefrac{\query}{\opekernel{m}}\|_{\infty}\right)^{-1}
			& \text { otherwise }
		\end{cases}
		\right)\\
	& := \boundrate_{\textrm{Breuer}}(\qset, \epsilon,m)
\end{align*}
where we lastly used the definition \ref{def_boundratecheb} of $\boundrate_{\textrm{Cheb}}$, and where $\epsilon_{\query, m} := \frac{2 A \Var{}{\estloss{\mathcal{S}}{\query} }}{3\|\nicefrac{\query}{\opekernel{m}}  \|_{\infty}}$.
\end{tcolorbox}



\begin{tcolorbox}[colback=red!25, title=Useless?]
	For all $\epsilon$, we note $m_{\textrm{Cheb}}(\epsilon, \qset') := c_{1/2} \left(\frac{\lipschitz^2_{\qset'}}{\epsilon^2} \right)^{\frac{1}{1+\frac 1 d}}$
\end{tcolorbox}