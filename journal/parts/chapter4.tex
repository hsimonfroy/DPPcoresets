

\chapter{Improving concentration with DPP}
\section{Concentration for fixed query}

\cite{bardenet2021sgddpp} show the existence of a sequence of DPP kernels $(\opekernel{m})$, independent of $f$, whose induced estimator has asymptotic variance $\OO( m ^{-(1+\frac 1 d)})$. More precisely, 

\begin{tcolorbox}
	\begin{theorem}[From \cite{bardenet2021sgddpp}]
		$\mathcal{S} \sim  \mathcal{DPP}(\opekernel{m})$
		\begin{equation}
			\estloss{\mathcal{S}}{\query} := \sum_{x \in \mathcal{S}} \frac{f(x)}{\opekernel{m}(x,x)}
		\end{equation}
		Equation (S14) yields that 
		\begin{equation}
			\Var{}{\estloss{\mathcal{S}}{\query}} = \lipschitz^2_\query \OO( m ^{-(1+\frac 1 d)}) +\OO( n^{-1/2})
		\end{equation}
		where $\lipschitz_\query$ is the Lipschitz constant of $x \mapsto \query(x) (\frac{1}{m} K^{(m)}_{q, \tilde \gamma}(x,x))^{-1}$, supposedly bounded in $\queryset$ and whose bound we denote by $\lipschitz_{\queryset} := \sup_{\query \in \queryset}\lipschitz_\query$.
	\end{theorem}
\end{tcolorbox}
\note{}{explain how the estimator is built\\
$\mathcal{X} \subseteq \RR^d$}


\subsection{Chebyshov bound}

We can thus define for any set of function $\queryset$, any $m\in \NN$, and $\epsilon>0$, the Chebyshov concentration bound
\begin{equation*}
	\label{def_boundratecheb}
	\boundrate_{\textrm{Cheb}}(\queryset, \epsilon, m) := \frac{1}{\epsilon^2} \sup_{\query \in \queryset} \Var{}{\estloss{\mathcal{S}}{\query}}
\end{equation*}


\begin{tcolorbox}
	\begin{theorem}[Chebyshov bound for fixed query]
		\label{thm_fixedtheta}
		Let be $m\in \NN$ and $\mathcal{S} \sim  \mathcal{DPP}(\opekernel{m})$. 
		Define $\queryset':=\frac{\queryset}{\loss{\queryset}}=\left\{\frac{\query}{\loss{\query}}\mid \query \in \queryset\right\}$ and $\lipschitz_{\queryset'}$ its Lipschitz constant bound.\\

		Then for all $\epsilon, \delta >0$, all $\query \in \queryset$, and $n$ sufficiently large
		\begin{equation}
			m \gtrsim \left(\frac{\lipschitz^2_{\queryset'}}{\delta\epsilon^2} \right)^{\frac{1}{1+\frac 1 d}} \implies \text{$\mathcal{S}$ is $1-\delta$-surely an $\epsilon$-coreset for $\queryset$ }
		\end{equation}
		where $y \gtrsim x$ is a transitive notation for $y = \Omega(x)$ i.e. $y$ is lower bounded by $x$ up to a constant factor.
	\end{theorem}
\end{tcolorbox}
\note{}{lipschitz constant of?}


\begin{proof}
	Applying Bienaym\'e-Chebyshov inequality on the probability of $\mathcal{S}$ being a coreset yields 
	\begin{align*}
		 \PP{}{\dnude{\nu}{\estloss{\mathcal{S}}{\query}}{\loss{\query}}>\epsilon\loss{\query}} 
		 &\leq \frac{1}{\varepsilon ^{2}}\Var{}{\frac{\estloss{\mathcal{S}}{\query}}{\loss{\query}} }
		 = \frac{1}{\varepsilon ^{2}}\Var{}{\estloss{\mathcal{S}}{\frac{\query}{\loss{\query}}} }\\
		 &\leq \boundrate_{\textrm{Cheb}}(\queryset', \epsilon, m)\\
		 &=\frac {1} {\epsilon^2}\left(\lipschitz^2_{\queryset'} \OO( m ^{-(1+\frac 1 d)}) +\OO( n^{-1/2})\right)
	\end{align*}
	Hence, a sufficient condition for $\mathcal{S} \sim \mathcal{DPP}(\opekernel{m})$ to satisfy $1-\delta$-surely the $\epsilon$-coreset property \ref{def_coresetprop} is to have
	\begin{gather*}
		\boundrate_{\textrm{Cheb}}(\queryset', \epsilon, m) \leq \delta\\
		\iff\\
		m^{1+\frac 1 d} \gtrsim \frac{\lipschitz^2_{\queryset'}}{\delta \epsilon^2 + \OO(n^{-1/2})} = \frac {\lipschitz^2_{\queryset'}} {\delta\epsilon^2} \frac{1}{1 + \frac{1}{\delta \epsilon^2}\OO(n^{-1/2})}
	\end{gather*} 
	Then this means that for sufficiently large $n$ (potentially $n\gtrsim \delta^{-2} \epsilon^{-4}$), we can control the second factor and thus obtain the bound
	\begin{equation*}
		m \gtrsim \left(\frac{\lipschitz^2_{\queryset'}}{\delta\epsilon^2} \right)^{\frac{1}{1+\frac 1 d}}
	\end{equation*}
\end{proof}



\begin{tcolorbox}
	\begin{corollary}
		By taking $\delta = 1/2$ in previous Theorem \ref{thm_fixedtheta}, we know it exists some constant we denote $c_{1/2}$ such that for all $\epsilon >0$ it holds that
		\begin{equation*}
			m \geq c_{1/2} \left(\frac{\lipschitz^2_{\queryset'}}{\epsilon^2} \right)^{\frac{1}{1+\frac 1 d}} \implies \forall \query \in \queryset,\ \PP{}{ \dnude{\nu}{\estloss{\mathcal{S}}{\query}}{\loss{\query}}  \leq \epsilon \loss{\query}} \geq \frac 1 2
		\end{equation*}
		For all $\epsilon$, we note $m_{\textrm{Cheb}}(\epsilon, \queryset') := c_{1/2} \left(\frac{\lipschitz^2_{\queryset'}}{\epsilon^2} \right)^{\frac{1}{1+\frac 1 d}}$
	\end{corollary}
\end{tcolorbox}



\subsection{Breuer bound}

% \begin{tcolorbox}
%     \begin{theorem}[From \cite{breuer2013nevai}]
%         $$\PP{}{\left|X_f-\mathbb{E} X_f\right| > \epsilon} \leq \begin{cases}2 \exp \left(-\frac{\epsilon^2}{4 A \Var{}{X_f}}\right), & \text { if } \epsilon<\frac{2 A \Var{}{X_f}{3\|f\|_{\infty}} \\ 2 \exp \left(-\frac{\epsilon}{6\|f\|_{\infty}}\right), & \text { if } \epsilon \geq \frac{2 A \Var{}{X_f}}{3\|f\|_{\infty}}\end{cases}$$
% \end{tcolorbox}


\begin{tcolorbox}
	\begin{theorem}[From \cite{breuer2013nevai}]
		\label{thm_breuer}
		Let be any $\epsilon>0$, any bounded function $\query$, any projective DPP kernel $K$, and let $\mathcal{S} \sim  \mathcal{DPP}(K)$.\\

		Then for any linear statistics $X_f := \sum_{x\in\mathcal{S}}\query(x)$
		\begin{equation*}
			\PP{}{\dnude{}{X_f}{\EE{}{X_f}} > \epsilon} \leq	
			\begin{cases}
				2 \exp \left(-\frac{\epsilon^2}{4 A \Var{}{X_f}}\right) 
				& \text { if } \epsilon<\frac{2 A \Var{}{X_f}}{3\|\query\|_{\infty}}\\
				2 \exp \left(-\frac{\epsilon}{6\|\query\|_{\infty}}\right) 
				& \text { otherwise }
			\end{cases}
		\end{equation*}	
		where $A > 0$ is a constant that does not depend on $\query$, $K$ or $\epsilon$.
	\end{theorem}
\end{tcolorbox}


We apply Theorem \ref{thm_breuer} for $\estloss{\mathcal{S}}{\query} = X_{\nicefrac{\query}{\opekernel{m}}}$. We define the Breuer concentration bound $\boundrate_{\textrm{Breuer}}$ such that
\begin{align*}
	\PP{}{ \dnude{\nu}{\estloss{\mathcal{S}}{\query}}{\loss{\query}}  > \epsilon} 
	&\leq 2 \exp \left(
	\begin{cases}
		\frac{-\epsilon^2}{4 A \Var{}{\estloss{\mathcal{S}}{\query} }}
		& \text { if } \epsilon<\epsilon_{\query, m}\\
		\frac{-\epsilon}{6\|\nicefrac{\query}{\opekernel{m}}\|_{\infty}} 
		& \text { otherwise }
	\end{cases}
	\right)\\
	&\leq 2 \exp \left(
		\begin{cases}
			\left(-4 A \boundrate_{\textrm{Cheb}}(\queryset, \epsilon, m)\right)^{-1}
			& \text { if } \epsilon<\epsilon_{\query, m}\\
			\left(-\epsilon^{-1} 6\sup_{\query}\|\nicefrac{\query}{\opekernel{m}}\|_{\infty}\right)^{-1}
			& \text { otherwise }
		\end{cases}
		\right)\\
	& := \boundrate_{\textrm{Breuer}}(\queryset, \epsilon,m)
\end{align*}
where we lastly used the definition \ref{def_boundratecheb} of $\boundrate_{\textrm{Cheb}}$, and where $\epsilon_{\query, m} := \frac{2 A \Var{}{\estloss{\mathcal{S}}{\query} }}{3\|\nicefrac{\query}{\opekernel{m}}  \|_{\infty}}$.



\begin{tcolorbox}[title=Breuer]
	\begin{theorem}[Breuer bound for fixed query]
		\label{thm_breuerfixedtheta}
		Let be $m\in \NN$ and $\mathcal{S} \sim  \mathcal{DPP}(\opekernel{m})$. 
		Define $\queryset':=\frac{\queryset}{\loss{\queryset}}=\left\{\frac{\query}{\loss{\query}}\mid \query \in \queryset\right\}$ and $\lipschitz_{\queryset'}$ its Lipschitz constant bound.\\

		Then for all $\epsilon, \delta >0$, all $\query \in \queryset$, and $n$ sufficiently large
		\begin{equation}
			m \gtrsim \left(\frac{\lipschitz^2_{\queryset'}}{\delta\epsilon^2} \right)^{\frac{1}{1+\frac 1 d}} \implies \text{$\mathcal{S}$ is $1-\delta$-surely an $\epsilon$-coreset for $\queryset$ }
		\end{equation}
		where $y \gtrsim x$ is a transitive notation for $y = \Omega(x)$ i.e. $y$ is lower bounded by $x$ up to a constant factor.
	\end{theorem}
\end{tcolorbox}






\begin{proof}
	
	
	Hence, a sufficient condition for $\mathcal{S} \sim \mathcal{DPP}(\opekernel{m})$ to satisfy $1-\delta$-surely the $\epsilon$-coreset property \ref{def_coresetprop} is to have
	\begin{gather*}
		\boundrate_{\textrm{Breuer}}(\queryset', \epsilon,m) \leq \delta\\
		\iff \\
		\left(\log \left( \frac{2}{\delta }\right)\right)^{-1} \geq
		\begin{cases}
			4 A \boundrate_{\textrm{Cheb}}(\queryset, \epsilon, m)
			& \text { if } \epsilon<\epsilon_{\nicefrac{\query}{\loss{\query}}, m}\\
			\frac{1}{\epsilon} 6\sup_{\query}\|\nicefrac{\query}{\opekernel{m}}\|_{\infty}
			& \text { otherwise }
		\end{cases}
	\end{gather*}
	If $\epsilon<\epsilon_{\nicefrac{\query}{\loss{\query}}, m}$, we know by \ref{thm_fixedtheta} a sufficient condition is that for $n$ sufficiently large 
	\begin{equation*}
		m \gtrsim \left(\frac{\lipschitz^2_{\queryset'}}{A\epsilon^2} \log \left( \frac{2}{\delta }\right) \right)^{\frac{1}{1+\frac 1 d}}
	\end{equation*}
	If $\epsilon\geq\epsilon_{\nicefrac{\query}{\loss{\query}}, m}$
	\begin{equation*}
		la
	\end{equation*}


\end{proof}
	
		 
		 
		 
	  $$m \geq \left( \frac{c}{L^{2}\varepsilon ^{2}}\log \left( \frac{2}{\delta }\right) \left( 1+O\left( n^{-\frac{1}{2}}\right) \right) \right) ^{\frac{1}{1+\frac{1}{d}}}$$
	


\section{Extension to all queries}
In order to obtain an $\epsilon$-coreset, the $\epsilon$-coreset property \ref{def_coresetprop} must holds for all queries, thus the previous result must be generalized to all $\query \in \queryset$.

We first recall some classical notations.
\begin{itemize}
	\item We denote by $\empdistr{\mathcal{S}}{}:=\frac{1}{|\mathcal{S}|} \sum_{x \in \mathcal{S}}$ the empirical measure based on sample $\mathcal{S}  \subseteq \mathcal{X}$. 
	\item Hence $\forall \query \in \queryset$, we denote by $\empdistr{\mathcal{S}}{\query}:=\int_{\mathcal{X}} \query(x) d\empdistr{\mathcal{S}}{x} = \frac{1}{|\mathcal{S}|} \sum_{x \in \mathcal{S}} \query(x)$ the expectation of $f$ with respect to $\empdistr{\mathcal{S}}{}$. Furthermore, given a distribution $\PP{}{}$ on $\mathcal{S}$, we denote by $\meanempdistr{f}:=\EE{}{\empdistr{\mathcal{S}}{\query}}$, its expectation with respect to $\PP{}{}$. 
	\item Finally, the induced $L^1(\empdistr{\mathcal{S}}{})$ distance between two functions $\query$ and $\query'$ would be denoted by $\dlone{\empdistr{\mathcal{S}}{}}{\query}{\query'}:=\empdistr{\mathcal{S}}{|\query - \query'|}$.
\end{itemize}  


\begin{tcolorbox}
	\begin{definition}[pseudo-dimension]
		The pseudo-dimension of a set $\queryset$ of functions defined on $\mathcal{X}$, denoted by $\operatorname{pdim}\queryset$, is the largest $\pdim$ such that 
	\begin{itemize}
		\item there exists $(x_{i})_{i\in \intint{1}{\pdim}} \subseteq \mathcal{X}^\pdim$, a sequence of $\pdim$ elements from $\mathcal{X}$,
		\item there exists $(t_i)_{i\in \intint{1}{\pdim}} \subseteq  \RR^\pdim$ a sequence of $\pdim$ real thresholds,
		\item such that for each $(b_i)_{i\in \intint{1}{\pdim}} \subseteq \{0,1\}^\pdim$
		\item there is an $\query \in \queryset$ such that $\forall i \in \intint{1}{\pdim}$, we have $\query(x_i) \geq r_i \iff b_i = 1$. 
	\end{itemize}
	Put differently it always exists functions in $\queryset$ to have values above or below some threshold for every $2^\pdim$ combinations of above/below.
\end{definition}
Pseudo-dimension can also be defined through VC-dimension. Indeed, considering the function
\begin{align*}
	h_\query \colon \mathcal{X} \times \RR &\to \{0,1\}\\
	(x,r) &\mapsto \1\{f(x) \geq r\}
\end{align*}
we have
\begin{equation}
	\operatorname{pdim}\queryset := \operatorname{VCdim}\{h_\query \mid \query \in \queryset\}
\end{equation}
\end{tcolorbox}








\begin{tcolorbox}
	\begin{theorem}
		\label{thm_infi_union_bound}
		Let be $\queryset \subseteq [0,\bound]^{\mathcal{X}}$ a set of bounded functions defined on a base set $\mathcal{X}$, with $\pdim := \operatorname{pdim}\queryset$ its pseudo-dimension. Let moreover be $(\PP{m}{})_{m\in \NN}$ a sequence of distributions supported on $\binom{\mathcal{X}}{m}$.\\

		Assume it exists a bounding function $\boundrate$ such that for all $\epsilon >0$, function $\query \in \queryset$, integer $m \in \NN$, and multiset $\mathcal{S} \sim \PP{m}{}$, we have the bound
		\begin{equation}
			\PP{m}{\dnude{\nu}{\empdistr{\mathcal{S}}{f}}{\meanempdistr{f}} > \epsilon} \leq \boundrate(\queryset, \epsilon, m)
		\end{equation}

		and such that for all $\epsilon>0$, it exists some $m_{\epsilon}$ such that
		\begin{equation}
			m \geq m_{\epsilon} \implies \boundrate(\queryset, \epsilon, m) \leq 1/2
		\end{equation}
		Then for all $\epsilon >0$ and $m \geq m_{\epsilon}$
		\begin{equation}
			\PP{m}{\exists \query \in \queryset,\ \dnude{\nu}{\empdistr{\mathcal{S}}{f}}{\meanempdistr{f}} > \epsilon} \leq 12(\pdim+1) \left(\frac{48 \bound}{\epsilon}\right)^\pdim \boundrate(\queryset, \epsilon/16, m)
		\end{equation}
	\end{theorem}
\end{tcolorbox}





\begin{tcolorbox}
	\begin{corollary}
		\label{thm_anyboundrate}
		Define the following space of functions defined on $\mathcal{X}$
		\begin{equation}
			\querysettwo_m := \frac{m\queryset}{\opekernel{m}\loss{\queryset}} = \left\{\frac{m\query}{\opekernel{m}\loss{\query}} \mid \query \in \queryset\right\}
		\end{equation}
		We denote its pseudo-dimension $\pdim:=\operatorname{pdim}\querysettwo_m$, and assume its functions are bounded by a constant $\bound$.\\

		Then for all $\epsilon>0$ and all $m \geq m_{\epsilon}$
		\begin{equation}
			\PP{}{\exists \query \in \queryset,\ \dnude{\nu}{\estloss{\mathcal{S}}{\query}}{\loss{\query}}>\epsilon \loss{\query}} 
			\leq 12(\pdim+1) \left(\frac{48 \bound}{\epsilon}\right)^{\pdim} \boundrate(\queryset', \epsilon, m)
		\end{equation}
	\end{corollary}
\end{tcolorbox}
	


\begin{proof}
	Observe first that for any function $f$
	\begin{equation*}
		\estloss{\mathcal{S}}{f} = \empdistr{\mathcal{S}}{ \frac{m\query}{\opekernel{m}} }
	\end{equation*}
	Then by Theorem \ref{thm_fixedtheta}, for all $\epsilon >0$ , $m\in \NN$, and $\querytwo \in \querysettwo_m$
	\begin{align*}
		\PP{}{\dnude{\nu}{\empdistr{\mathcal{S}}{\querytwo}}{\meanempdistr{\querytwo}}>\epsilon} 
		&= \PP{}{\dnude{}{\estloss{\mathcal{S}}{\frac{\querytwo\opekernel{m}}{m}}}{\loss{\frac{\querytwo\opekernel{m}}{m}}}>\epsilon} \\
		&\leq  \boundrate(\frac{\querysettwo_m\opekernel{m}}{m}, \epsilon, m) = \boundrate(\queryset', \epsilon, m)
   \end{align*}
   Moreover, we know that $\boundrate$ and $m_\epsilon$ are such that
   \begin{equation*}
		m \geq m_{\epsilon} \implies \boundrate(\queryset, \epsilon, m) \leq 1/2
   \end{equation*}
   
   Therefore $\boundrate$ satisfies the two hypothesis of Theorem \ref{thm_infi_union_bound}, and thus for all $\epsilon >0$ and $m \geq m_\epsilon$
   \begin{align*}
	\PP{}{\exists \query \in \queryset,\ \dnude{\nu}{\estloss{\mathcal{S}}{\query}}{\loss{\query}}>\epsilon \loss{\query}} 
	&= \PP{}{\exists \querytwo \in \querysettwo_m,\ \dnude{\nu}{\empdistr{\mathcal{S}}{\querytwo}}{\meanempdistr{\querytwo}} > \epsilon}\\
	&\leq 12(\pdim+1) \left(\frac{48 \bound}{\epsilon}\right)^{\pdim} \boundrate(\queryset', \epsilon, m)
   \end{align*}
\end{proof}

\note{}{Conclude!}




\section{Proof of Theorem \ref{thm_infi_union_bound}}


We follow a similar proof scheme as in section 9.4 of \cite{haussler1992decisiontheoricgeneralizationofPACmodel}. We specifically revisit Lemma 12. and 13., getting rid of independency hypothesis, and making intermediary results more flexible to further improvements.
\note{}{state of the art slightly better than haussler, cf. Li, but strongly? require independancy}

\begin{tcolorbox}
	\begin{lemma}[Symmetrisation]
		\label{lem_symm}
		Assume the hypothesis of \ref{thm_infi_union_bound}.
		Let furthermore be $\epsilon>0$, $m\in \NN$, and $\mathcal{S}_1, \mathcal{S}_2 \overset{i.i.d.}{\sim} \PP{m}{}$, two multisets of size $m$ independently sampled from the same distribution.\\
		  
		Then for all $m \geq m_{\epsilon/2}$
		\begin{equation*}
			\PP{m}{ \exists \query \in \queryset,\ \dnude{\nu}{\empdistr{\mathcal{S}_1}{\query}}{\meanempdistr{\query}} > \epsilon} \leq 2\PP{m}{\exists \query \in \queryset,\ \dnude{\nu}{\empdistr{\mathcal{S}_1}{\query}}{\empdistr{\mathcal{S}_2}{\query}} > \epsilon/2 }
		\end{equation*}
	\end{lemma}
\end{tcolorbox}


\begin{proof}
	Take $m \geq m_{\epsilon/2}$. Then let be $\mathcal{S}_1$ sampled such that $\exists \query \in \queryset,\ \dnude{\nu}{\empdistr{\mathcal{S}_1}{\query}}{\meanempdistr{\query}} > \epsilon$. This obviously happens with probability $\PP{m}{\exists \query \in \queryset, \dnude{\nu}{\empdistr{\mathcal{S}_1}{\query}}{\meanempdistr{\query}} > \epsilon }$.

	For such an $f$, we then independently sample $\mathcal{S}_2$ such that $\dnude{\nu}{\empdistr{\mathcal{S}_2}{\query}}{\meanempdistr{\query}} \leq \epsilon/2$. Because $m \geq m_{\epsilon/2}$, we know this happens with probability greater than $1/2$, and we thus have
	\begin{align*}
		&\PP{m}{ \exists \query \in \queryset,\ \dnude{\nu}{\empdistr{\mathcal{S}_1}{\query}}{\meanempdistr{\query}} > \epsilon}
		\frac 1 2 \\
		\leq\ &\PP{m}{\exists \query \in \queryset,\ \dnude{\nu}{\empdistr{\mathcal{S}_1}{\query}}{\meanempdistr{\query}} > \epsilon \wedge \dnude{\nu}{\empdistr{\mathcal{S}_2}{\query}}{\meanempdistr{\query}} \leq \epsilon/2 } \\
		\leq\ &\PP{m}{\exists \query \in \queryset,\ \dnude{\nu}{\empdistr{\mathcal{S}_1}{\query}}{\empdistr{\mathcal{S}_2}{\query}} > \epsilon/2 }
	\end{align*}
	where we lastly used the triangular inequality 
	\begin{equation*}
		\dnude{\nu}{\empdistr{\mathcal{S}_1}{\query}}{\meanempdistr{\query}} - \dnude{\nu}{\empdistr{\mathcal{S}_2}{\query}}{\meanempdistr{\query}} \leq   \dnude{\nu}{\empdistr{\mathcal{S}_1}{\query}}{\empdistr{\mathcal{S}_2}{\query}}
	\end{equation*} 
\end{proof}







\begin{tcolorbox}
    \begin{definition}[Covering and Packing]
        Let be $(\queryset, d)$ a metric space. 
        \begin{itemize}
            \item For any $\epsilon> 0$, a subset $\queryset^* \subseteq \queryset$ is said to be $\epsilon$-separated if for all distinct $\query, \query' \in \queryset^*$, $d(\query, \query') > \epsilon$.
            \item The $\epsilon$-packing number on $(\queryset, d)$, denoted by $\packing(\epsilon, \queryset, d)$, is then defined as cardinality of the largest $\epsilon$-separated subset $\queryset^*$ of $\queryset$.
        \end{itemize}
        Intuitively, the $\epsilon$-packing number is the maximal number of balls of radius $\epsilon/2$ that can fit into $\queryset$ without intersecting.
        \begin{itemize}
            \item For any $\epsilon> 0$, a subset $\queryset^*$ of $\queryset$ is said to be an $\epsilon$-cover of $\queryset$ if for all $\query \in \queryset$, it exists $\query^* \in \queryset^*$ with $d(\query, \query') \leq \epsilon$.
            \item The $\epsilon$-covering number on $(\queryset, d)$, denoted by $\covering(\epsilon, \queryset, d)$, is then defined as cardinality of the smallest $\epsilon$-cover of $\queryset$.
        \end{itemize}
        Intuitively, the $\epsilon$-covering number is the minimal number of balls of radius $\epsilon$ than can fill $\queryset$, with possible overlaps.
    \end{definition}

    One can easily check that for all $\epsilon>0$
    \begin{equation}
        \packing(2\epsilon, \queryset, d) \leq \covering(\epsilon, \queryset, d) \leq \packing(\epsilon, \queryset, d)
    \end{equation}
\end{tcolorbox}





\begin{tcolorbox}
    \begin{theorem}[From \cite{haussler1995spherepacking}]
        \label{thm_pack}
        For any set $\mathcal{X}$, any probability distribution $\mu$ on $\mathcal{X}$, any
        set $\queryset \subseteq [0,\bound]^{\mathcal{X}}$ of $\mu$-measurable positive functions on $\mathcal{X}$ bounded by some real $\bound$, and any $\epsilon>0$, one have
        \begin{equation*}
            \packing(\epsilon, \queryset, \dlone{\mu}{}{}) \le e(\pdim+1) \left(\frac{2e \bound}{\epsilon}\right)^\pdim
        \end{equation*}
        where $\pdim=\operatorname{pdim}\queryset$ is the pseudo-dimension of $\queryset$.
    \end{theorem}
\end{tcolorbox}







\begin{tcolorbox}
	\begin{lemma}[Conjectured]
		\label{lem_infi_union_bound}
		Let be $\epsilon >0$, $m\in \NN$, and $\mathcal{S}_1, \mathcal{S}_2 \overset{i.i.d.}{\sim} \PP{}{}$, two multisets of size $m$ independently sampled from the same distribution. We denote their multiset union by $\mathcal{S} := \mathcal{S}_1 \uplus \mathcal{S}_2 \in \binom{\mathcal{X}}{2m}$. Then
		
		\begin{equation*}
			\PP{}{\exists \query \in \queryset,\ \dnude{\nu}{\empdistr{\mathcal{S}_1}{\query}}{\empdistr{\mathcal{S}_2}{\query}} > \epsilon} \leq \sup_{\mathcal{S} \in \binom{\mathcal{X}}{2m} } \covering(\epsilon/4, \queryset, \dlone{\empdistr{\mathcal{S}}{}}{}{}) \sup_{\query \in \queryset} \PP{}{\dnude{\nu}{\empdistr{\mathcal{S}_1}{\query}}{\empdistr{\mathcal{S}_2}{\query}} > \epsilon/4 }
		\end{equation*}
	\end{lemma}
\end{tcolorbox}



\begin{proof}[Draft of Proof]
	Let be $\mathcal{S}$ sampled such that $\exists \query \in \queryset,\ \dnude{\nu}{\empdistr{\mathcal{S}_1}{\query}}{\empdistr{\mathcal{S}_2}{\query}} > \epsilon$. 
	
	Let then be taken $\mathcal{F^*_\mathcal{S}}$, a minimal $\epsilon/4$-cover of $\queryset$ for the $\dlone{\empdistr{\mathcal{S}}{}}{}{}$ topology, then $|\mathcal{F^*_\mathcal{S}}| = \covering(\epsilon/4, \queryset, \dlone{\empdistr{\mathcal{S}}{}}{}{}$. We thus know it exists $f^* \in \mathcal{F^*_\mathcal{S}}$ such that $\dlone{\empdistr{\mathcal{S}}{}}{\query}{\query^*} \leq \epsilon/4$.


	Then triangular inequalities yields that
	\begin{align*}
		\dnude{\nu}{\empdistr{\mathcal{S}_1}{\query}}{\empdistr{\mathcal{S}_2}{\query}} &\leq \dnude{\nu}{\empdistr{\mathcal{S}_1}{\query^*}}{\empdistr{\mathcal{S}_2}{\query^*}} + \dnude{\nu}{\empdistr{\mathcal{S}_1}{\query}}{\empdistr{\mathcal{S}_1}{\query^*}} + \dnude{\nu}{\empdistr{\mathcal{S}_2}{\query}}{\empdistr{\mathcal{S}_2}{\query^*}}  \\
		&\leq \dnude{\nu}{\empdistr{\mathcal{S}_1}{\query^*}}{\empdistr{\mathcal{S}_2}{\query^*}}  + \empdistr{\mathcal{S}_1}{\lvert \query-\query^*\rvert} + \empdistr{\mathcal{S}_2}{\lvert\query-\query^*\rvert}\\
		&\leq \dnude{\nu}{\empdistr{\mathcal{S}_1}{\query^*}}{\empdistr{\mathcal{S}_2}{\query^*}} + 2 \dlone{\empdistr{\mathcal{S}}{}}{\query}{\query^*}\\
		\iff \dnude{\nu}{\empdistr{\mathcal{S}_1}{\query^*}}{\empdistr{\mathcal{S}_2}{\query^*}} &\geq \dnude{\nu}{\empdistr{\mathcal{S}_1}{\query}}{\empdistr{\mathcal{S}_2}{\query}} - 2  \dlone{\empdistr{\mathcal{S}}{}}{\query}{\query^*} > \epsilon/2 
	\end{align*}
	Therefore
	\begin{equation*}
		\PP{}{\exists \query \in \queryset,\ \dnude{\nu}{\empdistr{\mathcal{S}_1}{\query}}{\empdistr{\mathcal{S}_2}{\query}} > \epsilon} \leq \PP{}{\exists \query \in \mathcal{F^*_\mathcal{S}},\ \dnude{\nu}{\empdistr{\mathcal{S}_1}{\query}}{\empdistr{\mathcal{S}_2}{\query}} > \epsilon/4}
	\end{equation*}
	By the law of total expectation, we obtain
	\begin{align*}
		\PP{}{\exists \query \in \mathcal{F^*_\mathcal{S}},\ \dnude{\nu}{\empdistr{\mathcal{S}_1}{\query}}{\empdistr{\mathcal{S}_2}{\query}} > \epsilon/4}
		&=\EE{}{ \1 \{\exists \query \in \mathcal{F^*_\mathcal{S}},\ \dnude{\nu}{\empdistr{\mathcal{S}_1}{\query}}{\empdistr{\mathcal{S}_2}{\query}} > \epsilon/4 \} }\\
		&=\EE{}{ \PP{}{\exists \query \in \mathcal{F^*_\mathcal{S}},\ \dnude{\nu}{\empdistr{\mathcal{S}_1}{\query}}{\empdistr{\mathcal{S}_2}{\query}} > \epsilon/4 \mid \mathcal{F^*_\mathcal{S}}}  }\\
		&=\EE{}{ \PP{}{\bigcup_{\query \in \mathcal{F^*_\mathcal{S}}} \{ \dnude{\nu}{\empdistr{\mathcal{S}_1}{\query}}{\empdistr{\mathcal{S}_2}{\query}} > \epsilon/4 \mid \mathcal{F^*_\mathcal{S}}\} }  }\\		
        &\overset{?}{\leq} \sup_{\mathcal{F^*_\mathcal{S}}} |\mathcal{F^*_\mathcal{S}}| \sup_{\query \in \queryset} \PP{}{\dnude{\nu}{\empdistr{\mathcal{S}_1}{\query}}{\empdistr{\mathcal{S}_2}{\query}} > \epsilon/4 }\\
		&= \sup_{\mathcal{S} \in \binom{\mathcal{X}}{2m} } N(\epsilon/4, \queryset, \dlone{\empdistr{\mathcal{S}}{}}{}{} ) \sup_{\query \in \queryset} \PP{}{\dnude{\nu}{\empdistr{\mathcal{S}_1}{\query}}{\empdistr{\mathcal{S}_2}{\query}} > \epsilon/4 }
	\end{align*}
$\overset{?}{\leq}$ indicates this inequality is still to be proven. It consists of bounding the union probability of random event other a random set. Since we can bound uniformly both, event probability, and set cardinality, it could seem intuitive this random union bound hold. However, the conditioning by $\mathcal{F^*_\mathcal{S}}$ makes this bound non trivial and require further assumptions.

\end{proof}


From this we are now able to prove Theorem \ref{thm_infi_union_bound}. 
\begin{proof}
	Let be $\epsilon>0$ and take $m \geq m_{\epsilon/2}$. Combining Lemmas \ref{lem_symm} and \ref{lem_infi_union_bound} gives
	\begin{align*}
		\PP{}{ \exists \query \in \queryset,\ \dnude{\nu}{\empdistr{\mathcal{S}_1}{\query}}{\meanempdistr{\query}} > \epsilon} 
		&\leq 2\PP{}{\exists \query \in \queryset,\ \dnude{\nu}{\empdistr{\mathcal{S}_1}{\query}}{\empdistr{\mathcal{S}_2}{\query}} > \epsilon/2 }\\
		&\leq 2 \sup_{\mathcal{S} \in \binom{\mathcal{X}}{2m} } \covering(\epsilon/8, \queryset, \dlone{\empdistr{\mathcal{S}}{}}{}{}) \sup_{\query \in \queryset} \PP{}{\dnude{\nu}{\empdistr{\mathcal{S}_1}{\query}}{\empdistr{\mathcal{S}_2}{\query}} > \epsilon/8 }
	\end{align*}

In order to bound the last term, first consider applying the union bound
	\begin{align*}
		\PP{}{\dnude{\nu}{\empdistr{\mathcal{S}_1}{\query}}{\empdistr{\mathcal{S}_2}{\query}} > \epsilon/8 } 
		&\leq \PP{}{\dnude{\nu}{\empdistr{\mathcal{S}_1}{\query}}{\meanempdistr{\query}} + \dnude{\nu}{\meanempdistr{\query}}{\empdistr{\mathcal{S}_2}{\query}}> \epsilon/8 }\\
		&\leq \PP{}{\dnude{\nu}{\empdistr{\mathcal{S}_1}{\query}}{\meanempdistr{\query}} > \epsilon/16 \vee  \dnude{\nu}{\meanempdistr{\query}}{\empdistr{\mathcal{S}_2}{\query}}> \epsilon/16 }\\
		&\leq \PP{}{\dnude{\nu}{\empdistr{\mathcal{S}_1}{\query}}{\meanempdistr{\query}} > \epsilon/16 \vee  \dnude{\nu}{\meanempdistr{\query}}{\empdistr{\mathcal{S}_2}{\query}}> \epsilon/16 }\\
		&\leq 2\PP{}{\dnude{\nu}{\empdistr{\mathcal{S}_1}{\query}}{\meanempdistr{\query}} > \epsilon/16 }\\
		&\leq 2 \boundrate(\queryset, \epsilon/16, m)
	\end{align*}
Second, we know from Theorem \ref{thm_pack}
\begin{align*}
	\covering(\epsilon/8, \queryset, \dlone{\empdistr{\mathcal{S}}{}}{}{})
	\leq \packing(\epsilon/8, \queryset, \dlone{\empdistr{\mathcal{S}}{}}{}{})
	\leq e(\pdim+1) \left(\frac{16e \bound}{\epsilon}\right)^\pdim
\end{align*}

The two precedent bound does not depend on $\mathcal{S}$ and $f$, and therefore 
\begin{align*}
	\PP{}{ \exists \query \in \queryset,\ \dnude{\nu}{\empdistr{\mathcal{S}_1}{\query}}{\meanempdistr{\query}} > \epsilon} 
	&\leq 4e(\pdim+1) \left(\frac{16e \bound}{\epsilon}\right)^\pdim \boundrate(\queryset, \epsilon/16, m)\\
	&\leq 12(\pdim+1) \left(\frac{48 \bound}{\epsilon}\right)^\pdim \boundrate(\queryset, \epsilon/16, m)\\
\end{align*}
which is the desired result.
\end{proof}













\begin{tcolorbox}[colback=red!10,title= Useless?]
	We define $\forall a,b \geq 0$ and $\forall \nu >0$
	\begin{equation}
		\dnu{\nu}{a}{b} := \frac{|a-b|}{a+b+\nu}
	\end{equation}
	
	From that definition, one can easily check
	\begin{proposition}
		\label{prop_dnu}
		For all $\nu>0$, the function $d_\nu$ verifies the following properties
		\begin{itemize}
			\item $d_\nu$ is a distance i.e. it verifies positivity, separation, symmetry and triangular inequality.
			\item $\forall a,b \geq 0,\ 0\leq \dnu{\nu}{a}{b} < \min( \frac{|a-b|}{a}, 1)$
			\item Moreover, if $\exists \bound\geq 0$ such that $a,b \le \bound$, then 
			\begin{equation*}
				\frac{|a-b|}{\nu + 2\bound} \le \dnu{\nu}{a}{b} \le \frac{|a-b|}{\nu}
			\end{equation*}
		\end{itemize}
	\end{proposition}


    
    Using properties \ref{prop_dnu} of $\dnude{\nu}{}{}$ distance yields that
    \begin{align*}
        \dnude{\nu}{\empdistr{\mathcal{S}_1}{\query}}{\empdistr{\mathcal{S}_1}{\query^*}} + \dnude{\nu}{\empdistr{\mathcal{S}_2}{\query}}{\empdistr{\mathcal{S}_2}{\query^*}}
        &\le \frac{\lvert \empdistr{\mathcal{S}_1}{\query-\query^*}\rvert}{\nu} + \frac{\lvert \empdistr{\mathcal{S}_2}{\query-\query^*}\rvert}{\nu} \\
        &\le \frac{2}{\nu}  \dlone{\empdistr{\mathcal{S}}{}}{\query}{\query^*}
    \end{align*}
\end{tcolorbox}