
\chapter{Introduction to coresets}
\label{chap_intro_coresets}
\section{Motivations}

% ##################
% REPLACE MULTISET BY SEQUENCE
% ##################

A common if not the standard approach in machine learning
is to formulate learning problems as optimization problems.

Let $\mathcal{X}=\begin{Bmatrix}
x_{i} \mid i\in \intint{1}{n}
\end{Bmatrix}$ be a multiset (possibly with repetitions) of $n$ data points. Let $\qset$ be a space of functions called queries defined on $\mathcal{X}$, and $\query$ an element of $\qset$. Classical learning problem aims to find a solution $\query ^*$ in $\qset$ that minimizes a cost function $\loss{}$ over the given data $\mathcal{X}$. In this work, we focus on cost functions that are positive and additively decomposable, i.e. we consider cost functions of the form
\begin{equation}
    \label{eqn_lossquery}
\loss{\query}:=\sum_{x \in \mathcal{X}} \query (x),
\end{equation}
where the queries $\query \in \qset \subseteq \RR_+^{\mathcal{X}}$ are positive functions defined on $\mathcal{X}$.

A large amount of machine learning problems falls into the framework \cref{eqn_lossquery}, including support vector machines, logistic regression, linear regression and k-means clustering. 
\begin{example}[$k$-means]  
    The goal of Euclidean $k$-means clustering is to find a set of $k$ cluster centers $\mathcal{C}\subseteq \RR^d$ minimizing the quantization error
    \begin{equation*}
    \loss{\query} = \sum_{x \in \mathcal{X}} \min_{q \in \mathcal{C} } \lVert x - q \rVert^2_2.
    \end{equation*}
    In this case, $\query$ is the squared distance to the nearest cluster center $q$ in a set of cluster centers $\mathcal{C}$. Formally, 
    \begin{equation*}
        \qset = \left\{\query : x \mapsto \min_{q \in \mathcal{C} } \lVert x - q \rVert^2_2 \mid \mathcal{C}  \in \binom{\mathcal{X}}{k}\right\}
    \end{equation*}
    where $\binom{\mathcal{X}}{k}$ denotes ``from $\mathcal{X}$ choose $k$'', the set of all subsets of $\mathcal{X}$ of size $k$.
\end{example}

\begin{example}[linear regression]
    The goal of ordinary least squares linear regression is to find a vector $a \in \RR^d$ and a scalar $b \in \RR$ that minimizes the sum of squares 
    \begin{equation*}
        \loss{\query} = \sum_{(y,z)\in \mathcal{X}} (a\T y +b -z)^2
    \end{equation*}
    where the data is $\mathcal{X} =\begin{Bmatrix}x_i :=(y_{i}, z_i) \mid i\in \intint{1}{n}\end{Bmatrix}$ with for all $i\in \intint{1}{n}$, $y_i \in \RR^d$ and $z_i \in \RR$. In this case, we have \begin{equation*}
        \qset = \left\{(y,z) \mapsto (a\T y + b -z)^2 \mid a \in \RR^d, b \in \RR\right\} 
    \end{equation*}
\end{example}
    

In many machine learning applications, the induced optimization problem can be hard to solve. Given a learning task, if an algorithm is too slow on large datasets, one can either speed up the algorithm or reduce the amount of data.
The second alternative is theoretically guaranteed by the "coresets" idea.
A coreset is a weighted subset of the original data with the assurance that, up to a controlled relative error, the task's estimated cost function on the coreset will match the cost calculated on the complete dataset for any learning parameter.

An elegant outcome of such property is the ability to execute learning algorithms only on the coreset, assuring nearly-equal performance while significantly reducing the computational cost. There are many algorithms that generate coresets, some of which are more specialized and are designed for a particular purpose (such as k-means, k-medians, logistic regression, etc.). Additionally, keep in mind that there are results for coresets in both the streaming and offline settings. We will focus here on the offline setting.


\section{The coreset property}

The key idea behind coresets is to approximate the original data
set $\mathcal{X}$ by a weighted set $\mathcal{S}$ which satisfies the coreset property. Such property then guarantee $1+\epsilon$-approximations.

Let $\mathcal{S}=\begin{Bmatrix}
x_{i} \mid i\in \intint{1}{m}
\end{Bmatrix}$ be a submultiset of $\mathcal{X}$, and to any element $x \in \mathcal{S}$, associate a weight $\omega\left(x\right) \in \mathbb{R}^{+}$ that only depends on the value $x \in  \mathcal{X}$. Define the estimated cost based on the weighted multiset $\mathcal{S}$ as
\begin{equation*}
    \estloss{\mathcal{S}}{\query}:=\sum_{x \in \mathcal{S}} \omega\left(x\right) \query(x).
\end{equation*}

The aim of this estimator is ``to approximate'' $\loss{\query}$ defined in \cref{eqn_lossquery}. Depending on the context, there are plenty of meaning ``to approximate'' can get. We focus especially on the coreset property.
\begin{tcolorbox}
    
    \begin{definition}[Coreset]
        \label{def_coresetprop}
        Let $\epsilon \in {]}0,1{]}$ and $\query \in \qset$. We say $\mathcal{S}$ is an $\epsilon$-coreset for $\query$ if the estimated cost based on $\mathcal{S}$ is equal to the exact cost up to a relative error $\epsilon$. Formally
        \begin{equation}
            \label{eqn_querycoresetprop}
            \left|\frac{\estloss{\mathcal{S}}{\query}}{\loss{\query}}-1\right| \le \epsilon.
        \end{equation}
        We say $\mathcal{S}$ is an $\epsilon$-coreset for $\qset$ if it is a $\epsilon$-coreset for any $\query \in \qset$. Formally
        \begin{equation}
            \label{eqn_qsetcoresetprop}
            \forall \query \in \qset,\ \left|\frac{\estloss{\mathcal{S}}{\query}}{\loss{\query}}-1\right| \le \epsilon.
        \end{equation}
    \end{definition}
\end{tcolorbox}

An important consequence of the coreset property is the following

\begin{tcolorbox}
    \begin{theorem}
        \label{thm_optcoreset}
        Let $\mathcal{S}$ be an $\epsilon$-coreset for $\qset$. Define $\query^*:=\min_{\query \in \qset}\loss{\query}$ and $\hat \query^*:=\min_{\query \in \qset}\estloss{\mathcal{S}}{\query}$. Then $\loss{\hat\query^*} $ is an $(1+3\epsilon)$-approximation of $\loss{\query^*}$, i.e.
    
        \begin{equation*}
            \loss{\query^*} \le {L}( \hat\query^*)\leq (1+ 3 \epsilon)\loss{\query^*}.
        \end{equation*}
    \end{theorem}
\end{tcolorbox}
\begin{proof}
    $\mathcal{S}$ being a $\epsilon$-coreset for $\qset$ yields \cref{eqn_qsetcoresetprop}, which is equivalent to 
    \begin{equation*}
        \forall \query \in \qset ,\ (1-\epsilon) \loss{\query} \le \estloss{\mathcal{S}}{\query} \le(1+\epsilon) \loss{\query}.
    \end{equation*}
    In particular, this is true for $\query^*$ and $\hat \query^*$ thus
    \begin{equation}
        (1-\epsilon) \loss{\query^*} \le(1-\epsilon) \loss{\hat\query^*} \le \estloss{\mathcal{S}}{\hat\query^*} \le \estloss{\mathcal{S}}{\query^*} \le(1+\epsilon) \loss{\query^*},
    \end{equation}
    and moreover
    \begin{equation*}
        \loss{\query^*} \le {L}( \hat\query^*) \le \frac{(1+\epsilon)}{(1-\epsilon) } \loss{\query^*} \leq (1+ 3 \epsilon)\loss{\query^*}.
        \end{equation*}
\end{proof}
A key consequence of that theorem is that one can minimize on the estimated loss $\estloss{\mathcal{S}}{}$ and still guarantee a low error on the true loss, even when the size of $\mathcal{S}$ is small before the size of $\mathcal{X}$. Therefore, it makes coreset very relevant in a machine learning context, and  inscribes them into a more general learning framework that is PAC learning.

\section{Coresets and PAC learning}
In computational learning theory, probably approximately correct (PAC) learning is a framework for mathematical analysis of machine learning. It was proposed in 1984 in \cite{valiant1984learnable}. The first idea is that a learning problem can be formulate into an expected risk minimization. In another words, by learning, one is interested in minimizing errors over a distribution of guesses it would have to make. To do so, the learner will receives samples and must select a prediction function based on them. The PAC framework states that the learner ability can be quantified by how probable (the "probably" part) the learner have a low generalization error (the "approximately correct" part) in some sense.



In that framework, several practical issues can occur. 
\begin{enumerate}
    \item The richness of the considered class of prediction function can be too small to embrace the complexity of the studied phenomena.
    \item The risk optimizer algorithm could struggle finding the minimizing function, for instance only finding local minima, or yielding high computational complexity.
    \item The sample complexity required for reaching a given level of "probable" in the approximately correctness can vary.
\end{enumerate}

However, a learning problem is generally not separable into these three issues. This means their resolution is not independent and had to be tackled jointly. For instance, making more expressive a class of prediction function can make its optimization more difficult, or make the sample complexity required higher. The latter case is well known as overfitting.


The use of coresets tackles the last two issues. It aims to reduce the number of samples required to compute an optimal prediction function, and still control the error of the optimization step. The computational complexity of an optimizer being often linked to the number of samples, it is by the way reduced. In general, if the time complexity for an optimization algorithm to optimize on $n$ data points is $\OO(a_n)$, and that it takes $\OO(b_m)$ time to sample an $\epsilon$-coreset which is of size $m \le n$, then we have interest in building coreset as soon as $\OO(a_n) \geq \OO(b_m) + \OO(a_m)$.



% \subsection{Link with coresets}

% Let us see how coresets can naturally intervene into the PAC framework. Formally, let be given a probability distribution $\PP{}{}$ generating the data $\mathcal{S}$, and let be $\qset$ a family of loss functions. Minimizing the expected loss is equivalent to finding $\query^*:= \arg \min_{\query \in \qset} \loss{\query}$. Because we are only given $\mathcal{S}$ and not the full distribution, we have to approximate $\loss{\query}$ by some estimate $\estloss{\mathcal{S}}{\query}$ based on the data, and then minimizing it with $\hat{\query}^* := \arg \min_{\query\in \qset} \estloss{\mathcal{S}}{\query}$. 


% In order to evaluate this scheme, we fix some $\epsilon>0$, and we want with the highest probability as possible that the relative error of $\loss{\hat\query^*}$ against $\loss{\query^*}$ is less than $\epsilon$. Put differently we want
% \begin{equation*}
% 	\mathbb{P}\left[|\loss{\hat\query^*} - \loss{\query^*}| \ge \epsilon \loss{\query^*}\right] \leq \delta
% \end{equation*}
% for the smallest $\delta$ as possible.

% But we know a sufficient condition to control this error, that's the coreset property. Indeed, suppose we sample a data set $\mathcal{S}$ such that $\mathcal{S}$ is an $\epsilon/3$-coreset for $\qset$ with probability at least $1-\delta$. Formally
% \begin{equation*}
%     \PP{}{\forall f \in \qset,\ |\frac{\estloss{\mathcal{S}}{\query}}{\loss{\query}} - 1| \leq \epsilon/3} \geq 1- \delta.
% \end{equation*}
% Then by \cref{thm_optcoreset} we have with probability at least $1-\delta$ (w.p. $1-\delta$), that
% \begin{equation*}
%     \loss{\query^*} \leq \loss{\hat\query^*} \leq (1+ 3 \epsilon/3) \loss{\query^*} \iff
%     |\loss{\hat\query^*} - \loss{\query^*}| \leq \epsilon \loss{\query^*}.
% \end{equation*}

% We thus see that the PAC framework translates to approximating with high probability the evaluation of a function on a data subset, which is guaranteed by the coreset property. 

% On another hand, the use of coresets leverage one of the three issues that occur in PAC learning, that is to reduce the number of samples required to compute an optimal prediction function, and still controlling the error. If the time complexity for an optimization algorithm to optimize on $n$ data points is $\OO(a_n)$, and that it takes $\OO(b_m)$ time to sample an $\epsilon$-coreset which is of size $m \le n$, then we have interest in building coreset as soon as $\OO(a_n) \geq \OO(b_m) + \OO(a_m)$.





    

\section{State-of-the-art results on coresets}
The existence of coresets is trivial, the original data set $\mathcal{X}$ itself  being a $0$-coreset, taking all its elements weighted by 1. The key question is the existence of small coresets where the coreset size is sublinear, if not independent, in the number of data points $n$, while at the same time having slow rate with respect to other parameters, in particular $d$ the dimension of data, $\epsilon$ the desired error, and in the case where the coreset is obtained probabilistically, $\delta$ the probability bound of not being a coreset.


\subsection{Importance multinomial sampling}

In the stochastic case, a well-established approach to coreset construction is importance multinomial sampling. Given any distribution $q$ on $\mathcal{X}$, one can sample a sequence $\mathcal{S} \in \mathcal{X}^m$ from the multinomial distribution of size $m$ based on $q$, $\mathcal S \sim \mathcal M(m, q)$, i.e. $m$ i.i.d. sampling of $X$ such that $\forall x \in \mathcal{X},\ \PP{}{X=x} = q(x)$.

By the importance sampling trick, an unbiased estimator of $\loss{\query}$ is then
\begin{equation*}
	\estloss{\textrm{iid}}{\query} := \sum_{x\in \mathcal S} \frac{\query(x)}{m q(x)}.
\end{equation*}
And its variance is
\begin{equation*}
	\Var{\textrm{iid}}{\query} :=\frac{1}{m} \Var{}{\frac {\query(x)} {q(x)}}
	=\frac{1}{m} \sum_{x \in \mathcal{X}} \frac{\query(x)^{2}}{q(x)} -\frac{1}{m} \loss{\query}^{2}
\end{equation*}
where $Q = \operatorname{diag}(q)$ and $\moones = \voones \voones \T$ the matrix full of ones. 


Now observe that for any query $\query \in \qset$, the variance is reduced to 0 by taking
\begin{equation*}
    q_{\query} :=\frac{ \query}{\loss{\query}} = x \mapsto \frac{ \query(x)}{\loss{\query}}.
\end{equation*}
Of course, attempting to sample from $q_\query$ is quiet limited in practice. First, we would prefer not having to make our sampling depend on the query function $\query$. Second and main obstacle is that using $q_\query$ implies already knowing $\loss{\query}$, for which we are supposedly looking an approximation for via building coreset. In \cref{sect_senstsampl} we see one way to bypass these two limitations.

\subsection{Sensitivity sampling}
\label{sect_senstsampl}



Intuitively, in order to build a coreset of small size, we want to only select data points that are relevant. This means that for a given $x \in \mathcal{X}$, we want to make its probability to be sampled as small as possible, unless it plays a relevant role in the evaluation of $\loss{\query}$ for some $\query$, which translates to $\frac{\query}{\loss{\query}}$ being high.

The idea of \cite{langberg2010_universal_approximator} is thus to take for every $x$, the sampling probability $q_\query$ in the worst case $\query$, i.e. for which $x$ is the most relevant in the evaluation of $\loss{\query}$. Formally, they define the following notion of sensitivity.
\begin{tcolorbox}
    \begin{definition}[Sensitivity]
        The sensitivity $\sigma(x)$ of a data point $x \in \mathcal{X}$ with respect to $\qset$ is defined as
        \begin{equation*}
            \sigma(x) = \sup _{\query \in \qset} \frac{{\query}\left(x\right)}{L(\query)} \ \in[0,1].
        \end{equation*}
        and the total sensitivity with respect to $\qset$ as $\mathfrak{S}=\sum_{x\in \mathcal{X}} \sigma(x) \ \in[1,n]$.
    \end{definition} 
\end{tcolorbox}

If we were now too sample $x$ from a distribution proportional to the sensitivity $\PP{}{X=x} \propto \sigma(x)$, it would not depend on $\query$. However, we would still need to know $\loss{\query}$ for every $\query$, in order to compute it.

But assume that we know an upper bound $s$ on sensitivity $\sigma$ i.e. $\forall x \in \mathcal{X}, s(x) \geq \sigma(x)$, and define $S := \sum_{x\in \mathcal{X}} s(x) \geq \mathfrak{S}$. We can then sample $x$ from a distribution proportional to $s$, i.e.  $\mathcal S \sim \mathcal M(m, s/S)$, and we have the following result.


\begin{tcolorbox}
    \begin{theorem}[Hoeffding bound for fixed query]
        \label{thm_hoeffdingfixedquery}
        Let $m\in \NN$ and $\mathcal S \sim \mathcal M(m, s/S)$. 

		Then for all $\epsilon >0 $ and all $\query \in \qset$
		\begin{equation*}
			\PP{}{\dnude{\nu}{\estloss{\textrm{iid}}{\query}}{\loss{\query}}>\epsilon\loss{\query}} \leq 2 \exp \left(-2 m \epsilon^2/S^2\right).
		\end{equation*}
		
		
		Moreover, for all $\delta>0$ 
		\begin{equation*}
            m \geq \frac{S^{2}}{2 \epsilon^{2}} \log \frac{2}{\delta}
			\implies 
			\text{$\mathcal{S}$ is an $\epsilon$-coreset for $\query$ w.p. $1-\delta$}
		\end{equation*}
    \end{theorem}
\end{tcolorbox}

\begin{proof}
    Define $g_s(x) :=  \frac{\query(x)}{s(x) L(\query)}  \, \in[0,1]$.
    Because of boundedness, we can apply Hoeffding inequality, and have for any $\query \in \qset$ and $\epsilon >0$
    \begin{equation*}
        \mathbb{P}\left[\left|\frac{1}{m} \sum_{x \in \mathcal{S}} g_s(x) - \mathbb{E}\left[g_s(x)\right]\right|>\epsilon/S\right] \leq 2 \exp \left(-2 m \epsilon^2/S^2\right).
    \end{equation*}
    Furthermore, $\mathbb{E}\left[g_s(x)\right]=\frac{1}{S}$ and $\frac{1}{m} \sum_{x \in \mathcal{S}} g_s(x)=\frac{\estloss{\textrm{iid}}{\query}}{S L(\query)}$, thus
    \begin{equation*}
        \mathbb{P}\left[|\estloss{\textrm{iid}}{\query} - L(\query)|>\epsilon L(\query)\right] \leq 2 \exp \left(-2 m \epsilon^2/S^2\right).
    \end{equation*}
    Hence, $\mathcal{S}$ satisfies \cref{eqn_querycoresetprop} i.e. $\mathcal{S}$ is an $\epsilon$-coreset for $\query$  with probability at least $1-\delta$, if we choose
    \begin{equation*}
        m \geq \frac{S^{2}}{2 \epsilon^{2}} \log \frac{2}{\delta}.
    \end{equation*}
\end{proof}



\newpage


The required number of samples depends quadratically on $S$ the upper bound on total sensitivity $\mathfrak{S}$. Hence, the tighter the bound on the sensitivity $s \geq \sigma$ is, the less samples are required. If one take the looser bound $s=1$, so $S=n$, this implies $m \gtrsim n^2$ which is useless in practice.

Note that total sensitivity $\mathfrak{S} \in [1,n]$ depends on the richness of function space $\qset$. 
\begin{itemize}
    \item $\mathfrak{S}=1$ if and only if for all $x\in\mathcal{X}$, $\sigma(x)=1/n$. This means $\qset$ only contains constant functions.
    \item $\mathfrak{S}=n$ if and only if for each $x\in \mathcal{X}$, there exists some $\query \in \qset$ such that for any other element $y$ of $\mathcal{X}$, $f(y)=0$.
\end{itemize}

Fortunately, it is possible to compute tight bounds on the sensitivity for many machine learning problems. For instance, \cite{lucic2016_lineartime_detection_via_sensitivity} show that the sensitivity bound for $k$-means problem is $\mathfrak{S} = \Theta(k)$. When $k = n-1$, the query set $\qset$ contains all functions that are zero on $n-1$ points, taking $n-1$ points to be centers of their own cluster, and we recover the worst case $S = n$.

\vspace{1cm}

\subsection{Extension to all queries}

In previous section, we obtained a sample complexity bound on obtaining coreset for a single query $\query$. But in order to obtain an $\epsilon$-coreset for $\qset$, the $\epsilon$-coreset for $\query$ must holds simultaneously for all queries $\query \in \qset$.

Intuitively, we would like to invoke a union bound argument/Bonferroni correction over all queries $\query \in \qset$. In the case where $\qset$ is finite, applying union bound on the previous bound from \cref{thm_hoeffdingfixedquery} yields
\begin{equation*}
    \PP{}{\dnude{\nu}{\estloss{\textrm{iid}}{\query}}{\loss{\query}}>\epsilon\loss{\query}} \leq 2 |\qset|\exp \left(-2 m \epsilon^2/S^2\right).
\end{equation*}
Hence, for all $\delta>0$ 
\begin{equation*}
    m \geq \frac{S^{2}}{2 \epsilon^{2}} \log \frac{2|\qset|}{\delta}
    \implies 
    \text{$\mathcal{S}$ is an $\epsilon$-coreset for $\query$ w.p. $1-\delta$}
\end{equation*}

However, in the case $\qset$ is not finite, this bound diverge. We give here the intuition how a finite bound can still be obtained and we will detail this method in \cref{sec_extension_all_queries}.

The key idea is to construct a set $\qset'_\epsilon \subseteq \qset$ which still is finite, and that approximate the set $\qset$ with $\epsilon$ granularity. It would further imply that if $\mathcal{S}$ is an $\epsilon$-coreset for $\qset$, then it would be an $\epsilon'$-coreset for $\qset'_\epsilon$, for controlled $\epsilon'$, and then one could apply union bound on $\qset'_\epsilon$ instead.

This would summon the size of $\qset'_\epsilon$ into the obtained bound, which depends on the richness of the set $\qset$ it approximate. Richness in that case can be quantified through pseudo-dimension, which generalize the VC-dimension to real-valued function sets.

\newpage


\begin{tcolorbox}
	\begin{definition}[pseudo-dimension]
        \label{def__pseudodim}
		The pseudo-dimension of a set $\threeqset$ of functions defined on $\mathcal{X}$, denoted by $\operatorname{pdim}\threeqset$, is the largest $\pdim$ such that 
	\begin{itemize}
		\item there exists $(x_{i})_{i\in \intint{1}{\pdim}} \subseteq \mathcal{X}^\pdim$, a sequence of $\pdim$ elements from $\mathcal{X}$,
		\item there exists $(t_i)_{i\in \intint{1}{\pdim}} \subseteq  \RR^\pdim$ a sequence of $\pdim$ real thresholds,
		\item such that for each $(b_i)_{i\in \intint{1}{\pdim}} \subseteq \{0,1\}^\pdim$
		\item there is an $\query \in \threeqset$ such that $\forall i \in \intint{1}{\pdim}$, we have $\query(x_i) \geq r_i \iff b_i = 1$. 
	\end{itemize}
	Put differently it always exists functions in $\threeqset$ to have values above or below some threshold for every $2^\pdim$ combinations of above/below.
\end{definition}
Pseudo-dimension can also be defined through VC-dimension. Indeed, considering the function
\begin{align*}
	\operatorname{above}_\query \colon \mathcal{X} \times \RR &\to \{0,1\}\\
	(x,r) &\mapsto \1\{f(x) \geq r\}
\end{align*}
we have
\begin{equation}
	\operatorname{pdim}\threeqset := \operatorname{VCdim}\{\operatorname{above}_\query \mid \query \in \threeqset\}
\end{equation}
\end{tcolorbox}

Let now introduce the function space
\begin{equation}
    \label{eqn__g_sensitivity}
    \twoqset_s := \frac{\qset}{s\loss{\qset}} \ =\left\{x \mapsto \frac{\query(x)}{s(x)\loss{\query}} \mid \query \in \qset\right\} \subseteq [0,1]^{\mathcal{X}}.
\end{equation}
It can be shown the size of previously mentioned $\qset'_\epsilon$ is $\OO(\epsilon^{-\pdim})$ where $\pdim:=\operatorname{pdim}\twoqset_s$. Applying union bound then leads to
\begin{equation*}
	m \gtrsim \frac{S^{2}}{\epsilon^{2}} (d'\log\frac{1}{\epsilon} + \log \frac{1}{\delta})
\end{equation*}
where $y \gtrsim x$ is a transitive notation for $y = \Omega(x)$ i.e. $y$ is lower bounded by $x$ up to a constant factor. 
%Likewise, we denote $y \lesssim x \iff y = \OO(x)$ and $y \asymp x \iff y = \Theta(x)$.

This result presented here follows from seminal works on theoretic generalizations of the PAC model from \cite{haussler1992decisiontheoricgeneralizationofPACmodel}.
Improving that scheme, more recent results from \cite{li2001_improved_bound_sample_complexity} involving chaining arguments leads to
\begin{equation*}
	m \gtrsim \frac{S^{2}}{\epsilon^{2}} (d' + \log \frac{1}{\delta}).
\end{equation*}
Moreover, they shown this bound to be tight in the i.i.d. sampling framework, with respect to $\epsilon$ and $\delta$.
Finally, \cite{braverman2016coresetsota} improved it with respect to $S$ by showing that under the same framework
\begin{equation}
    \label{eqn__sota_coreset}
	m \gtrsim \frac{S}{\epsilon^{2}} (d' \log S + \log \frac{2}{\delta}).
\end{equation}
As an example, we saw that in the $k$-means case, $\mathfrak{S} = \Omega(k)$, and it can be shown that for $d$-dimensional data, $\pdim = \OO(dk\log k)$, which leads to
\begin{equation*}
    m \gtrsim \frac{k}{\epsilon^{2}} (dk \log^2 k + \log \frac{2}{\delta}).
\end{equation*}
We refer to \cite{bachem2017coresetML} for further insights on sensitivity and pseudo-dimension bounding methods.
