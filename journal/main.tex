\documentclass{report} % [fleqn, leqno]: flush left equations, left equation numbers


% if you need to pass options to natbib, use, e.g.:
%     \PassOptionsToPackage{numbers, compress}{natbib}
% before loading neurips_2021

% ready for submission
% \usepackage{style}
\usepackage[nonatbib]{style_nips}

% to compile a preprint version, e.g., for submission to arXiv, add add the
% [preprint] option:
%     \usepackage[preprint]{neurips_2021}

% to compile a camera-ready version, add the [final] option, e.g.:
%     \usepackage[final]{neurips_2021}

% to avoid loading the natbib package, add option nonatbib:
%    \usepackage[nonatbib]{neurips_2021}
% neurips
\usepackage[utf8]{inputenc} % allow utf-8 input
\usepackage[T1]{fontenc}    % use 8-bit T1 fonts
\usepackage{hyperref}       % hyperlinks
\usepackage{url}            % simple URL typesetting
\usepackage{booktabs}       % professional-quality tables
\usepackage{amsfonts}       % blackboard math symbols
\usepackage{nicefrac}       % compact symbols for 1/2, etc.
\usepackage{microtype}      % microtypography
\usepackage{xcolor}         % colors
% formatting
% \usepackage[fontsize=11pt, parindent=0pt]{fontsize} % or change in style.sty
\usepackage{setspace} % spacing, \setstretch{}
\usepackage{longtable} % longtables
\usepackage{titlesec}
    \titlespacing{\paragraph}{0pt}{3.25ex}{3pt}
% writing
\usepackage{amsmath} % most maths
\usepackage{graphicx} % figures 
\usepackage{subcaption} % subfigures
\usepackage{dsfont} % \mathds
\usepackage{amsthm} % \theoremstyle
\usepackage{cleveref} % clever referencing
\usepackage{amssymb} % \nsucc
\usepackage{stmaryrd} % \llbracket
    \SetSymbolFont{stmry}{bold}{U}{stmry}{m}{n} % compatibility between amsmath and stmaryrd
\usepackage{tcolorbox} % \begin{tcolorbox}[colframe=red!25,colback=blue!10,title=blue!60!black,title= More Test]\end{tcolorbox} red!25!black means 25% red, 75%black
\usepackage[colorinlistoftodos,prependcaption,textsize=small]{todonotes} % \todo
\usepackage[style=authoryear-comp, maxbibnames=8, maxcitenames=1, sorting=ydnt, uniquelist=false, citetracker=true, backend=biber]{biblatex} % backend=biber/bibtex, authoryear-comp for compact, else authoryear
    \renewcommand*{\revsdnamepunct}{} % no comma between given and family name
    \AtEveryCitekey{\ifciteseen{}{\defcounter{maxnames}{8}}} % maxnames authors for first cite, else maxcitenames
    \DeclareNameAlias{sortname}{family-given} % family then given name
    \addbibresource{biblio.bib}
% TODO: respectful citation, e.g. MSSW2018 instead of Munteanu2018, for Alexander Munteanu, Chris Schwiegelshohn, Christian Sohler, and David P. Woodruff. 2018


    % ###################
    % general commands
    \renewcommand{\epsilon}{\varepsilon}
    \renewcommand{\phi}{\varphi}
    \newcommand{\dd}{\mathrm{d}}
    \newcommand{\NN}{\mathbb{N}}
    \newcommand{\RR}{\mathbb{R}}
    \newcommand{\CC}{\mathbb{C}}
    \newcommand{\PP}[2]{\mathbb{P} \ifstrempty{#1}{}{_{#1}} \ifstrempty{#2}{}{\left[#2\right]}}
    \newcommand{\EE}[2]{\mathbb{E} \ifstrempty{#1}{}{_{#1}} \ifstrempty{#2}{}{\left[#2\right]}}
    \newcommand{\OO}{\mathcal{O}}
    \newcommand{\Var}[2]{\operatorname{\mathbb{V}ar} \ifstrempty{#1}{}{_{#1}} \ifstrempty{#2}{}{\left[#2\right]}}
    \newcommand{\Cov}[2]{\operatorname{\mathbb{C}ov} \ifstrempty{#1}{}{_{#1}} \ifstrempty{#2}{}{\left[#2\right]}}
    \newcommand{\T}{^\top}  % or ^\intercal
    \newcommand{\1}{\mathds{1}} % indicator
    \newcommand{\voones}{\boldsymbol{j}} % vector of ones
    \newcommand{\moones}{\boldsymbol{J}} % matrix of ones
    \newcommand{\intint}[2]{\llbracket #1,#2 \rrbracket} % integer interval
    \newcommand{\note}[2]{\todo[linecolor=red,backgroundcolor=red!25,bordercolor=red, #1]{#2}} % \note{inline}{"note"}
    % Show fontsize: \fontname\font\ at \the\fontdimen6\font 

% ##################
% amsthm package settings
\newtheorem{theorem}{Theorem}
\newtheorem{lemma}{Lemma}
\newtheorem{corollary}{Corollary}
\newtheorem{conjecture}{Conjecture}
\newtheorem{proposition}{Proposition}
\theoremstyle{definition} % no italics in definition
\newtheorem{definition}{Definition} % [section] for reset counting in each sections 
\newtheorem{example}{Example}
% \numberwithin{equation}{chapter} % chapter already by default



% #################
% specifics commands for this report
\newcommand{\query}{f}
\newcommand{\twoquery}{g}
\newcommand{\threequery}{h}
\newcommand{\qset}{\mathcal{F}}
\newcommand{\twoqset}{\mathcal{G}}
\newcommand{\threeqset}{\mathcal{H}}
\newcommand{\loss}[1]{L \ifstrempty{#1}{}{(#1)}}
\newcommand{\estloss}[2]{\hat{L}_{#1} \ifstrempty{#2}{}{(#2)}}
\newcommand{\empdistr}[2]{\mu_{#1} \ifstrempty{#2}{}{(#2)}}
\newcommand{\meanempdistr}[1]{\mu \ifstrempty{#1}{}{(#1)}}
\newcommand{\dnu}[3]{d_{#1}  \ifstrempty{#2#3}{}{(#2,#3)}}
\newcommand{\dnude}[3]{\lvert\ifstrempty{#2#3}{\cdot}{#2-#3}\rvert}
\newcommand{\dlone}[3]{d_{L^1(#1)} \ifstrempty{#2#3}{}{(#2,#3)}}
\newcommand{\lipschitz}{\boldsymbol\ell}
\newcommand{\bound}{M}
\newcommand{\boundrate}{\mathcal{R}}
\newcommand{\covering}{\mathbf{C}}
\newcommand{\packing}{\mathbf{P}}
\newcommand{\pdim}{{d'}}
\newcommand{\opekernel}[1]{\tilde{\boldsymbol K}_{#1}}
% #################



\begin{document}
    \title{Journal}

% \maketitle
\begin{titlepage}
    \begin{center}
        \vspace*{25pt} {
        \begin{spacing}{0.05}
            \rule{400pt}{2pt}\\
            \rule{400pt}{0.75pt}\\
        \end{spacing}
        \vspace{20pt}
        \begin{spacing}{0.9}
            \fontsize{23.6pt}{26pt}\selectfont%
            \textsc{\textbf{Improving Sample Complexity of Coresets with Determinantal Point Processes}}\\%
        \end{spacing}
        \vspace{5pt}
        \begin{spacing}{0.05}
            \rule{400pt}{0.75pt}\\
            \rule{400pt}{2pt}\\
        \end{spacing}
        }
    
        \vspace*{1cm}
        \begin{large}
        \textit{By}\\%
        \end{large}
    
    
        \vspace*{4pt}
        \begin{Large}
        \textbf{Hugo Simon-Onfroy}\\%
        \end{Large}
    
    \begin{large}
        \vspace*{5cm}
        \textbf{
        A Master MVA internship report }
        \vspace*{6pt}\\
        under the supervision of
        \vspace*{20pt}\\
    
        \begin{longtable*}{ p{0.4\textwidth} p{0.5\textwidth} }
            R\'emi Bardenet & CNRS \& Université de Lille
            \tabularnewline Subhroshekhar Ghosh & National University of Singapore
            \tabularnewline
        \end{longtable*}
    \end{large}
    
    
        \vspace*{1cm}
    
    
        \begin{figure}[!ht]
            \centering
            \begin{subfigure}[t]{.5\linewidth}
                \centering
                \includegraphics[width=\linewidth]{pics/logoCRIStAL.png}
            \end{subfigure}
            \begin{subfigure}[t]{.4\linewidth}
                \centering
                \includegraphics[width=0.6\linewidth]{pics/logoENS_PS.jpg}
            \end{subfigure}
        \end{figure}
    \end{center}
    \end{titlepage}

    

\begin{abstract}

    The last several years have seen the rise of data sets of extraordinary scale across a variety of scientific fields. Such massive data sets bring new challenges, because of the high cost of collecting, maintaining, and interpreting them. Existing well-proven methods can become computationally impractical when dealing with millions or even billions of data points, and when they may no longer fit on a single system but instead need to be stored on clusters of servers. New algorithms are therefore required to scale to this massive data setting. 
    
    While one could concentrate on specific machine learning issues and develop many new algorithms, we concentrated in this internship on a more comprehensive strategy. We investigate coresets, which are compressed descriptions of potentially enormous data sets, and that still guarantee provable computational properties.

    Thanks to the careful selection of sample distribution, i.i.d. random sampling has proven to be one of the most effective techniques for creating coresets. However, independent samples can sometimes be extremely redundant, and one could hope that
    enforcing diversity would yield better performances. 
    
    Because proving coreset property in a general non-i.i.d. framework is difficult, we rely on determinantal point processes (DPPs), a class of point processes that naturally introduce diversity through repulsiveness. DPPs are interesting because they are a rare example of repulsive point processes with tractable theoretical properties, while still being sufficiently expressive to address a variety of problems.

    This report is an attempt to provide improved sample complexity results on coresets by the use of well chosen DPP sampling. We start with an inventory on coreset literature, then we bring to light determinantal point processes as an improvement way. Thirdly, we qualitatively justify the use of DPPs over existing sampling methods. We finish by quantitative results, conclusion, and perspectives.
    
\end{abstract}

\tableofcontents
\setstretch{1.2}




\chapter{Introduction to coresets}

\section{Motivations}

% Let $\mathcal{X}=\begin{Bmatrix}
%     x_{i} \mid i\in \intint{1}{n}
% \end{Bmatrix}$ be a multiset (possibly with repetitions) of $n$ data points. Let $\mathcal{F}$ be a space of parameters, or queries, and $f$ an element of $\mathcal{F}$. We consider cost functions of the form
% \begin{equation*}
%     \EE f = \frac{1}{|\mathcal{X}|}\sum_{x \in \mathcal{X}} f(x)    
% \end{equation*}


% Let $\mathcal{S}=\begin{Bmatrix}
%     x_{i} \mid i\in \intint{1}{m}
% \end{Bmatrix}$ be a submultiset of $\mathcal{X}$. To each element $x \in \mathcal{S}$, associate a weight $\omega\left(x\right) \in \mathbb{R}^{+}$. Define the estimated cost associated to the weighted submultiset $\mathcal{S}$ as
% \begin{equation*}
%     \EE f_{\mathcal{S}}=\frac{1}{|\mathcal{S}|}\sum_{x \in \mathcal{S}} \omega\left(x\right) f(x)
% \end{equation*}

A common if not the standard approach in machine learning
is to formulate learning problems as optimization problems.

Let $\mathcal{X}=\begin{Bmatrix}
x_{i} \mid i\in \intint{1}{n}
\end{Bmatrix}$ be a multiset (possibly with repetitions) of $n$ data points. Let $\Theta$ be a space of parameters, or queries, and $\theta$ an element of $\Theta$. Given the data $\mathcal{X}$ and a space of possible solutions Q, one aims to find a solution $\theta^{\text {opt }}$ that minimizes a cost function $L$. In this work, we focus on cost functions that are additively decomposable, i.e. we consider cost functions of the form
\begin{equation*}
L(\theta):=\sum_{x \in \mathcal{X}} f_\theta(x)
\end{equation*}
for some function $f_\theta \in \{f_\theta \mid \theta \in \Theta\}$.

A large amount of machine learning problems falls into that framework, including support vector machines, logistic regression, linear regression and k-means clustering. For example, the goal of the euclidean k-means clustering is to find a set of k cluster centers in $\RR^d$ minimizing the quantization error
\begin{equation*}
L(\theta) = \sum_{x \in \mathcal{X}} \min_{q \in \theta} \lVert x - q \rVert^2_2
\end{equation*}
In this case, $\theta \in \binom{\mathcal{X}}{k}$, the set of all subsets of $\mathcal{X}$ of size $k$, and $f_\theta =  \min_{q \in \theta} \lVert x - q \rVert^2_2$

In many machine learning applications, the induced optimization problem can be hard to solve. Given a learning task, if an algorithm is too slow on large datasets, one can either speed up the algorithm or reduce the amount of data.
The second alternative is theoretically guaranteed by the "coresets" idea.
A coreset is a weighted subset of the original data with the assurance that, up to a controlled relative error, the task's estimated cost function on the coreset will match the cost calculated on the complete dataset for any learning parameter.

An elegant outcome of such property is the ability to execute learning algorithms only on the coreset, assuring nearly-equal performance while significantly reducing the computational cost. There are other algorithms that generate coresets, some of which are more specialized and are designed for a particular purpose (such as k-means, k-medians, logistic regression, etc.). Additionally, keep in mind that there are results for the coreset in both the streaming and offline settings. Nevertheless, we will concentrate here on the offline setting.


\section{The coreset property}

The key idea behind coresets is to approximate the original data
set $\mathcal{X}$ by a weighted set $\mathcal{S}$ which satisfies the coreset property. Such property then guarantee $1+\epsilon$-approximations.

Let $\mathcal{S}=\begin{Bmatrix}
x_{i} \mid i\in \intint{1}{m}
\end{Bmatrix}$ be a submultiset of $\mathcal{X}$. To each element $x \in \mathcal{S}$, associate a weight $\omega\left(x\right) \in \mathbb{R}^{+}$. Define the estimated cost associated to the weighted submultiset $\mathcal{S}$ as
$$
\hat{L}(\theta):=\sum_{x \in \mathcal{S}} \omega\left(x\right) f_\theta(x)
$$
\begin{definition}[Coreset]
    Let $\epsilon \in {]}0,1{[}$. $\mathcal{S}$ is a $\epsilon$-coreset for $L$ if, for any query $\theta$, the estimated cost is equal to the exact cost up to a relative error, i.e. for all $\theta \in \Theta$
    \begin{equation}
        \left|\frac{\hat{L}(\theta)}{L(\theta)}-1\right| \le \epsilon 
        \label{def_coresetprop}
    \end{equation}
\end{definition}
An important consequence of the coreset property is the following
\begin{theorem}
    Let be $\mathcal{S}$, a $\epsilon$-coreset for $L$. Define $\theta^{\text {opt }}:=\min_{\theta \in \Theta}L(\theta)$ and $\hat \theta^{\text {opt }}:=\min_{\theta \in \Theta}\hat L(\theta)$. Then $L( \hat \theta^{\text {opt }}) $ is an $(1+\epsilon)$-approximation of $L( \theta^{\text {opt }})$, i.e.
    \begin{equation*}
        L( {\theta}^{\text {opt }}) \le {L}( \hat{\theta}^{\text {opt }})\leq (1+ 3 \epsilon)L( \theta^{\text {opt }})
    \end{equation*}
    \label{thm_optcoreset}
\end{theorem}

\begin{proof}
    If $\mathcal{S}$ is a $\epsilon$-coreset for $L$, we have from \ref{def_coresetprop} that 
    \begin{equation}
        (1-\epsilon) L(\theta^{\text {opt }}) \le(1-\epsilon) L( \hat{\theta}^{\text {opt }}) \le \hat{L}( \hat{\theta}^{\text {opt }}) \le \hat{L}( \theta^{\text {opt }}) \le(1+\epsilon) L( \theta^{\text {opt }})
    \end{equation}
    and moreover
    \begin{equation}
        L( {\theta}^{\text {opt }}) \le {L}( \hat{\theta}^{\text {opt }}) \le \frac{(1+\epsilon)}{(1-\epsilon) } L( \theta^{\text {opt }}) \leq (1+ 3 \epsilon)L( \theta^{\text {opt }})
        \end{equation}
\end{proof}
This key property makes coreset very relevant in a machine learning context, and  inscribes them into a more general learning framework that is PAC learning.

\section{Coresets and PAC learning}
\subsection{PAC learning
}In computational learning theory, probably approximately correct (PAC) learning is a framework for mathematical analysis of machine learning. It was proposed in 1984 in \cite{valiant1984learnable}. The first idea is that a learning problem can be formulate into an expected risk minimization. In another words, by learning, one is interested in minimizing errors over a distribution of guesses it would have to make. To do so, the learner will receives samples and must select a prediction function based on them. The PAC framework states that the learner ability can be quantified by how probable (the "probably" part) the learner have a low generalization error (the "approximately correct" part) in some sense.



In that framework, several practical issues can occur. 
\begin{itemize}
    \item The richness of the considered class of prediction function can be too small to embrace the complexity of the studied phenomena.
    \item The risk optimizer algorithm could struggle finding the minimizing function, for instance only finding local minima, or yielding high computational complexity.
    \item The sample complexity required for reaching a given level of "probable" in the approximately correctness can vary.
\end{itemize}

However, a learning problem is generally not separable into these three issues. This means their resolution is not independent and had to be tackled jointly. For instance, making more expressive a class of prediction function can make its optimization more difficult, or make the sample complexity required higher. The latter case is well known as overfitting.

\subsection{Link with coresets}

Let us see how coresets can naturally intervene into the PAC framework. Formally, let be given a probability distribution $\PP$ generating the data $\mathcal{S}$, and let be $\mathcal{F}$ a family of loss functions. Minimizing the expected loss is equivalent to finding $f^*:= \arg \min_{f\in \mathcal{F}} \EE f$. Because we are only given $\mathcal{S}$ and not the full distribution, we have to approximate $\EE f$ by some estimate $\EE f_{\mathcal{S}}$ based on the data, and then minimizing it with $\hat{f}^* := \arg \min_{f\in \mathcal{F}} \EE f_{\mathcal{S}}$. 


In order to evaluate this scheme, we fix some $\epsilon>0$, and we want with the highest probability $1-\delta$ as possible
\begin{equation*}
    |\EE \hat f^* - \EE f^*| \le \epsilon
\end{equation*} 
Put differently we want
\begin{equation*}
	\mathbb{P}\left[|\EE \hat f^* - \EE f^*| \ge \epsilon\right] \leq \delta
\end{equation*}

But we know a sufficient condition to control this error, that's the coreset property! Indeed, if we have sample a data set $\mathcal{S}$ such that
\begin{equation*}
    \PPP{}{\forall f \in \mathcal{F},\ |\frac{\EE f_{\mathcal{S}}}{\EE f} - 1| \leq \epsilon/3} \geq 1- \delta
\end{equation*}
i.e. $\mathcal{S}$ is an $\epsilon/3$-coreset for $\EE$ with probability $1-\delta$, then by \ref{thm_optcoreset} we know 
\begin{equation*}
    \EE f^* \leq \EE \hat f^* \leq (1+ 3 \epsilon/3) \EE f^* \iff
    |\EE \hat f^* - \EE f^*| \leq \epsilon
\end{equation*}

We thus see that the PAC framework translates to approximating with high probability the evaluation of a function on a data subset, which is guaranteed by the coreset property. 

On another hand, the use of coresets leverage one of the three issues that occur in PAC learning, that is to reduce the number of samples required to compute an optimal prediction function, and still controlling the error. If the time complexity for an optimization algorithm to optimize on $n$ data points is $O(a_n)$, and that it takes $O(b_m)$ time to sample an $\epsilon$-coreset which is of size $m \le n$, then we have interest in building coreset as soon as $O(a_n) \geq O(b_m) + O(a_m)$.







\section{State-of-the-art results on coresets}
\begin{definition}[Sensitivity]
	The sensitivity $\sigma_i$ of a data point $x_{i}$ and the total sensitivity $\mathfrak S$ of $\mathcal X$ are
	$$
	\begin{cases}
		\sigma_{i}=\sup_{\theta \in \Theta} q_{\theta}(x_i) = \sup _{\theta \in \Theta} \frac{f_{\theta}\left(x_{i}\right)}{L(\theta)} \quad \in[0,1]\\
		\mathfrak{S}=\sum_{i=1}^{n} \sigma_{i}
	\end{cases}
	$$
\end{definition} 

\subsection{Main proof}
Let be $s$ an upper bound on sensitivity $\sigma$ i.e. $\forall i, s_i \geq \sigma_i$, and $S := \sum_{i=1}^n s_i$. Furthermore, let be sampled  $\mathcal S \sim \mathcal M(m, s/S)$, the multinomial sampling case. Define $g_\theta(x_i) := \frac{q_\theta(x_i)}{s_i} = \frac{f_{\theta(x_i)}}{s_i L(\theta)}  \, \in[0,1]$

By Hoeffding's inequality, we thus have for any $\theta \in \Theta$ and $\epsilon^{\prime}>0$
\begin{equation}
	\mathbb{P}\left[\left|\frac{1}{m} \sum_{x \in \mathcal{S}} g_{\theta}(x) - \mathbb{E}\left[g_{\theta}(x)\right]\right|>\epsilon^{\prime}\right] \leq 2 \exp \left(-2 m \epsilon^{\prime 2}\right)
\end{equation}
and by definition, $\mathbb{E}\left[g_{\theta}(x)\right]=\frac{1}{S}$ and $\frac{1}{m} \sum_{x \in \mathcal{S}} g_{\theta}(x)=\frac{\hat L_{\textrm{iid}}(\theta)}{S L(\theta)}$, thus
\begin{equation*}
	\mathbb{P}\left[|\hat L_{\textrm{iid}}(\theta) - L(\theta)|>\epsilon^{\prime} S L(\theta)\right] \leq 2 \exp \left(-2 m \epsilon^{\prime 2}\right)
\end{equation*}
Hence, $\mathcal{S}$ satisfies the $\epsilon$-coreset property \ref{def_coresetprop} for any single query $\theta \in \Theta$ with probability at least $1-\delta$, if we choose
\begin{equation}
	m \geq \frac{S^{2}}{2 \epsilon^{2}} \log \frac{2}{\delta}
\end{equation}


\subsection{Extension to all queries}
\note{}{developp sota}
See \textbf{Uniform guarantee for all queries} in \cite{bachem2017coresetML}. Introducing the pseudo-dimension $d'$, it gives
\begin{equation}
	m \geq \OO(\frac{S^{2}}{2 \epsilon^{2}} (d' + \log \frac{2}{\delta}))
\end{equation}

See \textbf{Theorem 5.5} of \cite{braverman2016coresetsota} for an improved bound.
\begin{equation}
	m \geq \OO(\frac{S}{2 \epsilon^{2}} (d' \log S + \log \frac{2}{\delta}))
\end{equation}

See \cite{bachem2017coresetML}.

\chapter{Determinantal Point Processes}
\label{chap_DPP}
We saw in \cref{chap_intro_coresets} the uniform bound on sample complexity of coresets. These bounds are based on Probabilistically Approximately Correct (PAC) learning theory results, where sample complexity bounds (with respect to $\delta$ and $\epsilon$) are known and tight in the i.i.d. framework. One could then be tempted to extend to the case where samples are drawn dependently.

Whereas the space of distributions for $m$ i.i.d samples has size that does not depend on $m$, the way $m$ samples can be correlated grows exponentially with $m$. In order to tackle this space of correlations, more recent results restricted to the cases of martingales or $\beta$-mixing processes, e.g. \cite{gao2016_learnability_beta_mixing}.

In the current chapter, we introduce Determinantal Point Processes (DPPs), another restriction of correlated sampling, which admits useful tractability properties, while still maintaining expressiveness into the sub-category of negatively correlated sampling. This negative correlation is expected to be a key property in order to perform better sample complexity.

\section{Some intuition}

A Determinantal Point Processes (DPP) is a random sampling over subsets of a given ground set. Noticeably, its distribution is entirely encoded by a given positive kernel, which can be tuned to a range of specific contexts. In a sense, DPPs can be said to be the kernel machine of point processes, as they allow both tractability and flexibility. 

An essential characteristic of a DPP is that the occurrences of the element of the ground set are negatively correlated, i.e. the inclusion of one item makes the inclusion of other items less likely. The strengths of these negative correlations are derived from a kernel matrix that defines a global measure of similarity between pairs of items, so that more similar items are less likely to co-occur. As a result, DPPs assign higher probability to sets of items that are diverse.




\section{Definition}
Determinantal Point Processes are before all point processes, which can be described as processes for selecting a collection of mathematical points randomly located on a mathematical space. Formally, a point process $\PP{}{}$ on a ground set $\mathcal{X}$ is a probability measure over "point patterns" or "point configurations" of $\mathcal{X}$, which are subsets of $\mathcal{X}$. For instance, $\mathcal{X}$ could be a continuous region of the euclidean plane in which a scientist injects some quantum particles trapped into a potential well. Then $\PP{}{\left\{x_1, x_2, x_3\right\}}$ characterizes the likelihood of seeing these particles at places $x_1, x_2$, and $x_3$. Depending on the type of the particles, the measurements might tend to cluster together, or they might occur independently, or they might tend to spread out into space. $\PP{}{}$ captures these correlations.

In the following, we focus on finite point processes, where we assume without loss of generality that $\mathcal{X}=\begin{Bmatrix}
    x_{i} \mid i\in \intint{1}{n}
    \end{Bmatrix}$, in this setting we sometimes refer to elements of $\mathcal{X}$ as items. The discrete setting is computationally simpler and often more appropriate for real-world data. We refer to \cite{hough2006_hkpv} for a review of DPPs in the continuous case.

In the discrete case, a point process is simply a probability measure on $2^{\mathcal X}$ i.e. the power set of $\PP{}{}$ i.e. the set of all subsets of $\mathcal{X}$. A sample from $\PP{}{}$ might be the empty set, the entirety of $\mathcal{X}$, or anything in between. 
\begin{tcolorbox}
    \begin{definition}[Determinantal Point Process]
        \label{def_dpp}
        $\PP{}{}$ is called a determinantal point process if, when $\mathcal{S}$ is a random subset drawn according to $\PP{}{}$, we have, for every $A \subseteq \mathcal{X}$,
        \begin{equation}
            \PP{}{A \subseteq \mathcal{S}}=\operatorname{det}K_A
        \end{equation}
        for some real, symmetric matrix $K \in \RR^{n \times n}$ indexed by the elements of $\mathcal{X}$.
    \end{definition}
    Here, $K_A :=$ $\left[K_{x y}\right]_{x, y \in A}$ denotes the submatrix of $K$ indexed by elements of $A$, and we adopt $\operatorname{det}K_\emptyset=1$. Note that normalization is unnecessary here, since we are defining marginal probabilities that need not sum to $1$ .
\end{tcolorbox}

Since $\PP{}{}$ is a probability measure, all principal minors $\operatorname{det}K_A$ of $K$ must be positives, and thus $K$ itself must be positive (for Loewner order $\preceq$). It is possible to show in the same way that the eigenvalues of $K$ are bounded above by one. These requirements turn out to be sufficient. By the Macchi-Soshnikov theorem from \cite{macchi1975dpp},any $K$ such that $0 \preceq K \preceq I$ defines a DPP.

We refer to $K$ as the marginal kernel since it contains all the information needed to compute the probability of any subset $A$ being included in $\mathcal{S}$. A few simple observations follow from \cref{def_dpp}. If $A=\{x\}$ is a singleton, then we have
\begin{equation}
    \label{eqn__marginal}
	\PP{}{x \in \mathcal{S}}=K_{x x}
\end{equation}
also denoted $K(x,x)$. That is, the diagonal of $K$ gives the marginal probabilities of inclusion for individual elements of $\mathcal{X}$. Diagonal entries close to 1 correspond to elements of $\mathcal{X}$ that are almost always selected by the DPP. Furthermore, if $A=\{x, y\}$ is a two-element set, then
\begin{equation}
    \label{eqn_paircorrel}
	\begin{aligned}
        \PP{}{\{x, y\} \subseteq \boldsymbol{X}} &=\left|\begin{array}{ll}
	K_{x x} & K_{x y} \\
	K_{y x} & K_{y y}
	\end{array}\right| \\
	&=K_{x x} K_{y y}-K_{x y} K_{y x} \\
	&=\PP{}{x \in \boldsymbol{X}} \PP{}{y \in \boldsymbol{X}}-K_{x y}^2
	\end{aligned}
\end{equation}
Thus, the off-diagonal elements determine the negative correlations between pairs of elements: large values of $K_{x y}$ imply that $x$ and $y$ tend not to co-occur.

\cref{eqn_paircorrel} demonstrates why DPPs are "diversifying". If we think of the entries of the marginal kernel as measurements of similarity between pairs of elements in $\mathcal{X}$, then highly similar elements are unlikely to appear together. If $K_{x y}=\sqrt{K_{x x} K_{y y}}$, then $i$ and $j$ are "perfectly similar" and do not appear together almost surely. Conversely, when $K$ is diagonal there are no correlations and the elements appear independently. Note that DPPs cannot represent distributions where elements are more likely to co-occur than if they were independent: correlations are always negative.





A DPP of kernel $K$ can have a random sample size. From \cref{eqn__marginal}, we know that the average number of samples is equal to the trace of $K$, Formally
\begin{equation*}
    \EE{}{|\mathcal{S}|} = \EE{}{\sum_{x \in \mathcal{X}} \1\{x \in \mathcal{S}\}} = \sum_{x \in \mathcal{X}} \PP{}{x \in \mathcal{\mathcal{S}}} = \sum_{x \in \mathcal{X}}K_{xx}=  \operatorname{Tr}K.
\end{equation*} 
But in many cases, one prefers to specify deterministically the number of samples, instead of having a random number of them. This leads to the definition of $m$-DPP.
\begin{tcolorbox}
    \begin{definition}[$m$-DPP]
        A $m$-DPP is a DPP conditioned to a fixed sample size $m$. Therefore, it is a probability distribution supported on $\binom{\mathcal{X}}{m}$ only.
    \end{definition}
\end{tcolorbox}


Finally, we introduce a subclass of DPPs called projective DPPs that admits useful properties.

\begin{tcolorbox}
    \begin{definition}[Projective DPP]
        A DPP is projective if every eigenvalue of its kernel is in $\{0,1\}$. This is equivalent to the kernel being a projection matrix.
    \end{definition}
\end{tcolorbox}

Projective DPPs are sometimes called elementary DPPs, because it turns out any DPP can be written as a mixture of projective DPP, see Lemma 2.6. from \cite{kulesza2012_dpp_for_ml}. 

For a DPP of kernel $K$, being a projective DPP is equivalent to having deterministic sample size. We know this size is equal to $\operatorname{Tr}K$ which in that case is by definition the rank of $K$. As a corollary, the set of projective DPPs of rank $m$ are precisely the intersection of the set of DPPs and the set of $m$-DPPs.  


\section{Examples}

DPPs occur naturally in some simple random models. Obviously, any independent sampling of elements of a set is trivially a (diagonal) DPP. But maybe the simpler non-trivial instance of a DPP is the descents in random sequences.


Take a sequence of $N$ random numbers drawn uniformly and independently from a finite set e.g. the digits, $\intint{0}{9}$. The locations in the sequence where the current number is less than the previous number form a subset of  $\intint{2}{N}$. Noticeably, this subset is distributed as a DPP. Intuitively, if the current number is less than the previous number, it is probably not too large, thus it becomes less likely that the next number will be smaller yet. In this sense, the positions of decreases repel one another.

Edges in uniform spanning trees, eigenvalues of random matrices, as well as some quantum experimental models are also well-known instances of DPP. By the way, and for the history, DPPs were first identify as a class by Macchi, who called them "fermion process" because they give the distributions of fermion systems at thermal equilibrium. The Pauli exclusion principle states that no two fermions can occupy the same quantum state; as a consequence fermions exhibit what is known as the "anti-bunching" effect. This repulsion is described precisely by a DPP.

\begin{figure}[!ht]
    \centering
    \includegraphics[width=0.6\linewidth]{pics/dpp_vs_iid.png}
    \caption{(left) A set of points in the plane drawn from a DPP, with $K_{x y}$ inversely related to the distance between points $i$ and $j$. (right) The same number of points sampled independently using a Poisson point process , which results in random clumping.}
    \label{fig_dpp_vs_iid}
\end{figure}



\section{Geometric interpretation}
DPPs are defined on determinants, that have an intuitive geometric interpretation. Since a DPP kernel $K$ is symmetric, there exists $V \in \RR^{d \times n}$ such that $K= V V\T$. 
Denote the columns of $V$ by $(V_x)$ for $x\in \mathcal{X}$. Then $\forall A \subseteq \mathcal{X}$

\begin{equation}
    \PP{}{A \subseteq \mathcal{S}} = \operatorname{Vol}^2(V_A)
\end{equation}

The right hand side is the squared $|A|$-dimensional volume of the parallelepiped spanned by the columns of $V$ corresponding to elements in $A$.

Intuitively, we can think of the columns of $V$ as feature vectors describing the elements
of $\mathcal{X}$. Then the kernel $K$ measures similarity using dot products between feature vectors, and \cref{def_dpp} says that the probability assigned by a DPP to the inclusion of a set $A$ is related to the volume spanned by its associated feature vectors. This is illustrated in \cref{fig_geometric_interpret}.

\begin{figure}[!ht]
    \centering
    \includegraphics[width=0.8\linewidth]{pics/geometric_interpret.png}
    \caption{from \cite{kulesza2012_dpp_for_ml}. A geometric interpretation of a DPP relates each column of $V$ to an element of $\mathcal{X}$. (a) The  probability of inclusion of a subset $A$ is the square of the volume spanned by its associated feature vectors. (b) As the magnitude of an item's feature vector increases, so do the probabilities of sets containing that item. (c) As the similarity between two items increases, the probabilities of sets containing both of them decrease.}
    \label{fig_geometric_interpret}
\end{figure}

This geometric interpretation explain why diverse sets are more probable. It is because their feature vectors are more orthogonal, and hence span larger volumes. Conversely, items with parallel feature vectors are selected together with probability zero, since their feature vectors define a degenerate parallelepiped. Ceteris paribus, items with large magnitude feature vectors are more likely to appear, because the spanned volume for sets containing them evolves linearly with respect to their magnitude, and thus the probability evolves quadratically with respect to it.








\section{Sampling from a DPP}
\label{sec__sampling_DPP}
\subsection{Exact DPP sampling}


Although DPPs can be impressively efficient given the exponential number of subsets being sampled from, sampling can be rapidly limited by performance. 
Except for a few specialised kernels like the edges in uniform spanning trees mentioned previously, the default exact sampler is a spectral algorithm due to \cite{hough2006_hkpv}.

It leverages the fact that DPPs are mixtures of projective DPPs to generate repeated samples given the spectral content of the kernel. This method is commonly called the spectral method since it requires the spectral/eigendecomposition of the positive kernel. 

Formally, if a DPP is defined by a kernel $K$ defined on $n$ data points, one requires the eigendecomposition $K= V V\T$  where $V \in \RR^{d \times n}$. 
This can often be the computational bottleneck since it generally requires $O(n^3)$ time. Note however that for some DPPs based on specific kernels like OPE kernels, $K$ is built via this decomposition and thus it is trivially known.

In any cases when multiple samples are required, this eigendecomposition can be reused. Then each sample from the spectral algorithm requires only $O(n m^2)$ time, where $m$ is the number of elements sampled. This means $O(n (\operatorname{Tr} K)^2)$ time on average. If $K$ is a projective kernel, $m = \operatorname{Tr} K = d$ which is a constant than can be small in many practical applications, e.g. in a recommendation context, $k$ would often be less than 10.

Some recent works from \cite{gillenwater2019_treebased_fast_dpp_sampling} improved somewhat this complexity. Based on the still needed eigendecomposition, it implements a binary tree structure storing appropriate summary statistics of the eigenvectors, requiring $O(n d^2)$ to build, but can then generate repeated
samples in $O(\log(n)m^2d^2 + d^3)$ time, hence $O(\log(n)d^4)$ for a projective kernel. Therefore, this method becomes a viable alternative to the spectral method when the total
number of items $n$ is large and when the dimensionality $d$ of the
features and the expected sample size $\operatorname{Tr} K$ are small compared to $n$.





\subsection{Approximate DPP sampling}
Several sampling methods have been developed in the case we only need an approximated DPP sampling.

A first class of methods involves a kernel approximation of a given DPP kernel, using random projections such as in \cite{kulesza2012_dpp_for_ml}, or low-rank factorization techniques.

A second class involves Monte Carlo Markov Chain (MCMC). This is often down in an inexact fashion using target distribution close but different from a DPP one. Noticeably, \cite{gautier2017_zonotope_for_dpp_sampling} proposed an exact MCMC sampler for projective DPPs.


\chapter{Correlated importance sampling}
\label{chap_correlated_sampling}



We saw in \cref{chap_DPP} that DPPs are a restriction of correlated sampling that admits useful tractability properties. Moreover, DPPs still maintain expressiveness into the sub-category of negatively correlated sampling, which is the kind of processes we expect to perform better for sample complexity. The intuition is that negatively correlated sampling can eliminate redundancy in sampling sets, an independent sampling can not.

In this chapter, we present current results on coreset sampling with DPPs, and show qualitative results on variance reduction from DPPs. 



\section{A first result with DPPs}


\cite{tremblay2018dppcoreset} first introduce DPPs into the coreset problems, based on the idea of diversity sampling. Their results holds for both DPPs and $m$-DPPs. Since projection DPPs are precisely the intersection of both DPPs and $m$-DPPs, all results apply to them. For the sake of conciseness, we state here their result for $m$-DPPs and we refer to their article for the DPP case.

\begin{tcolorbox}
	\begin{theorem}[\cite{tremblay2018dppcoreset}]
		\label{thm__tremblay}
		Let $m \in \NN$, $K_m$ a $m$-DPP kernel and let sample $\mathcal{S} \sim \mathcal{DPP}(K_m)$. Assume that the query space $\qset$ is parametrized by some $\theta \in \Theta$, and that all Lipschitz constant with respect to $\theta$ of $\query_\theta \in \qset$ are bounded by some $\lipschitz := \sup_{x \in \mathcal{X}} \operatorname{Lip}\left\{\theta \mapsto \query_\theta(x)\right\}$.\\

		If the minimal sensitivity satisfies $\min_{x\in \mathcal{X}}\sigma(x) \geq 1/n$, then for all $\epsilon, \delta \in [0,1]$ 
		\begin{equation*}
            m \geq \frac{32}{\epsilon^{2}}\left(\max_{x\in \mathcal{X}}\frac{m\sigma(x)}{K_m(x,x)}\right)^2 \log \frac{4\eta}{\delta}
			\implies 
			\text{$\mathcal{S}$ is an $\epsilon$-coreset for $\qset$ w.p. $1-\delta$}
		\end{equation*}
		where $\eta$ is the minimal number of balls of radius $\frac{\epsilon \inf_{f}\loss{f}}{6 n \lipschitz}$ necessary to cover $\Theta$.
	\end{theorem}
\end{tcolorbox}
Note first that the fraction $\frac{m}{K_m(x,x)}$ appearing in the right hand side of the bound is due to the correlated importance sampling framework. This fraction does not appear in the i.i.d. framework because the numerator $m$ cancel with the marginal intensity $mq(x)$. In practice, this fraction can be bounded uniformly on $m$, because $K_m(x,x)$ would typically grow linearly with $m$.

Also note that typically $\log \eta = \OO\left(\pdim \log \frac{n}{\epsilon \inf_{f}\loss{f}}\right)$ with $\pdim = \operatorname{pdim}\qset$, and therefore depends on $n$ and $\epsilon$.


Thus, the obtained bound for DPPs of \cref{thm__tremblay} does not improve the sample complexity bound on coreset size in i.i.d. framework, from \cite{braverman2016coresetsota}. There are two reasons for this. 

\begin{itemize}
	\item First, the result crucially relies on a concentration inequality for strongly Rayleigh measures (especially DPPs) from \cite{pemantle2011rayleighconcentration}, which does not improve Hoeffding bound used in \cref{chap_intro_coresets}. 

	However, one important fact is that it doesn't rely on more advanced concentration for projective DPPs from \cite{breuer2013nevai} that involves the variance of the estimator. Since recent results from \cite{bardenet2020mcdpp}, it is known DPPs can improve variance rate, and we hope this result to be leveraged into an improved bound on coreset for fixed query.

	\item Second, the argument to generalize to all queries from \cite{tremblay2018dppcoreset} introduce a $\log \epsilon^{-1}$ term, and foremost a dependency in $n$, through $\eta$. If not tackled, this could ruin the effort finding improved bound for fixed queries. An improvement way would be to extend classical VC theory arguments in a correlated context.
\end{itemize}


Despite these mitigated results on concentrations, DPPs has already been shown to perform variance reduction, e.g. \cite{bardenet2020mcdpp}. In the following \cref{sec__variance_arguments}, we present qualitative variance reductions in favour of DPP and $m$-DPP sampling, against Bernoulli process sampling and multinomial sampling.


\section{Variance arguments}
\label{sec__variance_arguments}
We express variance formulas in four sampling cases: multinomial, DPP, Bernoulli process, and $m$-DPP. Then we compare these variances under a domination criteria.
\subsection{Four sampling cases}
\label{subsec__foursampl}
\paragraph{In the multinomial case}, we have $\mathcal S \sim \mathcal M(m, q)$. Then an unbiased estimator of $L$ is
\begin{equation*}
	\estloss{\textrm{iid}}{\query} := \sum_{x\in \mathcal S} \frac{\query(x)}{m q(x)}
\end{equation*}
and its variance is
\begin{equation*}
	\Var{\textrm{iid}}{\query} :=\frac{1}{m} \Var{}{\frac {\query(x)} {q(x)}}
	=\frac{1}{m} \sum_{x \in \mathcal{X}} \frac{\query(x)^{2}}{q(x)} -\frac{1}{m} \loss{\query}^{2} = \boldsymbol\query\T(\frac{Q^{-1}} m - \frac{\moones} m)\boldsymbol\query
\end{equation*}
where $\boldsymbol\query := (f(x))_{x\in \mathcal{X}}$, $Q := \operatorname{diag}(q)$ and $\moones := \voones \voones \T$ the matrix full of ones. 


\paragraph{In the DPP case}, we have $ \mathcal S \sim \mathcal{DPP}(K)$, and for all $x \in \mathcal{X}$, we denote its marginals $\pi_x := K_{xx}$. Then an unbiased estimator of $L$ is
\begin{equation*}
	\estloss{\textrm{DPP}}{\query} := \sum_{x\in \mathcal S} \frac{\query(x)}{\pi_x}
\end{equation*}
Its variance can be computed using $\epsilon_x$ as the counting variable for $x$
\begin{align*}
	\Var{\textrm{DPP}}{\query}
:=\sum_{x,y \in \mathcal{X}}\EE{}{\epsilon_{x} \epsilon_{y}} \frac{\query(x) \query(y)} {\pi_{x} \pi_{y}}  - \loss{\query}^{2}\\
\quad \text{with} \quad
\EE{}{\epsilon_{x} \epsilon_{y}}=
\begin{cases}
	\det K_{\{x, y\}}=\pi_{x} \pi_{y}-K_{xy}^{2}, & \text{if } x \neq y \\
	\EE{}{\epsilon_{x}}=\pi_{x},&\text{if } x = y
\end{cases}
\end{align*}



Introducing $\Pi := \operatorname{diag}(\pi)$ and $\tilde K := \Pi^{-1}K^{\odot 2} \Pi^{-1}$, we can rewrite  

\begin{equation}
	\Var{\textrm{DPP}}{\query}=\sum_{x \in \mathcal{X}}\left(\frac{1}{\pi_{x}}-1\right) \query(x)^{2}-\sum_{x \neq y} \frac{K_{xy}^{2}}{\pi_{x} \pi_{y}} \query(x) \query(y) =  \boldsymbol\query\T (\Pi^{-1}  - \tilde{K}) \boldsymbol\query 
\end{equation}

\paragraph{In the Bernoulli process case}, where for all $x \in \mathcal{X}$, $\PP{}{x \in \mathcal S} = \pi_x$ independently, we have a special case of DPP, where the kernel reduces to its diagonal, i.e. $K = \Pi$ and then $\tilde K = I$. We denote its variance $\Var{\textrm{diag}}{f} := \boldsymbol\query\T (\Pi^{-1}  - I) \boldsymbol\query $.


\paragraph{In the m-DPP case}, we have $\mathcal S \sim \mathcal{DPP}(K) \mid |S|=m$, and we denote its marginals $b_{x} := \mathbb{E}\left[\epsilon_{i}\right]$, that admit an analytic form one can find in \cite{kulesza2012_dpp_for_ml}. Then an unbiased estimator of $L$ is
\begin{equation*}
	\estloss{\textrm{mDPP}}{\query} := \sum_{x\in \mathcal S} \frac{\query(x)}{b_x}
\end{equation*}

and its variance is
\begin{equation}
	\Var{\textrm{mDPP}}{\query}:=\sum_{i}\left(\frac{1}{b_x}-1\right) \query(x)^2
	+ \sum_{x \neq y} C_{xy}\query(x) \query(y)
\end{equation}
where $C_{xy}:=\frac{\mathbb{E}\left[\left(\epsilon_{x}-b_{y}\right)\left(\epsilon_{y}-b_{y}\right)\right]}{\mathbb{E}\left[\epsilon_{i}\right] \mathbb{E}\left[\epsilon_{j}\right]}=\frac{\mathbb{E}\left[\epsilon_{x} \epsilon_{y}\right]}{b_{x} b_{y}}-1
$

Observe that if the m-DPP kernel is reduced to its diagonal ($C_{xy} = 0$), we recover $\Var{\textrm{diag}}{}$, the variance of a Bernoulli process with same marginals ($\pi_x = b_x$), though the former has fixed sample size $m$, and the latter not.

In order to benefit from some variance reduction, one should find a $m$-DPP where $\forall x\neq y \,,\, C_{xy}\query(x) \query(y) <0$.

\cite{zhang2017dppminibatch} discuss that intuitively, if the $m$-DPP kernel rely on some similarity measure and that $f$ is smooth for it, then 2 similar points should have both negative correlation ($C_{xy}<0$) and their value have positive scalar product ($\query(x) \query(y) > 0$). This provides variance reduction.

Reversely, they argued that 2 dissimilar points should have positive correlation, and their value show ``no tendency to align'' hinting $\query(x) \query(y) < 0$, and again providing variance reduction. However, properties of strong Rayleigh measures implies always $C_{xy}\leq0$ (see \cite{pemantle2011rayleighconcentration}). But we could more conservatively consider that, whether DPP or $m$-DPP, two dissimilar points tend toward independence. Thus the induced variance change, whether positive or negative depending on the sign of $\query(x) \query(y)$, would in either case be small. 



\subsection{Variance comparison}
In the following, we compare processes with the same marginals, and therefore set $\Pi = mQ$. Also, 
since $m$-DPP marginals admits analytic but complicated form, we drop the $m$-DPP case comparison. We show in \cref{subsec__foursampl} that $\Var{\textrm{iid}}{}$, $\Var{\textrm{diag}}{}$ and $\Var{\textrm{DPP}}{}$ are quadratic forms of $\boldsymbol\query$ associated with respective matrices
$$\begin{cases}
	\Var{\textrm{iid}}{} \equiv \Pi^{-1} - \frac{\moones}{m} \\
	\Var{\textrm{diag}}{} \equiv \Pi^{-1} - I \\
	\Var{\textrm{DPP}}{} \equiv \Pi^{-1} - \tilde K
\end{cases}$$

This allows to compare samplings through the Loewner ordering ($\preceq$) of the variance associated matrices. For instance, we say DPP variance strictly dominates Bernoulli process variance if it uniformly yields lower variance, which is equivalent to $\tilde K$ being strictly greater than identity. Formally 
\begin{equation*}
	\forall \query \in \RR^{\mathcal{X}}, \, \Var{\textrm{DPP}}{f} < \Var{\textrm{diag}}{f} \iff \tilde K \succ I.
\end{equation*}
Massaging some linear algebra thus gives

\begin{tcolorbox}
	\begin{proposition}[Variance comparison]\ \\
		\label{prop__var_comp}
		DPP variance dominates Bernoulli process variance on positive-valued functions
		\begin{align}
			\label{eqn__posvar}
			\forall \query \in \RR_+^{\mathcal{X}},\ \Var{\textrm{DPP}}{f} \leq \Var{\textrm{diag}}{f}.
		\end{align}
		In the general case of real-valued functions, DPP variance does not dominate Bernoulli process variance but does up to a factor three
		\begin{align}
			\label{eqn__nodom}
			&\exists \query \in \RR^{\mathcal{X}},\ \Var{\textrm{DPP}}{f} \geq \Var{\textrm{diag}}{f}\\
			\label{eqn__domtwo}
			&\forall \query \in \RR^{\mathcal{X}},\ \Var{\textrm{DPP}}{f} \leq 3\Var{\textrm{diag}}{f}.
		\end{align}
		Moreover, if the DPP is projective, then
		\begin{align}
			\label{eqn__projiid}
			\forall \query \in \RR^{\mathcal{X}},\ \Var{\textrm{DPP}}{f} \leq \Var{\textrm{i.i.d.}}{f}.
		\end{align}
	\end{proposition}
\end{tcolorbox}



\begin{proof}[Proof of:]\
	\begin{enumerate}
		\item[\cref{eqn__posvar}] Assume $\query \in \RR_+^{\mathcal{X}}$. Then $\boldsymbol\query\T (\tilde K - I)\boldsymbol\query = \sum_{x \neq y} \frac{K_{xy}^{2}}{\pi_{x} \pi_{y}} \query(x) \query(y) \geq 0$ and therefore $\Var{\textrm{DPP}}{f} \leq \Var{\textrm{diag}}{f}$.
		\item[\cref{eqn__nodom}] $\tilde K = \Pi^{-1}K^{\odot 2} \Pi^{-1}$ is a symmetric positive matrix and by Hadamard inequality $\det( \tilde K) \leq \prod_{x\in \mathcal{X}} \tilde K_{xx}= 1$. Therefore at least one of its eigenvalue is lower than 1, hence $\tilde K \nsucc I \iff \exists \query \in \RR^{\mathcal{X}},\ \Var{\textrm{DPP}}{f} \geq \Var{\textrm{diag}}{f}$.
		\item[\cref{eqn__domtwo}]
			For all $f\in \RR^{\mathcal{X}}$, let denote by $f = f_+ - f_-$ its decomposition into its positive and negative part, which both belong in $\RR_+^{\mathcal{X}}$. Then we have

		\begin{align*}
			\hspace{-1cm}\Var{\textrm{DPP}}{f} &= \Var{\textrm{DPP}}{f_+} + \Var{\textrm{DPP}}{f_-} - 2\Cov{\textrm{DPP}}{f_+,f_-} \ &\text{\footnotesize Al-Kashi}\\
			&\leq \Var{\textrm{DPP}}{f_+} + \Var{\textrm{DPP}}{f_-} + 2\sqrt[]{\Var{\textrm{DPP}}{f_+}\Var{\textrm{DPP}}{f_-}} \ &\text{\footnotesize Cauchy-Schwartz}\\
			&\leq \Var{\textrm{diag}}{f_+} + \Var{\textrm{diag}}{f_-} + 2\sqrt[]{\Var{\textrm{diag}}{f_+}\Var{\textrm{diag}}{f_-}} \ &\text{\footnotesize \cref{eqn__posvar}}\\
			&\leq 3\Var{\textrm{diag}}{f}
		\end{align*}

		where we lastly use that $\Var{\textrm{diag}}{f} = \Var{\textrm{diag}}{f_+} + \Var{\textrm{diag}}{f_-}$ since its associated matrix is diagonal.
		\item[\cref{eqn__projiid}]
		$K$ being symmetric positive of rank $r \in \intint{0}{n}$, it exists $V \in \RR^{r \times n}$ such that $K = V\T V$, and we denote by $V_i$ its colons, for $i \in \intint{1}{n}$.
	
		For any vector $v \in \RR^{r}$, \cite{copenhaver2013diagramvectors} define its diagram vector 
		$$\tilde v :=
			\frac{1}{\sqrt{r-1}} (v_k^{2}-v_l^{2} , \sqrt{2 r} v_k v_l )_{k<l}\T \in \RR^{r(r-1)}$$
		concatenating all the $\frac{r(r-1)}{2}$ differences of squares and $\frac{r(r-1)}{2}$ products.
		
		Then introduce $\tilde V = (\tilde V_i )_{i\in\intint{1}{n}}$, the matrix whose columns are diagram vectors of matrix $V$ columns. It allows us to rewrite $\tilde K = \frac{\moones}{r} + \frac{r-1}{r} \tilde V\T \tilde V$ thus $\tilde K - \frac{\moones}{m} = (\frac{1}{r}-\frac{1}{m})\moones + \frac{m-1}{m} \tilde V\T \tilde V$. Then in order to have 
		\begin{equation*}
			\tilde K - \frac{\moones}{m}\succeq 0 \iff \forall \query \in \RR^{\mathcal{X}},\ \Var{\textrm{DPP}}{f} \leq \Var{\textrm{i.i.d.}}{f}
		\end{equation*}
		it is sufficient to have DPP kernel $K$ such that $r \leq m$. On the other hand, we know its average number of samples is $\operatorname{Tr}K = \operatorname{Tr}\Pi^{-1} = m$, because we fixed its marginals. Moreover $\operatorname{Tr}K \leq r$ holds for every DPP, this implies $\operatorname{Tr}K=r$, and therefore it is a projective DPP. Put differently, for any multinomial sampling, we have a projective DPP that beats it uniformly.
	\end{enumerate}
	
\end{proof}

Note that \cref{eqn__domtwo} use the general inequality
\begin{equation*}
	\Var{}{f} \leq \Var{}{f_+} + \Var{}{f_-} + 2\sqrt[]{\Var{}{f_+}\Var{}{f_-}}
\end{equation*}
which justifies that in many cases, we can restrict ourselves to controlling variances of positive-valued functions without loss of generality.

In the case of positive valued functions, \cref{prop__var_comp} shows that for any Bernoulli process or multinomial sampling, taking any projective DPP sampling with same marginals would yield lower variance. This is a strong qualitative argument for the use of projective DPPs for the coreset problem, that we will now try to quantify.




\chapter{Improving concentration with DPP}
\label{chap__improv_conc_dpp}


\section{Quantitative results on variance}
\label{sec__quant_result_variance}
Based on the previous \cref{chap_correlated_sampling}, we are interested in quantifying the variance DPPs can yield, in order to leverage it into an improved sample complexity for coresets. We recall classical results on variance of Monte Carlo integration.

\subsection{Monte Carlo integration}
Denote $\mathcal{I}:= [-1,1]$ so that $\mathcal{I}^d$ is an hypercube of dimension $d$. Given some integrand $h \colon \mathcal{I}^d \to \RR$, we are interested in computing $\int_{\mathcal{I}^d} h(x) \dd \lambda(x)$, its integral on $\mathcal{I}^d$. The idea of Monte Carlo integration is to sample elements from $\mathcal{I}^d$ and to build an estimator $\estloss{}{\threequery}$ based on these elements.

We know from central limit theorem that as long as variance under sampling distribution exists, i.i.d. sampling Monte Carlo yields
\begin{equation}
	\Var{}{\estloss{}{\threequery}} \lesssim m^{-1}
\end{equation}
Moreover, we know from \cite{bakhvalov1959_approximate_calculation_integrals} that when the integrand $\threequery$ belongs to the Sobolev space $W^{s,2}(\mathcal{I})$, i.e. when its Fourier coefficients decay sufficiently rapidly,
\begin{equation}
	\label{eqn__lower_bound_variance_integration}
	\Var{}{\estloss{}{\threequery}} \gtrsim m^{-(1+\frac{2s}{d})}
\end{equation}
We refer to \cite{novak2014_complexity_numerical_integration} for a more recent inventory on complexity of numerical integration. 

Of course, there is plenty of room between this two variance rates. One would be tempted to get out of the i.i.d. framework, without assuming too much regularity of the integrand, and find an estimator that is as close as possible to the lower bound \cref{eqn__lower_bound_variance_integration}. We present next a recent result that falls into this description.



\subsection{Improved variance rate with DPPs}

Assume data $\mathcal{X}$ is strictly included in the hypercube $\mathcal{X} \subset \mathcal{I}^d$. For some integrand $h$, we are interested in constructing an estimator of the mean value of $h$ on $\mathcal{X}$.

\cite{bardenet2021sgddpp} show the existence of a sequence of DPP kernels $(\opekernel{m})_{m\in\NN}$, whose induced estimator has asymptotic variance $\OO( m ^{-(1+\frac 1 d)})$. We describe here the main key steps of the kernel construction, and introduce needed notations.

\paragraph{Kernel construction.}
Assume data $\mathcal{X}$ is generated by random samplings from a distribution $\gamma$ supported in the interior of the hypercube $\mathcal{I}^d$. Then, one can define its kernel density estimation (KDE)
\begin{equation*}
	\tilde \gamma(y) := \frac{1}{n\Delta^d} \sum_{x\in \mathcal{X}} k\left(\frac{x-y}{\Delta}\right)
\end{equation*}
where $\Delta >0$, and $k$ is a kernel unrelated to any DPP kernel, chosen so that $\int k \dd \lambda = 1$. 

Let now $\omega$ be a strictly positive probability density function on $\mathcal{I}$, and  denote by $q = q$ its tensor product density, supported on $\mathcal{I}^d$. Applying Gram-Schmidt algorithm in $L^2(q \dd \lambda)$ on multivariate monomials returns a sequence of orthonormal polynomial functions $(\phi_k)_{k\in \NN}$, the multivariate orthonormal polynomials with respect to density $q$. We refer to \cite{gautschi2004ope} for a review of orthogonal polynomials.

From there, define the multivariate Orthogonal Polynomial Ensemble (OPE) kernel associated to $q$
\begin{equation}
	K_q^{(m)}(x,y) := \sum_{k=1}^m \phi_k(x) \phi_k(y) \colon \mathcal{I}^d \times \mathcal{I}^d \to \RR
\end{equation}
which is the outer product of the $m$ first multivariate orthonormal polynomials. Because of orthonormality property, OPE kernel are projective kernels, and so it induces a projective DPP.

Noticeably, we obtained an OPE kernel associated to $q$, that we can correct to obtain an OPE kernel associated to the KDE distribution $\tilde \gamma$, by defining 
\begin{equation}
	K_{q, \tilde{\gamma}}^{(m)}(x, y):=\sqrt{\frac{q(x)}{\tilde{\gamma}(x)}} K_q^{(m)}(x, y) \sqrt{\frac{q(y)}{\tilde{\gamma}(y)}}.
\end{equation}
However, this projective DPP kernel is still defined on $\mathcal{I}^d \times \mathcal{I}^d$ but we are interested in sampling elements from $\mathcal{X}$, not $\mathcal{I}^d$. A last operation we not detail allows to restrict this kernel into $\opekernel{m}$, a projective DPP kernel defined on $\mathcal{X} \times \mathcal{X}$, or equivalently, a matrix of size $n \times n$ indexed by the elements of $\mathcal{X}$.

\paragraph{Improved variance rate.}
It turns out an estimator based on $\opekernel{m}$ yields improved variance. In particular, Proposition 4. and Equation (S14) from \cite{bardenet2021sgddpp} state that

\begin{tcolorbox}
	\begin{theorem}[\cite{bardenet2021sgddpp}]
		\label{thm_sgdpaper}
		Let $m\in \NN$ and $\mathcal{S} \sim  \mathcal{DPP}(\opekernel{m})$. Define for all $h \in \RR^{\mathcal{X}}$ the correlated importance sampling estimator
		\begin{equation}
			\estloss{\mathcal{S}}{\threequery} := \sum_{x \in \mathcal{S}} \frac{\threequery(x)}{\opekernel{m}(x,x)}.
		\end{equation}
		If  $\lipschitz_\threequery := \operatorname{Lip}\left\{\frac{m\threequery}{K^{(m)}_{q, \tilde \gamma}} \right\}$ i.e. the Lipschitz constant of $x \mapsto \frac{m\threequery(x)}{K^{(m)}_{q, \tilde \gamma}(x,x)}$ is defined, then 
		\begin{equation}
			\label{eqn__improved_var}
			\Var{}{\estloss{\mathcal{S}}{\threequery}} = \lipschitz^2_\threequery \OO( m ^{-(1+\frac 1 d)}) +\OO( n^{-1/2}).
		\end{equation}
	\end{theorem}
\end{tcolorbox}

It is worth to say this variance generalize to the continuous case. Actually, pre-existing results from \cite{bardenet2020mcdpp} treated this case, with same variance improvement. Since the same continuous arguments are invoked in both continuous and discrete case, it explains why the construction of $\opekernel{m}$ relies first on constructing continuous DPP kernels, though it is not known if that detour can be shortcut in the discrete case.

Also, note that the higher $d$ is, the less the variance gain in \cref{eqn__improved_var}. Intuitively, this is because as the dimension increase, all points become already repelled from each others, so the addition of repulsiveness is less and less effective, or put differently, there is less and less redundancy in an independent sampling that we can get rid of by a negatively correlated sampling.

\note{}{cf. independent sphere packing in high dimension?}

We now introduce some notations
\begin{itemize}
	\item We denote by $\empdistr{\mathcal{S}}{}:=\frac{1}{|\mathcal{S}|} \sum_{x \in \mathcal{S}} \delta_x$ the empirical measure based on sample $\mathcal{S}  \subseteq \mathcal{X}$. 
	\item Hence for all function $\threequery$, we denote by $\empdistr{\mathcal{S}}{\threequery}:=\int_{\mathcal{X}} \threequery(x) d\empdistr{\mathcal{S}}{x} = \frac{1}{|\mathcal{S}|} \sum_{x \in \mathcal{S}} \threequery(x)$ the expectation of $\threequery$ with respect to $\empdistr{\mathcal{S}}{}$. Furthermore, given a distribution $\PP{}{}$ on $\mathcal{S}$, we denote by $\meanempdistr{\threequery}:=\EE{}{\empdistr{\mathcal{S}}{\threequery}}$, its expectation with respect to $\PP{}{}$. 
	\item Finally, the induced $L^1(\empdistr{\mathcal{S}}{})$ distance between two functions $\threequery$ and $\threequery'$ is denoted by $\dlone{\empdistr{\mathcal{S}}{}}{\threequery}{\threequery'}:=\empdistr{\mathcal{S}}{|\threequery - \threequery'|}$.
\end{itemize}  


so that we can reformulate the result from \cref{thm_sgdpaper} into the following corollary.
\begin{tcolorbox}
	\begin{corollary}
		\label{cor_sgdpaper}
		Let $m\in \NN$ and $\mathcal{S} \sim  \mathcal{DPP}(\opekernel{m})$.
		For all function $\twoquery \in \RR^{\mathcal{X}}$, we have
		\begin{equation*}
			\Var{}{\empdistr{\mathcal{S}}{\twoquery}} = \lipschitz^2_\twoquery \OO( m ^{-(1+\frac 1 d)}) +\OO( n^{-1/2})
		\end{equation*}
		where $\lipschitz_\twoquery := \operatorname{Lip}\left\{\frac{\opekernel{m}}{K^{(m)}_{q, \tilde \gamma}} \twoquery \right\}$ is the Lipschitz constant of $x \mapsto\frac{\opekernel{m}(x,x)}{K^{(m)}_{q, \tilde \gamma}(x,x)} \twoquery(x) $.
	\end{corollary}
\end{tcolorbox}

\begin{proof}
	We simply apply \cref{thm_sgdpaper} with $\threequery = \frac{\opekernel{m}g}{m}$, such that
	\begin{align*}
		\Var{}{\empdistr{\mathcal{S}}{\twoquery}} = \Var{}{\estloss{\mathcal{S}}{\threequery}} = \lipschitz^2_\threequery \OO( m ^{-(1+\frac 1 d)}) +\OO( n^{-1/2}),
	\end{align*}
	where $\lipschitz_\threequery$ is the Lipschitz constant of $x \mapsto \frac{m\threequery(x)}{K^{(m)}_{q, \tilde \gamma}(x,x)} = \frac{\opekernel{m}(x,x)}{K^{(m)}_{q, \tilde \gamma}(x,x)} \twoquery(x) $.

\end{proof}





\section{Regularity assumptions}
In order to translate the improved variance rate from \cref{cor_sgdpaper} into a concentration inequality and then a sample complexity bound for coreset, several assumptions are made and are discussed.

Let $\qset \in \RR^{\mathcal{X}}$ be the function space on which we want the coreset property to hold. With the notations of \cref{chap_intro_coresets}, define the following function space 
		\begin{equation*}
		\twoqset_m := \frac{m\qset}{\opekernel{m}\loss{\qset}} = \left\{x \mapsto \frac{m\query(x)}{\opekernel{m}(x,x)\loss{\query}} \mid \query \in \qset\right\} \in \RR^{\mathcal{X}}.
	\end{equation*}
Note then that for any $\query \in \qset$, we can define $g := \frac{m\query}{\opekernel{m}\loss{\query}} \in \twoqset_m$ which verifies
	\begin{equation}
		\label{def_ftog}
		\frac{\estloss{\mathcal{S}}{\query}}{\loss{\query}} = \empdistr{\mathcal{S}}{g}
		\quad \text{ and }\quad 
		\meanempdistr{g} = \EE{}{\empdistr{\mathcal{S}}{g}} = \frac{\EE{}{\estloss{\mathcal{S}}{\query}}}{\loss{\query}} = 1.
	\end{equation}




In the following sections, we crucially assume the functions sets $\twoqset_m$ verify boundedness in Lipschitz constant, infinite norm, and pseudo-dimension.
Formally, we assume there exists positive reals $\lipschitz, \bound \in \RR_+$ and $\pdim \in \NN$, such that for all $m \in \NN$
\begin{enumerate}
	\item \label{assum__lip} $\forall \twoquery \in \twoqset_m,\ \operatorname{Lip}\left\{\frac{\opekernel{m}}{K^{(m)}_{q, \tilde \gamma}}\twoquery \right\} \leq \lipschitz $
	\item \label{assum__inf}$\forall \twoquery \in \twoqset_m,\ \|\twoquery\|_{\infty} \leq \bound $
	\item \label{assum__pseudodim}$\operatorname{pdim}\twoqset_m \leq \pdim$
\end{enumerate}

\subsection{Discussing the assumptions}
We invoke asymptotic phenomena to justify our precedent assumptions. With notations from previous \cref{sec__quant_result_variance}, we first have that for large $n$, the KDE estimation is close to the true density, namely $\tilde \gamma \xrightarrow[n \to +\infty]{} \gamma$. Second, next theorem describes the asymptotic behaviour of any OPE kernel, and in particular confirms that $K_q^{(m)}$ is of order $m$.
\begin{tcolorbox}
	\begin{theorem}[\cite{simon2010_totik_thm}]
		Assume $q$ is continuous. Then, for every $\epsilon>0$, we have uniformly for $x\in [-1+\epsilon,1-\epsilon]^d$
		\begin{equation}
			\frac{m}{K_q^{(m)}(x, x)} \xrightarrow[m \to +\infty]{} \frac{q(x)}{q_{\mathrm{eq}}(x)}
		\end{equation}
	\end{theorem}
	where $q_{\mathrm{eq}} = x \mapsto \prod_{k=1}^d \frac{1}{\pi \sqrt[]{1-x_k^2}}$.
\end{tcolorbox}

Finally, the restriction of $K^{(m)}_{q, \tilde \gamma}$ into $\opekernel{m}$ is such that for large $n$, we have $K^{(m)}_{q, \tilde \gamma} \simeq \opekernel{m}$. Put together, the asymptotic behaviour of the set $\twoqset_m$ is 
\begin{equation}
	\twoqset_m \to \twoqset := \frac{\qset\gamma}{\loss{\qset}q_{\mathrm{eq}}} = \left\{x \mapsto \frac{\query(x)\gamma(x)}{\loss{\query}q_{\mathrm{eq}}(x)} \mid \query \in \qset\right\} \in \RR^{\mathcal{X}}.
\end{equation}


 Then, assumptions we made on $\twoqset_m$ for every $m\in \NN$ translates to assumptions on $\twoqset$. Indeed, if $\twoqset$ verifies boundedness in Lipschitz constant, infinite norm, and pseudo-dimension, we have
\begin{enumerate}
	\item $\forall \twoquery \in \twoqset_m,\ \operatorname{Lip}\left\{\frac{\opekernel{m}}{K^{(m)}_{q, \tilde \gamma}}\twoquery \right\} \lesssim \sup_{\query \in \qset}\operatorname{Lip}\left\{\frac{\query\gamma}{\loss{\query}q_{\mathrm{eq}}} \right\} =: \lipschitz$
	\item $\forall \twoquery \in \twoqset_m,\ \|\twoquery\|_{\infty} \lesssim \|\frac{\query\gamma}{\loss{\query}q_{\mathrm{eq}}}\|_{\infty} \leq \|\frac{\gamma}{q_{\mathrm{eq}}}\|_{\infty} =:  \bound$
	\item $\operatorname{pdim}\twoqset_m \lesssim \operatorname{pdim}\twoqset =: \pdim$
\end{enumerate}

Note that $\twoqset$ plays a similar role as $\twoqset_s$ we introduced in \cref{eqn__g_sensitivity}. It is a function space whose properties influence the sample complexity bound, as we will now show.






\section{Concentration for fixed query}
\subsection{Chebyshov bound}

For any $\epsilon>0$ and $m\in \NN$, define the Chebyshov concentration bound
\begin{equation}
	\label{def_boundratecheb}
	\boundrate_{\textrm{Cheb}}(\epsilon, m) := \frac{1}{\epsilon^2} \sup_{\twoqset \in \{\twoqset_m \mid m \in \NN\}} \sup_{\twoquery \in \twoqset} \Var{}{\empdistr{\mathcal{S}}{\twoquery}}.
\end{equation}
Remark that from \cref{cor_sgdpaper} and the assumption \ref{assum__lip} on Lipschitz constant that $\boundrate_{\textrm{Cheb}}(\epsilon, m) = \frac {1} {\epsilon^2}\left(\lipschitz^2 \OO( m ^{-(1+\frac 1 d)}) +\OO( n^{-1/2})\right)$. From that, it follows


\begin{tcolorbox}
	\begin{theorem}[Chebyshov bound for fixed query]
		\label{thm_chebfixedtheta} 
		Let $m\in \NN$ and $\mathcal{S} \sim  \mathcal{DPP}(\opekernel{m})$. 

		Then for all $\epsilon >0$ and all $\query \in \qset$,
		\begin{equation*}
			\PP{}{\dnude{\nu}{\estloss{\mathcal{S}}{\query}}{\loss{\query}}>\epsilon\loss{\query}} \leq \boundrate_{\textrm{Cheb}}(\epsilon, m)
		\end{equation*}
		
		
		Moreover, for all $\delta>0$ and for $n$ sufficiently large,
		\begin{equation*}
			m \gtrsim \left(\frac{\lipschitz^2}{\delta\epsilon^2} \right)^{\frac{1}{1+\frac 1 d}}
			\implies 
			\text{$\mathcal{S}$ is an $\epsilon$-coreset for $\query$ w.p. $1-\delta$}.
		\end{equation*}
	\end{theorem}
\end{tcolorbox}





\begin{proof}
	Let $\query \in \qset$, so that $g := \frac{m\query}{\opekernel{m}\loss{\query}} \in \twoqset_m$ verifies \cref{def_ftog}.

	Applying the Bienaym\'e-Chebyshov inequality, we obtain 
	\begin{align*}
		\PP{}{\dnude{\nu}{\estloss{\mathcal{S}}{\query}}{\loss{\query}}>\epsilon\loss{\query}}
		&= \PP{}{\dnude{\nu}{\frac{\estloss{\mathcal{S}}{\query}}{\loss{\query}}}{1}>\epsilon}\\
		&= \PP{}{\dnude{\nu}{\empdistr{\mathcal{S}}{\twoquery}}{\meanempdistr{\twoquery}}>\epsilon}\\ 
		&\leq \frac{1}{\varepsilon ^{2}}\Var{}{\empdistr{\mathcal{S}}{\twoquery}}\\
		&\leq \boundrate_{\textrm{Cheb}}(\epsilon, m)
		=\frac {1} {\epsilon^2}\left(\lipschitz^2 \OO( m ^{-(1+\frac 1 d)}) +\OO( n^{-1/2})\right)
	\end{align*}
	Hence, a sufficient condition for $\mathcal{S}$ to be an $\epsilon$-coreset for $\query$ w.p. $1-\delta$ is to have
	\begin{gather*}
		\boundrate_{\textrm{Cheb}}(\epsilon, m) \leq \delta 
		\iff
		m^{1+\frac 1 d} \gtrsim \frac{\lipschitz^2}{\delta \epsilon^2 + \OO(n^{-1/2})} = \frac {\lipschitz^2} {\delta\epsilon^2} \frac{1}{1 + \frac{1}{\delta \epsilon^2}\OO(n^{-1/2})}.
	\end{gather*} 
	For sufficiently large $n$ (potentially $n\gtrsim \delta^{-2} \epsilon^{-4}$), we can control the second factor and thus obtain the bound
	\begin{equation*}
		m \gtrsim \left(\frac{\lipschitz^2}{\delta\epsilon^2} \right)^{\frac{1}{1+\frac 1 d}}.
	\end{equation*}
\end{proof}

We obtained a sample complexity bound with improved dependency in $\epsilon$ compared to the one from \cref{thm_hoeffdingfixedquery} for the i.i.d. sampling framework. Our result is $\OO(\epsilon^{\frac{-2}{1+\frac{1}{d}}})$ whereas the latter is $\OO(\epsilon^{-2})$. As for the variance, the higher the dimension $d$ is, the less is the gain in sample complexity, because the less there is redundancy to get rid of by a negatively correlated sampling.

On the other hand, the dependency in $\delta$ is worsened compared to i.i.d. sampling. Our result is $\OO(\delta^{\frac{-2}{1+\frac{1}{d}}})$ whereas the latter is $\log \delta^{-1}$.
We propose to tackle this dependency, before generalizing the obtained bound to all queries.









\subsection{Breuer and Duits bound}
 
There actually exists a concentration bound for projective DPPs from \cite{breuer2013nevai}, that still can leverage the increased variance rate from \cite{bardenet2021sgddpp}, and that is tighter than Chebyshov bound.

\begin{tcolorbox}
	\begin{theorem}[\cite{breuer2013nevai}]
		\label{thm_breuer}
		Let $\epsilon>0$, any bounded function $\threequery$, any projective DPP kernel $K$, and let $\mathcal{S} \sim  \mathcal{DPP}(K)$.\\

		Then for any linear statistic $X_\threequery := \sum_{x\in\mathcal{S}}\threequery(x)$,
		\begin{equation*}
			\PP{}{\dnude{}{X_\threequery}{\EE{}{X_\threequery}} > \epsilon} \leq	
			\begin{cases}
				2 \exp \left(-\frac{\epsilon^2}{4 A \Var{}{X_\threequery}}\right) 
				& \text { if } \epsilon<\frac{2 A \Var{}{X_\threequery}}{3\|\threequery\|_{\infty}}\\
				2 \exp \left(-\frac{\epsilon}{6\|\threequery\|_{\infty}}\right) 
				& \text { otherwise }
			\end{cases}
		\end{equation*}	
		where $A \simeq 7819$ so does not depend on $\threequery$, $K$ or $\epsilon$.
	\end{theorem}
\end{tcolorbox}
Note that the second case in this concentration bound is tighter than the first. Indeed, 
\begin{equation*}
	\epsilon\geq\frac{2 A \Var{}{X_\threequery}}{3\|\threequery\|_{\infty}} \implies \frac{4 A \Var{}{X_\threequery}}{\epsilon^2} \leq \frac{3\|\threequery\|_{\infty}}{2 A \Var{}{X_\threequery}}\frac{4 A \Var{}{X_\threequery}}{\epsilon} = \frac{6\|\threequery\|_{\infty}}{\epsilon}
\end{equation*}
and thus we always have
\begin{equation}
	\label{eqn__remark_tighter}
	\PP{}{\dnude{}{X_\threequery}{\EE{}{X_\threequery}} > \epsilon} \leq	2 \exp \left(-\frac{\epsilon^2}{4 A \Var{}{X_\threequery}}\right).
\end{equation}	

Based on that remark, we define for any $\epsilon>0$ and $m\in \NN$
\begin{equation*}
	\boundrate_{\textrm{BD}}(\epsilon,m) := 2 \exp \left( \left(-
			4 A \boundrate_{\textrm{Cheb}}(\epsilon, m)
			\right)^{-1}
			\right)
\end{equation*}
which we call a Breuer and Duits (BD)-type bound, and that verifies what follows.





\begin{tcolorbox}
	\begin{theorem}[BD-type bound for fixed query]
		\label{thm_breuerfixedtheta}
		Let $m\in \NN$ and $\mathcal{S} \sim  \mathcal{DPP}(\opekernel{m})$. Then for all $\epsilon >0$ and all $\query \in \qset$,
		\begin{equation*}
			\PP{}{\dnude{\nu}{\estloss{\mathcal{S}}{\query}}{\loss{\query}}>\epsilon\loss{\query}} \leq \boundrate_{\textrm{BD}}(\epsilon, m).
		\end{equation*}
		
		Moreover, for all $\delta>0$ and for $n$ sufficiently large,
		\begin{equation*}
			m \gtrsim \left(\frac{\lipschitz^2}{\epsilon^2}\log  \frac{2}{\delta } \right)^{\frac{1}{1+\frac 1 d}}
			\implies 
			\text{$\mathcal{S}$ is an $\epsilon$-coreset for $\query$ w.p. $1-\delta$}.
		\end{equation*}
	\end{theorem}
\end{tcolorbox}
\begin{proof}
	Let $\query \in \qset$, so that $g := \frac{m\query}{\opekernel{m}\loss{\query}} \in \twoqset_m$ verifies \cref{def_ftog}.

	Then we apply the Breuer and Duits bound from \cref{thm_breuer} taking $\threequery = \frac{\twoquery}{m}$.
	\begin{align*}
		\PP{}{\dnude{\nu}{\estloss{\mathcal{S}}{\query}}{\loss{\query}}>\epsilon\loss{\query}}
		&= \PP{}{\dnude{\nu}{\empdistr{\mathcal{S}}{\twoquery}}{\meanempdistr{\twoquery}}>\epsilon}\\ 
		&\leq 2 \exp \left(
		\begin{cases}
			\frac{-\epsilon^2}{4 A \Var{}{\empdistr{\mathcal{S}}{\twoquery}}}
			& \text { if } \epsilon< \frac{2 A m \Var{}{\empdistr{\mathcal{S}}{\twoquery}}}{3\|\twoquery \|_{\infty}}\\
			\frac{-\epsilon m}{6\|\twoquery\|_{\infty}} 
			& \text { otherwise }
		\end{cases}
		\right)\\
		&\leq 	2 \exp \left(-\frac{\epsilon^2}{4 A \Var{}{\empdistr{\mathcal{S}}{\twoquery}}}\right)\\
		&\leq 2 \exp \left( 
			\left(-4 A \boundrate_{\textrm{Cheb}}(\epsilon, m)\right)^{-1}
		\right)
		= \boundrate_{\textrm{BD}}(\epsilon,m)
	\end{align*}
	where we firstly used the remark in \cref{eqn__remark_tighter}, and secondly the definition \cref{def_boundratecheb} of $\boundrate_{\textrm{Cheb}}$.
	
	
	Hence, a sufficient condition for $\mathcal{S}$ to be an $\epsilon$-coreset for $\query$ w.p. $1-\delta$ is to have
	\begin{gather*}
		\boundrate_{\textrm{BD}}(\epsilon,m) \leq \delta
		\iff 
		\left(\log \frac{2}{\delta }\right)^{-1} \geq
		4 A \boundrate_{\textrm{Cheb}}(\epsilon, m)
	\end{gather*}
	and we know from \cref{thm_chebfixedtheta} this implies that for $n$ sufficiently large 
	\begin{equation*}
		m \gtrsim \left(\frac{\lipschitz^2}{\epsilon^2} \log  \frac{2}{\delta } \right)^{\frac{1}{1+\frac 1 d}}.
	\end{equation*}
\end{proof}

\paragraph{Discussing the tighter case.} Seeing \cref{thm_breuer}, one could wonder why we didn't leverage the second tighter case, noticed in \cref{eqn__remark_tighter}. Applied to our context, it would lead to the bound 
\begin{equation*}
	m \geq \frac{6\bound}{\epsilon} \log\frac{2}{\delta}
\end{equation*}
which is better than the one we shown. But for the second case to apply and obtain this bound, one would require

\begin{align*}
	\epsilon 
	\geq  \frac{2 A m \Var{}{\empdistr{\mathcal{S}}{\twoquery}}}{3\|\twoquery \|_{\infty}}
	= \OO\left(\frac{\lipschitz^2}{\|\twoquery \|_{\infty} m^{1/d}}\right)\\
	\iff
	m \gtrsim \left(\frac{\lipschitz^2}{\epsilon\|\twoquery \|_{\infty}}\right)^d
	\implies
	m \gtrsim \left(\frac{\lipschitz^2}{\epsilon\bound}\right)^d.
\end{align*}
Thereby, the condition on $\epsilon$ translates to a much worse bound on $m$, ruining the interest of the second case.
		 


\paragraph{Compared to the i.i.d. framework bound} from \cref{thm_hoeffdingfixedquery}, we obtained a better bound on coreset size for fixed query. This corroborate the qualitative results we obtained on variance comparison in \cref{sec__variance_arguments}, by quantifying the effect of increased variance rate on sampling complexity. 

Although the concentration bound from \cite{breuer2013nevai} pre-dates the works of \cite{tremblay2018dppcoreset}, it is only because of the improved variance rate from \cite{bardenet2021sgddpp} that it can yields improved concentration result. If one were to apply classical variance rate in $\OO(m^{-1})$ to it, it would find no improvements of the i.i.d. framework.



\section{Extension to all queries}
\label{sec_extension_all_queries}
In order to obtain an $\epsilon$-coreset for $\qset$, property \cref{eqn_querycoresetprop} must hold simultaneously for all queries $\query \in \qset$.









\begin{tcolorbox}
	\begin{theorem}[Infinite union bound]
		\label{thm_infinite_union_bound}
		Let $\threeqset \subseteq [0,\bound]^{\mathcal{X}}$ be a set of bounded functions defined on a base set $\mathcal{X}$, with $d'_{\threeqset} := \operatorname{pdim}\threeqset$ its pseudo-dimension (see \cref{def__pseudodim}). Let moreover $m\in \NN$ and $\PP{}{}$ a distribution supported on $\mathcal{X}^{m}$.\\

		Assume there exists a bounding function $\boundrate$ such that for all $\epsilon >0$, function $\threequery \in \threeqset$, integer $m \in \NN$, and let $\mathcal{S} \sim \PP{}{}$, we have the bound
		\begin{equation}
			\PP{}{\dnude{\nu}{\empdistr{\mathcal{S}}{\threequery}}{\meanempdistr{\threequery}} > \epsilon} \leq \boundrate(\epsilon, m)
		\end{equation}

		Then for all $\epsilon >0$ and all $m \in \NN$ such that $\boundrate(\epsilon/2, m) \leq 1/2$, it holds
		\begin{equation}
			\PP{}{\exists \threequery \in \threeqset,\ \dnude{\nu}{\empdistr{\mathcal{S}}{\threequery}}{\meanempdistr{\threequery}} > \epsilon} \leq 
			8\left(\frac{8e\bound}{\epsilon}\right)^{2d'_{\threeqset}} \boundrate(\epsilon/16, m)
		\end{equation}
	\end{theorem}
\end{tcolorbox}


We delay the proof of \cref{thm_infinite_union_bound} in the next \cref{sec__proof_thm_infinite_union_bound}

\begin{tcolorbox}
	\begin{theorem}[BD-type bound for all queries]
		\label{thm_breuerallqueries}
		Let $m\in \NN$ and $\mathcal{S} \sim  \mathcal{DPP}(\opekernel{m})$. 

		Then for all $\epsilon \in ]0,\bound]$ and all $\query \in \qset$
		\begin{equation*}
			\PP{}{\exists \query \in \qset,\ \dnude{\nu}{\estloss{\mathcal{S}}{\query}}{\loss{\query}}>\epsilon\loss{\query}} 
			\leq 
			8\left(\frac{8e\bound}{\epsilon}\right)^{2\pdim}   \boundrate_{\textrm{BD}}(\epsilon/16, m)
		\end{equation*}
		
		Moreover, for all $\delta >0$ and for $n$ sufficiently large
		\begin{equation*}
			m \gtrsim \left(\frac{\lipschitz^2}{\epsilon^2} \left( \pdim \log \frac{\bound}{\epsilon} + \log \frac{1}{\delta }  \right) \right)^{\frac{1}{1+\frac 1 d}} 
			\implies 
			\text{$\mathcal{S}$ is an $\epsilon$-coreset for $\qset$ w.p. $1-\delta$}
		\end{equation*}
	\end{theorem}
\end{tcolorbox}


\begin{proof}
	Let $\query \in \qset$, so that $g := \frac{m\query}{\opekernel{m}\loss{\query}} \in \twoqset_m$ verifies \cref{def_ftog}.

	We know from \cref{thm_breuerfixedtheta} that for all $\epsilon \in ]0,\bound]$ and $m \in \NN$ we have
	\begin{equation*}
		\PP{}{\dnude{\nu}{\empdistr{\mathcal{S}}{\twoquery}}{\meanempdistr{\twoquery}} > \epsilon}  \leq \boundrate_{\textrm{BD}}(\epsilon, m)
	\end{equation*}
	
	Moreover, the assumption \ref{assum__inf} implies $\twoqset_m \in [0,M]^{\mathcal{X}}$. Hypotheses of \cref{thm_infinite_union_bound} are thus satisfied, and we can apply it taking $\threeqset=\twoqset_m$, which yields that for all $m \in \NN$ such that $\boundrate_{\textrm{BD}}(\epsilon/2, m) \leq 1/2$, it holds
	\begin{align*}
		\PP{}{\exists \query \in \qset,\ \dnude{\nu}{\estloss{\mathcal{S}}{\query}}{\loss{\query}}>\epsilon\loss{\query}} 
		= \PP{}{\exists \twoquery \in \twoqset_m,\ \dnude{\nu}{\empdistr{\mathcal{S}}{\twoquery}}{\meanempdistr{\twoquery}} >  \epsilon}\\
		\leq  8\left(\frac{8e\bound}{\epsilon}\right)^{2d'_{\twoqset_m}}  \boundrate_{\textrm{BD}}(\epsilon/16, m)
		\leq  8\left(\frac{8e\bound}{\epsilon}\right)^{2\pdim} \boundrate_{\textrm{BD}}(\epsilon/16, m)
	\end{align*}
	where we used the assumption \ref{assum__pseudodim} that for all $m \in \NN$, $\operatorname{pdim}\twoqset_m \leq \pdim$.
	
	Hence, a sufficient condition for $\mathcal{S}$ to be an $\epsilon$-coreset for $\qset$ w.p. $1-\delta$ is to have
	\begin{gather*}
		\boundrate_{\textrm{BD}}(\epsilon/16, m) \leq \frac{\delta}{8} \left(\frac{8e\bound}{\epsilon}\right)^{-2\pdim}
		\iff
		m \gtrsim \left(\frac{256\lipschitz^2}{\epsilon^2} \log \left( \frac{16}{\delta }\left(\frac{8e\bound}{\epsilon}\right)^{2\pdim}\right) \right)^{\frac{1}{1+\frac 1 d}}\\
		\iff m \gtrsim \left(\frac{\lipschitz^2}{\epsilon^2} \left(\log \frac{1}{\delta } + \pdim \log \frac{\bound}{\epsilon} \right)\right)^{\frac{1}{1+\frac 1 d}}
	\end{gather*}
	This rate is conditioned to the fact that $m$ is such that $\boundrate_{\textrm{BD}}(\epsilon/2, m) \leq 1/2$. But we know it holds as soon as $m \gtrsim \left(\frac{4\lipschitz^2}{\epsilon^2} \log 4 \right)^{\frac{1}{1+\frac 1 d}}$, which is trivially implied by the obtained bound.

\end{proof}





\section{Proof of \cref{thm_infinite_union_bound}}
\label{sec__proof_thm_infinite_union_bound}


We follow a similar proof scheme as in section 9.4 of \cite{haussler1992decisiontheoricgeneralizationofPACmodel}. We specifically revisit Lemma 12. and 13., getting rid of independency hypothesis, and making intermediary results more flexible to further improvements.


\begin{tcolorbox}
	\begin{lemma}[Symmetrisation]
		\label{lem_symm}
		Assume the hypothesis of \cref{thm_infinite_union_bound} about $\boundrate$.
		Let furthermore be $\epsilon>0$, $m\in \NN$, and $\mathcal{S}_1, \mathcal{S}_2 \overset{i.i.d.}{\sim} \PP{}{}$, two sequences of size $m$ independently sampled from the same distribution supported on $\mathcal{X}^m$.\\
		  
		Then for all $m \in \NN$ such that $\boundrate(\epsilon/2, m) \leq 1/2$
		\begin{equation*}
			\PP{}{ \exists \threequery \in \threeqset,\ \dnude{\nu}{\empdistr{\mathcal{S}_1}{\threequery}}{\meanempdistr{\threequery}} > \epsilon} 
			\leq 2\PP{}{\exists \threequery \in \threeqset,\ \dnude{\nu}{\empdistr{\mathcal{S}_1}{\threequery}}{\empdistr{\mathcal{S}_2}{\threequery}} > \epsilon/2 }
		\end{equation*}
	\end{lemma}
\end{tcolorbox}


\begin{proof}
	Let $\epsilon>0$ and take $m \in \NN$ such that $\boundrate(\epsilon/2, m) \leq 1/2$. Then let $\mathcal{S}_1$ be sampled such that $\exists \threequery \in \threeqset,\ \dnude{\nu}{\empdistr{\mathcal{S}_1}{\threequery}}{\meanempdistr{\threequery}} > \epsilon$. This obviously happens with probability $\PP{}{\exists \threequery \in \threeqset, \dnude{\nu}{\empdistr{\mathcal{S}_1}{\threequery}}{\meanempdistr{\threequery}} > \epsilon }$.

	For such an $f$, we then independently sample $\mathcal{S}_2$ such that $\dnude{\nu}{\empdistr{\mathcal{S}_2}{\threequery}}{\meanempdistr{\threequery}} \leq \epsilon/2$. Because $\boundrate(\epsilon, m) \leq 1/2$, we know this happens with probability greater than $1-1/2 = 1/2$, and we thus have
	\begin{align*}
		&\PP{}{ \exists \threequery \in \threeqset,\ \dnude{\nu}{\empdistr{\mathcal{S}_1}{\threequery}}{\meanempdistr{\threequery}} > \epsilon}
		\frac 1 2 \\
		\leq\ &\PP{}{\exists \threequery \in \threeqset,\ \dnude{\nu}{\empdistr{\mathcal{S}_1}{\threequery}}{\meanempdistr{\threequery}} > \epsilon \wedge \dnude{\nu}{\empdistr{\mathcal{S}_2}{\threequery}}{\meanempdistr{\threequery}} \leq \epsilon/2 } \\
		\leq\ &\PP{}{\exists \threequery \in \threeqset,\ \dnude{\nu}{\empdistr{\mathcal{S}_1}{\threequery}}{\empdistr{\mathcal{S}_2}{\threequery}} > \epsilon/2 }
	\end{align*}
	where we used the triangular inequality 
	\begin{equation*}
		\dnude{\nu}{\empdistr{\mathcal{S}_1}{\threequery}}{\meanempdistr{\threequery}} - \dnude{\nu}{\empdistr{\mathcal{S}_2}{\threequery}}{\meanempdistr{\threequery}} \leq   \dnude{\nu}{\empdistr{\mathcal{S}_1}{\threequery}}{\empdistr{\mathcal{S}_2}{\threequery}}
	\end{equation*} 
\end{proof}







\begin{tcolorbox}
    \begin{definition}[Covering and Packing]
        Let $(\threeqset, d)$ be a metric space. 
        \begin{itemize}
            \item For any $\epsilon> 0$, a subset $\threeqset' \subseteq \threeqset$ is said to be $\epsilon$-separated if for all distinct $\threequery'_1, \threequery'_2 \in \threeqset'$, $d(\threequery'_1, \threequery'_2) > \epsilon$.
            \item The $\epsilon$-packing number on $(\threeqset, d)$, denoted by $\packing(\epsilon, \threeqset, d)$, is then defined as cardinality of the largest $\epsilon$-separated subset $\threeqset'$ of $\threeqset$.
        \end{itemize}
        Intuitively, the $\epsilon$-packing number is the maximal number of balls of radius $\epsilon/2$ that can fit into $\threeqset$ without intersecting.
        \begin{itemize}
            \item For any $\epsilon> 0$, a subset $\threeqset'$ of $\threeqset$ is said to be an $\epsilon$-cover of $\threeqset$ if for all $\threequery \in \threeqset$, there exists $\threequery' \in \threeqset'$ with $d(\threequery, \threequery') \leq \epsilon$.
            \item The $\epsilon$-covering number on $(\threeqset, d)$, denoted by $\covering(\epsilon, \threeqset, d)$, is then defined as cardinality of the smallest $\epsilon$-cover of $\threeqset$.
        \end{itemize}
        Intuitively, the $\epsilon$-covering number is the minimal number of balls of radius $\epsilon$ than can fill $\threeqset$, with possible overlaps.
    \end{definition}
\end{tcolorbox}

    One can easily check that for all $\epsilon>0$
    \begin{equation}
		\label{eqn__pack_cover_pack}
        \packing(2\epsilon, \threeqset, d) \leq \covering(\epsilon, \threeqset, d) \leq \packing(\epsilon, \threeqset, d)
    \end{equation}





\begin{tcolorbox}
    \begin{theorem}[\cite{pollard1984_convergence_stoch_proc} and \cite{haussler1995spherepacking}]
        \label{thm_pack}
        For any set $\mathcal{X}$, any probability distribution $\mu$ on $\mathcal{X}$, any
        set $\threeqset \subseteq [0,\bound]^{\mathcal{X}}$ of $\mu$-measurable positive functions on $\mathcal{X}$ bounded by some real $\bound$, and any $\epsilon\in]0,\bound]$, one has
        \begin{align*}
            \packing(\epsilon, \threeqset, \dlone{\mu}{}{}) 
			\leq \min \left\{
			2\left(\frac{2eM}{\epsilon}\log\frac{2e\bound}{\epsilon}\right)^{d'_{\threeqset}},\ 
			e(d'_{\threeqset}+1) \left(\frac{2e \bound}{\epsilon}\right)^{d'_{\threeqset}}\right\}
        \end{align*}
        where $d'_{\threeqset} =\operatorname{pdim}\threeqset$ is the pseudo-dimension of $\threeqset$.
    \end{theorem}
\end{tcolorbox}
Note that the second bound is better than the first one when $\epsilon$ is sufficiently small compared to $d'_{\threeqset}$ and vice versa.







\begin{tcolorbox}
	\begin{conjecture}
		\label{lem_infi_union_bound}
		Let $\epsilon >0$, $m\in \NN$, and  $\mathcal{S}_1, \mathcal{S}_2 \overset{i.i.d.}{\sim} \PP{}{}$, two sequences of size $m$ independently sampled from the same distribution supported on $\mathcal{X}^m$. Then
		
		\begin{gather*}
			\PP{}{\exists \threequery \in \threeqset,\ \dnude{\nu}{\empdistr{\mathcal{S}_1}{\threequery}}{\empdistr{\mathcal{S}_2}{\threequery}} > \epsilon} \\\leq \\
			\sup_{\mathcal{S} \in \mathcal{X}^{2m} } \covering(\epsilon/4, \threeqset, \dlone{\empdistr{\mathcal{S}}{}}{}{}) \sup_{\threequery \in \threeqset} \PP{}{\dnude{\nu}{\empdistr{\mathcal{S}_1}{\threequery}}{\empdistr{\mathcal{S}_2}{\threequery}} > \epsilon/4 }
		\end{gather*}
	\end{conjecture}
\end{tcolorbox}



\begin{proof}[Draft of Proof]
	Let $\mathcal{S}_1$ and $\mathcal{S}_2$ sampled such that $\exists \threequery \in \threeqset,\ \dnude{\nu}{\empdistr{\mathcal{S}_1}{\threequery}}{\empdistr{\mathcal{S}_2}{\threequery}} > \epsilon$. 
	
	We denote their concatenation by $\mathcal{S} := \mathcal{S}_1 \uplus \mathcal{S}_2$ which is of size $2m$. Let then be taken $\threeqset'_{\mathcal{S}}$, a minimal $\epsilon/4$-cover of $\threeqset$ for the $\dlone{\empdistr{\mathcal{S}}{}}{}{}$ topology, then $|\threeqset'_{\mathcal{S}}| = \covering(\epsilon/4, \threeqset, \dlone{\empdistr{\mathcal{S}}{}}{}{})$. We thus know there exists $\threequery' \in \threeqset'_{\mathcal{S}}$ such that $\dlone{\empdistr{\mathcal{S}}{}}{\threequery}{\threequery'} \leq \epsilon/4$.


	Then triangular inequalities yields that
	\begin{align*}
		\dnude{\nu}{\empdistr{\mathcal{S}_1}{\threequery}}{\empdistr{\mathcal{S}_2}{\threequery}} &\leq \dnude{\nu}{\empdistr{\mathcal{S}_1}{\threequery'}}{\empdistr{\mathcal{S}_2}{\threequery'}} + \dnude{\nu}{\empdistr{\mathcal{S}_1}{\threequery}}{\empdistr{\mathcal{S}_1}{\threequery'}} + \dnude{\nu}{\empdistr{\mathcal{S}_2}{\threequery}}{\empdistr{\mathcal{S}_2}{\threequery'}}  \\
		&\leq \dnude{\nu}{\empdistr{\mathcal{S}_1}{\threequery'}}{\empdistr{\mathcal{S}_2}{\threequery'}}  + \empdistr{\mathcal{S}_1}{\lvert \threequery-\threequery'\rvert} + \empdistr{\mathcal{S}_2}{\lvert\threequery-\threequery'\rvert}\\
		&\leq \dnude{\nu}{\empdistr{\mathcal{S}_1}{\threequery'}}{\empdistr{\mathcal{S}_2}{\threequery'}} + 2 \dlone{\empdistr{\mathcal{S}}{}}{\threequery}{\threequery'}\\
		\iff \dnude{\nu}{\empdistr{\mathcal{S}_1}{\threequery'}}{\empdistr{\mathcal{S}_2}{\threequery'}} &\geq \dnude{\nu}{\empdistr{\mathcal{S}_1}{\threequery}}{\empdistr{\mathcal{S}_2}{\threequery}} - 2  \dlone{\empdistr{\mathcal{S}}{}}{\threequery}{\threequery'} > \epsilon/2 
	\end{align*}
	Therefore
	\begin{equation*}
		\PP{}{\exists \threequery \in \threeqset,\ \dnude{\nu}{\empdistr{\mathcal{S}_1}{\threequery}}{\empdistr{\mathcal{S}_2}{\threequery}} > \epsilon} \leq \PP{}{\exists \threequery \in \threeqset'_{\mathcal{S}},\ \dnude{\nu}{\empdistr{\mathcal{S}_1}{\threequery}}{\empdistr{\mathcal{S}_2}{\threequery}} > \epsilon/4}
	\end{equation*}
	By the law of total expectation, we obtain
	\begin{align*}
		\PP{}{\exists \threequery \in \threeqset'_{\mathcal{S}},\ \dnude{\nu}{\empdistr{\mathcal{S}_1}{\threequery}}{\empdistr{\mathcal{S}_2}{\threequery}} > \epsilon/4}
		&=\EE{}{ \1 \{\exists \threequery \in \threeqset'_{\mathcal{S}},\ \dnude{\nu}{\empdistr{\mathcal{S}_1}{\threequery}}{\empdistr{\mathcal{S}_2}{\threequery}} > \epsilon/4 \} }\\
		&=\EE{}{ \PP{}{\exists \threequery \in \threeqset'_{\mathcal{S}},\ \dnude{\nu}{\empdistr{\mathcal{S}_1}{\threequery}}{\empdistr{\mathcal{S}_2}{\threequery}} > \epsilon/4 \mid \threeqset'_{\mathcal{S}}}  }\\
		&=\EE{}{ \PP{}{\bigcup_{\threequery \in \threeqset'_{\mathcal{S}}} \{ \dnude{\nu}{\empdistr{\mathcal{S}_1}{\threequery}}{\empdistr{\mathcal{S}_2}{\threequery}} > \epsilon/4 \mid \threeqset'_{\mathcal{S}}\} }  }\\		
        &\overset{?}{\leq} \sup_{\threeqset'_{\mathcal{S}}} |\threeqset'_{\mathcal{S}}| \sup_{\threequery \in \threeqset} \PP{}{\dnude{\nu}{\empdistr{\mathcal{S}_1}{\threequery}}{\empdistr{\mathcal{S}_2}{\threequery}} > \epsilon/4 }\\
		&= \sup_{\mathcal{S} \in \mathcal{X}^{2m} } N(\epsilon/4, \threeqset, \dlone{\empdistr{\mathcal{S}}{}}{}{} ) \sup_{\threequery \in \threeqset} \PP{}{\dnude{\nu}{\empdistr{\mathcal{S}_1}{\threequery}}{\empdistr{\mathcal{S}_2}{\threequery}} > \epsilon/4 }
	\end{align*}
$\overset{?}{\leq}$ indicates this inequality is still to be proven. It consists of bounding the union probability of random event other a random set. Since we can bound uniformly both, event probability, and set cardinality, it could seem intuitive this random union bound hold. However, the conditioning by $\threeqset'_{\mathcal{S}}$ makes this bound non trivial and require further assumptions.

\end{proof}


From there, we are now able to prove \cref{thm_infinite_union_bound}. 
\begin{proof}
	Let $\epsilon>0$ and take $m \in \NN$ such that $\boundrate(\epsilon/2, m) \leq 1/2$. Combining \cref{lem_symm} and \cref{lem_infi_union_bound} gives
	\begin{align*}
		\PP{}{ \exists \threequery \in \threeqset,\ \dnude{\nu}{\empdistr{\mathcal{S}_1}{\threequery}}{\meanempdistr{\threequery}} > \epsilon} 
		&\leq 2\PP{}{\exists \threequery \in \threeqset,\ \dnude{\nu}{\empdistr{\mathcal{S}_1}{\threequery}}{\empdistr{\mathcal{S}_2}{\threequery}} > \epsilon/2 }\\
		&\leq 2 \sup_{\mathcal{S} \in \mathcal{X}^{2m} } \covering(\epsilon/8, \threeqset, \dlone{\empdistr{\mathcal{S}}{}}{}{}) \sup_{\threequery \in \threeqset} \PP{}{\dnude{\nu}{\empdistr{\mathcal{S}_1}{\threequery}}{\empdistr{\mathcal{S}_2}{\threequery}} > \epsilon/8 }
	\end{align*}

In order to bound the last term, first consider applying the union bound
	\begin{align*}
		\PP{}{\dnude{\nu}{\empdistr{\mathcal{S}_1}{\threequery}}{\empdistr{\mathcal{S}_2}{\threequery}} > \epsilon/8 } 
		&\leq \PP{}{\dnude{\nu}{\empdistr{\mathcal{S}_1}{\threequery}}{\meanempdistr{\threequery}} + \dnude{\nu}{\meanempdistr{\threequery}}{\empdistr{\mathcal{S}_2}{\threequery}}> \epsilon/8 }\\
		&\leq \PP{}{\dnude{\nu}{\empdistr{\mathcal{S}_1}{\threequery}}{\meanempdistr{\threequery}} > \epsilon/16 \vee  \dnude{\nu}{\meanempdistr{\threequery}}{\empdistr{\mathcal{S}_2}{\threequery}}> \epsilon/16 }\\
		&\leq \PP{}{\dnude{\nu}{\empdistr{\mathcal{S}_1}{\threequery}}{\meanempdistr{\threequery}} > \epsilon/16 \vee  \dnude{\nu}{\meanempdistr{\threequery}}{\empdistr{\mathcal{S}_2}{\threequery}}> \epsilon/16 }\\
		&\leq 2\PP{}{\dnude{\nu}{\empdistr{\mathcal{S}_1}{\threequery}}{\meanempdistr{\threequery}} > \epsilon/16 }\\
		&\leq 2 \boundrate(\threeqset, \epsilon/16, m)
	\end{align*}
Second, we know from \cref{eqn__pack_cover_pack} and \cref{thm_pack}
\begin{align*}
	\covering(\epsilon/8, \threeqset, \dlone{\empdistr{\mathcal{S}}{}}{}{})
	\leq \packing(\epsilon/8, \threeqset, \dlone{\empdistr{\mathcal{S}}{}}{}{})
	&\leq 2\left(\frac{16eM}{\epsilon}\log\frac{16e\bound}{\epsilon}\right)^{d'_{\threeqset}}
	&\leq  2\left(\frac{8e\bound}{\epsilon}\right)^{2d'_{\threeqset}}
\end{align*}
where we used the fact that $a \log a < (a/2)^2$ whenever $a \geq 5$, which is the case for $\frac{8eM}{\epsilon} \geq 5$ since $\epsilon \in ]0,\bound]$.

The two precedent bound does neither depend on $\mathcal{S}$ nor $f$, and therefore 
\begin{align*}
	\PP{}{ \exists \threequery \in \threeqset,\ \dnude{\nu}{\empdistr{\mathcal{S}_1}{\threequery}}{\meanempdistr{\threequery}} > \epsilon} 
	&\leq 8\left(\frac{8e\bound}{\epsilon}\right)^{2d'_{\threeqset}}\boundrate(\epsilon/16, m)\\
\end{align*}
which is the desired result.

\end{proof}













\begin{tcolorbox}[colback=red!10,title= Useless?]
	We define $\forall a,b \geq 0$ and $\forall \nu >0$
	\begin{equation}
		\dnu{\nu}{a}{b} := \frac{|a-b|}{a+b+\nu}
	\end{equation}
	
	From that definition, one can easily check
	\begin{proposition}
		\label{prop_dnu}
		For all $\nu>0$, the function $d_\nu$ verifies the following properties
		\begin{itemize}
			\item $d_\nu$ is a distance i.e. it verifies positivity, separation, symmetry and triangular inequality.
			\item $\forall a,b \geq 0,\ 0\leq \dnu{\nu}{a}{b} < \min( \frac{|a-b|}{a}, 1)$
			\item Moreover, if $\exists \bound\geq 0$ such that $a,b \le \bound$, then 
			\begin{equation*}
				\frac{|a-b|}{\nu + 2\bound} \le \dnu{\nu}{a}{b} \le \frac{|a-b|}{\nu}
			\end{equation*}
		\end{itemize}
	\end{proposition}


    
    Using properties \cref{prop_dnu} of $\dnude{\nu}{}{}$ distance yields that
    \begin{align*}
        \dnude{\nu}{\empdistr{\mathcal{S}_1}{\query}}{\empdistr{\mathcal{S}_1}{\query^*}} + \dnude{\nu}{\empdistr{\mathcal{S}_2}{\query}}{\empdistr{\mathcal{S}_2}{\query^*}}
        &\le \frac{\lvert \empdistr{\mathcal{S}_1}{\query-\query^*}\rvert}{\nu} + \frac{\lvert \empdistr{\mathcal{S}_2}{\query-\query^*}\rvert}{\nu} \\
        &\le \frac{2}{\nu}  \dlone{\empdistr{\mathcal{S}}{}}{\query}{\query^*}
    \end{align*}
\end{tcolorbox}








Note that we could also be interested in a biased cost function such as the diversified risk introduced by \cite{zhang2017dppminibatch}
$$
\tilde L{\query} =\frac{1}{m}\EE{x \sim \textrm{mDPP}}{\query(x)}=\frac{1}{m}\sum_{x_i \in \mathcal X} b_{i} \query\left(x_{i}\right)
$$
Then an unbiased estimator of $\tilde L$ is
\begin{equation*}
	\hat{\tilde L}_{\textrm{mDPP}}{\query} = \frac{1}{m}\sum_{x_i\in \mathcal S} \query(x_i)
\end{equation*}
We can switch between $L$ and $\tilde L$, substituting $\query(x_i)$ by $\frac{b_i \query(x_i)}{m}$.
\note{}{complete}











\printbibliography
%	 \bibliographystyle{chicago}
%	 \bibliography{biblio.bib}








% \begin{tcolorbox}[colback=red!10,title= Useless?]
% 	We define $\forall a,b \geq 0$ and $\forall \nu >0$
% 	\begin{equation}
% 		\dnu{\nu}{a}{b} := \frac{|a-b|}{a+b+\nu}
% 	\end{equation}
	
% 	From that definition, one can easily check
% 	\begin{proposition}
% 		\label{prop_dnu}
% 		For all $\nu>0$, the function $d_\nu$ verifies the following properties
% 		\begin{itemize}
% 			\item $d_\nu$ is a distance i.e. it verifies positivity, separation, symmetry and triangular inequality.
% 			\item $\forall a,b \geq 0,\ 0\leq \dnu{\nu}{a}{b} < \min( \frac{|a-b|}{a}, 1)$
% 			\item Moreover, if $\exists \bound\geq 0$ such that $a,b \le \bound$, then 
% 			\begin{equation*}
% 				\frac{|a-b|}{\nu + 2\bound} \le \dnu{\nu}{a}{b} \le \frac{|a-b|}{\nu}
% 			\end{equation*}
% 		\end{itemize}
% 	\end{proposition}


    
%     Using properties \cref{prop_dnu} of $\dnude{\nu}{}{}$ distance yields that
%     \begin{align*}
%         \dnude{\nu}{\empdistr{\mathcal{S}_1}{\query}}{\empdistr{\mathcal{S}_1}{\query^*}} + \dnude{\nu}{\empdistr{\mathcal{S}_2}{\query}}{\empdistr{\mathcal{S}_2}{\query^*}}
%         &\le \frac{\lvert \empdistr{\mathcal{S}_1}{\query-\query^*}\rvert}{\nu} + \frac{\lvert \empdistr{\mathcal{S}_2}{\query-\query^*}\rvert}{\nu} \\
%         &\le \frac{2}{\nu}  \dlone{\empdistr{\mathcal{S}}{}}{\query}{\query^*}
%     \end{align*}
% \end{tcolorbox}




% Note that we could also be interested in a biased cost function such as the diversified risk introduced by \cite{zhang2017dppminibatch}
% $$
% \tilde L{\query} =\frac{1}{m}\EE{x \sim \textrm{mDPP}}{\query(x)}=\frac{1}{m}\sum_{x_i \in \mathcal X} b_{i} \query\left(x_{i}\right)
% $$
% Then an unbiased estimator of $\tilde L$ is
% \begin{equation*}
% 	\hat{\tilde L}_{\textrm{mDPP}}{\query} = \frac{1}{m}\sum_{x_i\in \mathcal S} \query(x_i)
% \end{equation*}
% We can switch between $L$ and $\tilde L$, substituting $\query(x_i)$ by $\frac{b_i \query(x_i)}{m}$.




% \textbf{Theorem 3.4.} from \cite{pemantle2011rayleighconcentration}: Let $\mathbb{P}$ be a k-homogeneous probability measure on $\mathcal{B}_{n}$ satisfying the Stochastic Covering Property (SCP). Let $f$ be a 1-Lipschitz function on $\mathcal{B}_{n}$. Then
% $$
% \mathbb{P}(\lvert f-\mathbb{E} f \rvert \geq a) \leq 2 \exp \left(-\frac{a^{2}}{8 k}\right)
% $$

% Bennett inequality: Let be $(X_i)_{i\in \intint{1}{n}}$ independent and centered real-valued random variables, and $\sigma^2 = \frac{1}{n}\sum_i \Var{}{X_i}$, then for any $t>0$
% $$
% \mathbb{P}\left(\sum_{i=1}^{n} X_{i}>t\right) \leq \exp \left(-n \sigma^{2} h\left(\frac{t}{n \sigma^{2}}\right)\right)
% $$
% where $h(u)=(1+u) \log (1+u)-u$ for $u \geq 0$.

% \section{Discrete OPE}
% \begin{tcolorbox}
% 	Can we bypass the Kernel Density Estimate (KDE) in SGD paper by using discrete OPE? See \cite{gautschi2004ope}.

% \end{tcolorbox}


% \section{Holydays questions}
% \begin{itemize}
% 	\item Variance for formula for k-DPP, in \cite{zhang2017dppminibatch}.
% 	\item How $\tilde K$ eigenspaces look like ? When $n \xrightarrow[]{} \infty$ ?
% 	\begin{itemize}
% 		\item How does it compare to \cite{bardenet2020mcdpp} ?
% 		\item If $f$ is given, can I find a $K$ for which $f$ is in "good" eigenspaces (eigenvalue $\geq$ 1).
% 	\end{itemize}
% 	\item Defining discrete OPE, because discretized continuous OPE is probably not a DPP. See Gautschi Orthogonal Polynomials, 2004.
% 	\begin{itemize}
% 		\item For making links with SGD paper \cite{bardenet2021sgddpp}
% 		\item Look at the limit e.g. for Jacobi ensembles. 
% 	\end{itemize}
% \end{itemize}
% \begin{itemize}
% 	\item Take a Bernoulli and beat it with a DPP.
% 	\item Focus on metric we could have advantages on, e.g. look how variance decay with coreset size. 
% 	\item Better with direct applications e.g. on k-means or linear regression.
% \end{itemize}
% \begin{itemize}
% 	\item Strong raylegh measure positive correlation ?
% 	\item Pemantle only for kDPP?
% 	\item Hoefding/Benett like proof for $\sum_{x_i \in \mathcal S} \epsilon_i f(x_i)$ using maybe recurisve property of HKPV sampling.
% \end{itemize}

% 	\vfill




\end{document}






