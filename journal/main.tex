\documentclass{article}

% if you need to pass options to natbib, use, e.g.:
%     \PassOptionsToPackage{numbers, compress}{natbib}
% before loading neurips_2021

% ready for submission
% \usepackage{style}
\usepackage[nonatbib]{style}

% to compile a preprint version, e.g., for submission to arXiv, add add the
% [preprint] option:
%     \usepackage[preprint]{neurips_2021}

% to compile a camera-ready version, add the [final] option, e.g.:
%     \usepackage[final]{neurips_2021}

% to avoid loading the natbib package, add option nonatbib:
%    \usepackage[nonatbib]{neurips_2021}

\usepackage[utf8]{inputenc} % allow utf-8 input
\usepackage[T1]{fontenc}    % use 8-bit T1 fonts
\usepackage{hyperref}       % hyperlinks
\usepackage{url}            % simple URL typesetting
\usepackage{booktabs}       % professional-quality tables
\usepackage{amsfonts}       % blackboard math symbols
\usepackage{nicefrac}       % compact symbols for 1/2, etc.
\usepackage{microtype}      % microtypography
\usepackage{xcolor}         % colors


\usepackage{amsmath}
\usepackage{graphicx}
\usepackage{subcaption}
\usepackage{dsfont}
\usepackage{amsthm}
\usepackage{txfonts}

\renewcommand{\epsilon}{\varepsilon}
\renewcommand{\phi}{\varphi}
\newcommand{\NN}{\mathbb{N}}
\newcommand{\RR}{\mathbb{R}}
\newcommand{\CC}{\mathbb{C}}
\newcommand{\PP}{\mathbb{P}}
\newcommand{\EE}{\mathbb{E}}
\newcommand{\Var}{\operatorname{\mathbb V ar}}
\newcommand{\Cov}{\operatorname{\mathbb C ov}}
%\newcommand{\EE}[2][]{\mathbb{E}_{#1}[#2]}
%\newcommand{\odds}{\mathbb{O\,}}
%\llbracket \rrbracket 
\newtheorem{theorem}{Theorem}
\theoremstyle{definition}
\newtheorem{definition}{Definition}[section]

\usepackage[style=authoryear-comp,maxnames=1,uniquelist=false, backend=bibtex]{biblatex} %backend=biber
\addbibresource{biblio.bib}




\title{Journal}




\begin{document}
	
	\maketitle
	
	\section{Introduction}
	
	Let $\mathcal{X}=\left\{\boldsymbol{x}_{1}, \ldots, \boldsymbol{x}_{n}\right\}$ be a set of $n$ datapoints. Let $\Theta$ be a space of parameters, and $\theta$ an element of $\Theta$. We consider cost functions of the form:
	$$
	L(\theta)=\sum_{\boldsymbol{x} \in \mathcal{X}} f(x, \theta)
	$$
	
	Let $\mathcal{S}=\left\{x_{s_{1}}, \ldots, x_{s_{m}}\right\}$ be a subset of $\mathcal{X}$ (possibly with repetitions). To each element $x_{s} \in \mathcal{S}$, associate a weight $\omega\left(x_{s}\right) \in \mathbb{R}^{+}$. Define the estimated cost associated to the weighted subset $\mathcal{S}$ as:
	$$
	\hat{L}(\theta)=\sum_{\boldsymbol{x}_{s} \in \mathcal{S}} \omega\left(\boldsymbol{x}_{s}\right) f\left(\boldsymbol{x}_{s}, \theta\right) .
	$$
	\begin{definition}[Coreset]
			Let $\epsilon \in {]}0,1{[}$. The weighted subset $\mathcal{S}$ is a $\epsilon$-coreset for $L$ if, for any parameter $\theta$, the estimated cost is equal to the exact cost up to a relative error:
		$$
		\forall \theta \in \Theta \quad\left|\frac{\hat{L}(\theta)}{L(\theta)}-1\right| \le \epsilon 
		$$
	\end{definition}

An important consequence of the coreset property is the following
$$
(1-\epsilon) L\left(\theta^{\text {opt }}\right) \le(1-\epsilon) L\left( \hat{\theta}^{\text {opt }}\right) \le \hat{L}\left( \hat{\theta}^{\text {opt }}\right) \le \hat{L}\left( \theta^{\text {opt }}\right) \le(1+\epsilon) L\left( \theta^{\text {opt }}\right)
$$

\begin{definition}[Sensitivity]
	The sensitivity $\sigma_i$ of a datapoint $\boldsymbol{x}_{i}$ and the total sensitivity $\mathfrak S$ of $\mathcal X$ are
	$$
	\begin{cases}
		\sigma_{i}=\sup _{\theta \in \Theta} \frac{f\left(\boldsymbol{x}_{i}, \theta\right)}{L(\mathcal{X}, \theta)} \quad \in[0,1]\\
			\mathfrak{S}=\sum_{i=1}^{n} \sigma_{i}
	\end{cases}
$$
\end{definition} 


\section{Variance argument}
\subsection{Multinomial case}
Multinomial case $\mathcal S \sim \mathcal M(m, q)$ i.e. $m$ independent categorical sampling where $\mathbb P(x_i)  = q(x_i)$

$$
	\operatorname{Var}[\hat L(\theta)] =\frac{1}{m} \operatorname{Var}\left[\frac {f_{\theta}(x)} {q(x)}\right] 
	=\frac{1}{m} \sum_{x \in \mathcal{X}} \frac{f_{\theta}(x)^{2}}{q(x)} -\frac{1}{m} L(\theta)^{2}
$$
For any query $\theta \in \Theta$, the variance is minimized by
$$
q_{\theta}(x)=\frac{ f_{\theta}(x)}{\sum_{x^{\prime} \in \mathcal{X}}  f_{\theta}\left(x^{\prime}\right)},
$$

\subsection{DPP case}
DPP case where $ \mathcal S \sim \mathcal{DPP}(K)$, \,$\pi_i := K_{ii}$. We have
$$
\operatorname{Var}[\hat L(\theta)]
=\sum_{i, j}\mathbb{E}\left[\epsilon_{i} \epsilon_{j}\right] \frac{f_\theta(x_{i}) f_\theta(x_{j})} {\pi_{i} \pi_{j}}  - L(\theta)^{2}
\quad \text{with} \quad
\mathbb{E}\left[\epsilon_{i} \epsilon_{j}\right]=
\begin{cases}
	\det\left(\mathrm{K}_{ij}\right)=\pi_{i} \pi_{j}-\mathrm{K}_{i j}^{2}, & \text{if } i \neq j \\
	\mathbb{E}\left[\epsilon_{i}\right]=\pi_{i}, & \text{if } i = j
\end{cases}
$$



Introducing $\Pi = \operatorname{diag}(\pi)$ and $\tilde K = \Pi^{-1}K^{\odot 2} \Pi^{-1}$, we can rewrite  
$$
\Var [\hat L(\theta)]=\sum_{i}\left(\frac{1}{\pi_{i}}-1\right) f_\theta(x_{i})^{2}-\sum_{i \neq j} \frac{\mathrm{K}_{i j}^{2}}{\pi_{i} \pi_{j}} f_\theta(x_{i}) f_\theta(x_{j}) =  f_\theta^\top (\Pi^{-1}  - \tilde{K}) f_\theta 
$$

	
For a Bernoulli process where $\mathbb P(x_i \in \mathcal S) = \pi_i$ independently, $K = \Pi$ then $\tilde K = I$. The DPP variance beats uniformly the Bernoulli process variance if $\tilde K$ dominates the identity i.e. 
$$ \forall f_\theta, \, \Var [\hat L_{K}(\theta)] < \Var [\hat L_{\Pi}(\theta)] \iff \tilde K \succ I$$

But $\tilde K$ is a symmetric positive definite matrix and by Hadamard inequality $\det( \tilde K) \leq \prod_{i} \tilde K_{ii}= 1$. Therefore at least one of its eigenvalue is lower than 1, hence $\tilde K \nsucc I$.
	
	
	
	
	\vfill
	
	
	\printbibliography
%	 \bibliographystyle{chicago}
%	 \bibliography{biblio.bib}
	
\end{document}