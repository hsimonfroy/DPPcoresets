\documentclass{article}

% if you need to pass options to natbib, use, e.g.:
%     \PassOptionsToPackage{numbers, compress}{natbib}
% before loading neurips_2021

% ready for submission
% \usepackage{style}
\usepackage[nonatbib]{style}

% to compile a preprint version, e.g., for submission to arXiv, add add the
% [preprint] option:
%     \usepackage[preprint]{neurips_2021}

% to compile a camera-ready version, add the [final] option, e.g.:
%     \usepackage[final]{neurips_2021}

% to avoid loading the natbib package, add option nonatbib:
%    \usepackage[nonatbib]{neurips_2021}

\usepackage[utf8]{inputenc} % allow utf-8 input
\usepackage[T1]{fontenc}    % use 8-bit T1 fonts
\usepackage{hyperref}       % hyperlinks
\usepackage{url}            % simple URL typesetting
\usepackage{booktabs}       % professional-quality tables
\usepackage{amsfonts}       % blackboard math symbols
\usepackage{nicefrac}       % compact symbols for 1/2, etc.
\usepackage{microtype}      % microtypography
\usepackage{xcolor}         % colors


\usepackage{amsmath}
\usepackage{graphicx}
\usepackage{subcaption}
\usepackage{dsfont}
\usepackage{amsthm}
\usepackage{txfonts}

\renewcommand{\epsilon}{\varepsilon}
\renewcommand{\phi}{\varphi}
\newcommand{\NN}{\mathbb{N}}
\newcommand{\RR}{\mathbb{R}}
\newcommand{\CC}{\mathbb{C}}
\newcommand{\PP}{\mathbb{P}}
\newcommand{\EE}{\mathbb{E}}
\newcommand{\Var}{\operatorname{\mathbb V ar}}
\newcommand{\Cov}{\operatorname{\mathbb C ov}}
%\newcommand{\EE}[2][]{\mathbb{E}_{#1}[#2]}
%\llbracket \rrbracket 
\newtheorem{theorem}{Theorem}
\theoremstyle{definition}
\newtheorem{definition}{Definition}[section]

\usepackage[style=authoryear-comp,maxnames=1,uniquelist=false, backend=biber]{biblatex} %backend=biber/bibtex
\addbibresource{biblio.bib}




\title{Journal}




\begin{document}
	
	\maketitle
	
	\section{Introduction}
	
	Let $\mathcal{X}=\left\{x_{1}, \ldots, x_{n}\right\}$ be a set of $n$ datapoints. Let $\Theta$ be a space of parameters, and $\theta$ an element of $\Theta$. We consider cost functions of the form:
	$$
	L(\theta)=\sum_{x \in \mathcal{X}} f(x, \theta)
	$$
	
	Let $\mathcal{S}=\left\{x_{s_{1}}, \ldots, x_{s_{m}}\right\}$ be a subset of $\mathcal{X}$ (possibly with repetitions). To each element $x \in \mathcal{S}$, associate a weight $\omega\left(x\right) \in \mathbb{R}^{+}$. Define the estimated cost associated to the weighted subset $\mathcal{S}$ as:
	$$
	\hat{L}(\theta)=\sum_{x \in \mathcal{S}} \omega\left(x\right) f\left(x, \theta\right) .
	$$
	\begin{definition}[Coreset]
			Let $\epsilon \in {]}0,1{[}$. The weighted subset $\mathcal{S}$ is a $\epsilon$-coreset for $L$ if, for any parameter $\theta$, the estimated cost is equal to the exact cost up to a relative error:
		\begin{equation}
			\forall \theta \in \Theta \quad\left|\frac{\hat{L}(\theta)}{L(\theta)}-1\right| \le \epsilon 
		\end{equation}
	\end{definition}

An important consequence of the coreset property is the following
$$
(1-\epsilon) L\left(\theta^{\text {opt }}\right) \le(1-\epsilon) L\left( \hat{\theta}^{\text {opt }}\right) \le \hat{L}\left( \hat{\theta}^{\text {opt }}\right) \le \hat{L}\left( \theta^{\text {opt }}\right) \le(1+\epsilon) L\left( \theta^{\text {opt }}\right)
$$

See \cite{bachem2017coresetML}.

\section{Variance argument}
\subsection{Multinomial case}
Multinomial case $\mathcal S \sim \mathcal M(m, q)$ i.e. $m$ independent categorical sampling where $\mathbb P(x_i)  = q(x_i)$

\begin{equation}
	\Var[\hat L(\theta)] =\frac{1}{m} \Var\left[\frac {f_{\theta}(x)} {q(x)}\right] 
	=\frac{1}{m} \sum_{x \in \mathcal{X}} \frac{f_{\theta}(x)^{2}}{q(x)} -\frac{1}{m} L(\theta)^{2}
\end{equation}

For any query $\theta \in \Theta$, the variance is reduced to 0 by
$$
q_{\theta}(x):=\frac{ f_{\theta}(x)}{\sum_{x^{\prime} \in \mathcal{X}}  f_{\theta}\left(x^{\prime}\right)}
$$

\subsection{DPP case}
DPP case where $ \mathcal S \sim \mathcal{DPP}(K)$, \,$\pi_i := K_{ii}$. We have
$$
\Var[\hat L(\theta)]
=\sum_{i, j}\mathbb{E}\left[\epsilon_{i} \epsilon_{j}\right] \frac{f_\theta(x_{i}) f_\theta(x_{j})} {\pi_{i} \pi_{j}}  - L(\theta)^{2}
\quad \text{with} \quad
\mathbb{E}\left[\epsilon_{i} \epsilon_{j}\right]=
\begin{cases}
	\det\left(K_{\{i, j\}}\right)=\pi_{i} \pi_{j}-K_{i j}^{2}, & \text{if } i \neq j \\
	\mathbb{E}\left[\epsilon_{i}\right]=\pi_{i}, & \text{if } i = j
\end{cases}
$$



Introducing $\Pi = \operatorname{diag}(\pi)$ and $\tilde K = \Pi^{-1}K^{\odot 2} \Pi^{-1}$, we can rewrite  

\begin{equation}
	\Var [\hat L(\theta)]=\sum_{i}\left(\frac{1}{\pi_{i}}-1\right) f_\theta(x_{i})^{2}-\sum_{i \neq j} \frac{K_{i j}^{2}}{\pi_{i} \pi_{j}} f_\theta(x_{i}) f_\theta(x_{j}) =  f_\theta^\top (\Pi^{-1}  - \tilde{K}) f_\theta 
\end{equation}


	
For a Bernoulli process where $\mathbb P(x_i \in \mathcal S) = \pi_i$ independently, $K = \Pi$ then $\tilde K = I$. The DPP variance beats uniformly the Bernoulli process variance if $\tilde K$ dominates the identity i.e. 
\begin{equation}
	\forall f_\theta, \, \Var [\hat L_{K}(\theta)] < \Var [\hat L_{\Pi}(\theta)] \iff \tilde K \succ I
\end{equation}

But $\tilde K$ is a symmetric positive definite matrix and by Hadamard inequality $\det( \tilde K) \leq \prod_{i} \tilde K_{ii}= 1$. Therefore at least one of its eigenvalue is lower than 1, hence $\tilde K \nsucc I$.


\subsection{m-DPP case}

The marginals $b_{i} \equiv \mathbb{E}\left[m_{i}\right]$ have an analytic form. Moreover, let be defined
$$
C_{i j}=\frac{\mathbb{E}\left[\left(m_{i}-b_{i}\right)\left(m_{j}-b_{j}\right)\right]}{\mathbb{E}\left[m_{i}\right] \mathbb{E}\left[m_{j}\right]}=\frac{\mathbb{E}\left[m_{i} m_{j}\right]}{b_{i} b_{j}}-1
$$

\begin{equation}
	\operatorname{Var}\left(g^{*}\right)=\frac{1}{m^{2}} \sum_{i=1}^{N}\left(b_{i}-b_{i}^{2}\right) f_\theta(x_i)^2
	+\frac{1}{m^{2}} \sum_{i \neq j} C_{i j} b_{i} b_{j} f_\theta(x_i)^{\top} f_\theta(x_j)
\end{equation}

So \cite{zhang2017dppminibatch} assume $\forall i\neq j, \quad  C_{ij}f_\theta(x_i)f_\theta(x_j) <0$

\section{Sensitivity}
\begin{definition}[Sensitivity]
	The sensitivity $\sigma_i$ of a datapoint $\boldsymbol{x}_{i}$ and the total sensitivity $\mathfrak S$ of $\mathcal X$ are
	$$
	\begin{cases}
		\sigma_{i}=\sup_{\theta \in \Theta} q_{\theta}(x_i) = \sup _{\theta \in \Theta} \frac{f_{\theta}\left(\boldsymbol{x}_{i}\right)}{L(\theta)} \quad \in[0,1]\\
		\mathfrak{S}=\sum_{i=1}^{n} \sigma_{i}
	\end{cases}
	$$
\end{definition} 

Let $s$ be an upper bound on sensitivity $\sigma$ i.e. $\forall i, s_i \geq \sigma_i$, and $S := \sum_{i=1}^n s_i$. Furthermore, let sample  $\mathcal S \sim \mathcal M(m, s/S)$, the multinomial sampling case. Define $g_\theta(x_i) := \frac{q_\theta(x_i)}{s_i}  \quad \in[0,1]$

By Hoeffding's inequality, we thus have for any $\theta \in \Theta$ and $\varepsilon^{\prime}>0$
$$
\mathbb{P}\left[\left|\mathbb{E}\left[g_{\theta}(x)\right]-\frac{1}{m} \sum_{x \in \mathcal{S}} g_{Q}(x)\right|>\varepsilon^{\prime}\right] \leq 2 \exp \left(-2 m \varepsilon^{\prime 2}\right) .
$$
By definition, $\mathbb{E}\left[g_{\theta}(x)\right]=\frac{1}{S}$ and $\frac{1}{m} \sum_{x \in \mathcal{C}} g_{Q}(x)=\frac{\operatorname{cost}(\mathcal{C}, Q)}{S \operatorname{cost}(\mathcal{X}, Q)}$. As such, for any $Q \in \mathcal{Q}$
$$
\mathbb{P}\left[|\operatorname{cost}(\mathcal{X}, Q)-\operatorname{cost}(\mathcal{C}, Q)|>\varepsilon^{\prime} S \operatorname{cost}(\mathcal{X}, Q)\right] \leq 2 \exp \left(-2 m \varepsilon^{\prime 2}\right)
$$
Hence, the set $\mathcal{C}$ satisfies the coreset property in (2.2) for any single query $Q \in \mathcal{Q}$ and $\varepsilon>0$ with probability at least $1-\delta$, if we choose
$$
m \geq \frac{S^{2}}{2 \varepsilon^{2}} \log \frac{2}{\delta}
$$




\section{SGD Paper}


\section{Pending questions}
\begin{itemize}
	\item Variance for formula for k-DPP, in \cite{zhang2017dppminibatch}.
	\item How $\tilde K$ eigenspaces look like ? When $n \xrightarrow[]{} \infty$ ?
	\begin{itemize}
		\item How does it compare to \cite{bardenet2020mcdpp} ?
		\item If $f$ is given, can I find a $K$ for which $f$ is in "good" eigenspaces (eigenvalue $\geq$ 1).
	\end{itemize}
	\item Defining discrete OPE, because discretized continuous OPE is probably not a DPP. See Gautschi Orthogonal Polynomials, 2004.
	\begin{itemize}
		\item For making links with SGD paper \cite{bardenet2021sgddpp}
		\item Look at the limit e.g. for Jacobi ensembles. 
	\end{itemize}
\end{itemize}

\vspace{1cm}

\begin{itemize}
	\item Take a Bernoulli and beat it with a DPP.
	\item Focus on metric we could have advantages on, e.g. look how variance decay with coreset size. 
	\item Better with direct applications e.g. on k-means or linear regression.
\end{itemize}
	
	\vfill
	
	
	\printbibliography
%	 \bibliographystyle{chicago}
%	 \bibliography{biblio.bib}
	
\end{document}